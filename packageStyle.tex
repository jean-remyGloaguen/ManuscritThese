%%%%%% MISE EN PAGES %%%%%%
\usepackage{geometry}
\geometry{outer=2cm,inner=3cm,top=3cm}

\setcounter{tocdepth}{3}     % Dans la table des matieres
\setcounter{secnumdepth}{3}  % Avec un numero.
\usepackage{setspace}

\usepackage{fancyhdr}	% marge en haut et en bas
\pagestyle{fancy}

\fancyhead{}	% vide l'entête
\fancyfoot{} % vide le pied~de~page

\fancyhead[RO]{\leftmark}
\fancyhead[LE]{\rightmark}
\fancyfoot[C]{\thepage}	% numéro de page en bas au centre

\renewcommand{\headrulewidth}{0.4pt} % épaisseur du trait en haut
\renewcommand{\footrulewidth}{0.4pt} % épaisseur du trait en bas

\fancypagestyle{mypagestyle}{%
    \fancyhead{}	
    \fancyfoot{} 
    \fancyfoot[C]{\thepage}
    \renewcommand{\headrulewidth}{0.4pt} 
	\renewcommand{\footrulewidth}{0.4pt} 
}

\usepackage{layout}
\usepackage{tocbibind} % include tableofcontent in itself

%%%%%% SYMBOLES %%%%%
\usepackage{tipa}	% pour avoir l'accent concave
\usepackage{lmodern}	% pour les guillemets

%%%%%% EQUATION %%%%%%
\usepackage{amssymb}
\usepackage{amsmath}
\usepackage{fancybox}
\usepackage{xfrac}	% fraction de type "1/4"
\usepackage{cases}	% système équation
\usepackage[overload]{empheq}
\usepackage{bm}		% pour mettre en gras .
\usepackage{units} 	% x/y barre latérale pour les fractions

%%%%%% FIGURE %%%%%%
\usepackage{graphicx}	% insérer des graphiques
\usepackage{subfigure}	% utiliser subfigure
\usepackage{float}	% utiliser H dans les figures

%%%%%% TABLEAUX %%%%%%
\usepackage{array,multirow,makecell}
%\addto\captionsfrench{\def\tablename{\textsc{Tableau}}}% pour avoir TABLEAU et pas TABLE dans les légendes des tableaux
\usepackage[table,xcdraw]{xcolor} % pour avoir des lignes colorées dans les tableau
\usepackage{slashbox} % pour les \backslashbox
%\usepackage{subcaption}
\usepackage{hhline}	% pour les lignes horizontales 
\usepackage{tabularx} % permet itemize dans les cellules


\newcolumntype{L}[1]{>{\raggedright\let\newline\\\arraybackslash\hspace{0pt}}m{#1}}
\newcolumntype{C}[1]{>{\centering\let\newline\\\arraybackslash\hspace{0pt}}m{#1}}
\newcolumntype{R}[1]{>{\raggedleft\let\newline\\\arraybackslash\hspace{0pt}}m{#1}}


%%%%% BIBLIO %%%%%
\usepackage[fixlanguage]{babelbib}
\selectbiblanguage{french}

%%%%% APPENDICES %%%%%%%
\usepackage[toc,page]{appendix}

%%%%%%%%%%%%%%%%%%%%%
\usepackage{url}	% gérer les adresses www.
\linespread{1.2}	% interligne

\cleardoublepage