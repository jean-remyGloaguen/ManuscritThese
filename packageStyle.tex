%%%%%% MISE EN PAGES %%%%%%
\usepackage{geometry}
\geometry{outer=2cm,inner=3cm,top=3cm}

\setcounter{tocdepth}{3}     % Dans la table des matieres
\setcounter{secnumdepth}{3}  % Avec un numero.
\usepackage{setspace}

\usepackage{fancyhdr}	% marge en haut et en bas
\pagestyle{fancy}

\fancyhead{}	% vide l'entête
\fancyfoot{} % vide le pied~de~page

\fancyhead[RO]{\leftmark}
\fancyhead[LE]{\rightmark}
\fancyfoot[C]{\thepage}	% numéro de page en bas au centre

\renewcommand{\headrulewidth}{0.4pt} % épaisseur du trait en haut
\renewcommand{\footrulewidth}{0.4pt} % épaisseur du trait en bas

\fancypagestyle{mypagestyle}{%
    \fancyhead{}	
    \fancyfoot{} 
    \fancyfoot[C]{\thepage}
    \renewcommand{\headrulewidth}{0.4pt} 
	\renewcommand{\footrulewidth}{0.4pt} 
}

\fancypagestyle{couvertureAbstract}{%
    \fancyhead{}	
    \fancyfoot{} 
    \fancyfoot[C]{}
	\renewcommand{\headrulewidth}{0pt} 
	\renewcommand{\footrulewidth}{0pt} 
}
%
\usepackage{layout}
\usepackage{tocbibind} % include tableofcontent in itself

%%%%%% PAGE DE GARDE %%%%%%

\geometry{outer=2cm,inner=3cm,top=3cm}
\usepackage[scaled]{helvet} % font used on cover (Helvetica)
\usepackage{eso-pic} % to set background picture
\usepackage{multicol} % for back cover (abstracts)
\usepackage{graphicx} % to include logos
\usepackage{tikz} % to compose background picture

% Colors (extracted from SPI's template)
\definecolor{boxcolor1}{rgb}{0.91373,0.92941,0.87451}
\definecolor{boxcolor2}{rgb}{0.94902,0.93333,0.91373}
\definecolor{boxcolor3}{rgb}{0.76078,0.87843,0.17647}
\definecolor{headercolor}{rgb}{0.94118,0.30980,0.17255}
\definecolor{namecolor}{rgb}{1.0,0.4,0.0}
\definecolor{titlecolor}{rgb}{0.19216,0.51765,0.60784}
% Also used: gray, teal (predefined by xcolor package, usually loaded by document class)

% Cover environment, to keep changes local
\newenvironment{cover}{%
  \fontfamily{phv}\selectfont % Select Helvetica font
  \pagestyle{empty} % No page number
}{
  \addtocounter{page}{-1}
  \cleardoublepage
}

% Macro for background common to front and back
\newcommand{\tikzBG}{%
  \path (0,0) rectangle (1,1);
  %TODO: You should adjust the bottom height of the following rectangle to fit your abstract's length
  \path [fill=boxcolor1] (.0571,.11) rectangle (.481,.963); 
  \path [fill=boxcolor2] (.4333,.697) rectangle (.9048,.7475);
  \path [fill=boxcolor2] (.4333,.7811) rectangle (.9048,.8316);
  \path [fill=boxcolor2] (.4333,.8687) rectangle (.9048,.9192);
  \path [fill=boxcolor3] (.0571,.7879) rectangle (.5762,.8316);
  \node[inner sep=0pt] at (0.2285,0.8788) [above left] {%
    \includegraphics[height=.0707\paperheight,keepaspectratio]{./figures/logo/logo_unb.png}};
  \node[inner sep=0pt] at (0.6667,0.8788) [above right] {%
    \includegraphics[height=.0808\paperheight,keepaspectratio]{./figures/logo/logo_ecn_color.png}};
  \node at (.0571,.8316) [above right,color=headercolor] {%
    \fontsize{29}{35}\selectfont\bfseries Th\`ese de Doctorat};
}

% Macro for repeated information (to avoid insconsistency)
%TODO: fill in with no formatting but desired case
\newcommand{\firstName}{Jean-Rémy}
\newcommand{\surname}{Gloaguen}
\newcommand{\thesisTitle}{Estimation du niveau sonore de sources d'intérêts au sein de mixtures sonores urbaines : application au trafic routier}

%%%%%%% SYMBOLES %%%%%
\usepackage{tipa}	% pour avoir l'accent concave
\usepackage{lmodern}	% pour les guillemets
\usepackage{gensymb}	% pour les degrés
\usepackage{enumitem}	% pour changer le symbole de l'item (\begin{itemize}[label=$\bullet$])

%%%%%%% EQUATION %%%%%%
\usepackage{amssymb}
\usepackage{amsmath}
\usepackage{fancybox}
\usepackage{xfrac}	% fraction de type "1/4"
\usepackage{cases}	% système équation
\usepackage[overload]{empheq}
\usepackage{bm}		% pour mettre en gras .
\usepackage{units} 	% x/y barre latérale pour les fractions
%
%%%%%%% FIGURE %%%%%%
\usepackage{subfigure}	% utiliser subfigure
\usepackage{float}	% utiliser H dans les figures
%
%%%%%% TABLEAUX %%%%%%
\usepackage{array,multirow,makecell}
%\addto\captionsfrench{\def\tablename{\textsc{Tableau}}}% pour avoir TABLEAU et pas TABLE dans les légendes des tableaux
\usepackage{colortbl} % pour avoir des lignes colorées dans les tableau
%\usepackage{slashbox} % pour les \backslashbox
%\usepackage{subcaption}
\usepackage{hhline}	% pour les lignes horizontales 
\usepackage{tabularx} % permet itemize dans les cellules
\usepackage{booktabs}
\usepackage{longtable}	% pour les tableaux longs

\newcolumntype{L}[1]{>{\raggedright\let\newline\\\arraybackslash\hspace{0pt}}m{#1}}
\newcolumntype{C}[1]{>{\centering\let\newline\\\arraybackslash\hspace{0pt}}m{#1}}
\newcolumntype{R}[1]{>{\raggedleft\let\newline\\\arraybackslash\hspace{0pt}}m{#1}}

%%%%% ALGORITHME %%%%%
\usepackage{algorithm}
\usepackage{algorithmic}

%%%%% BIBLIO %%%%%
\usepackage[fixlanguage]{babelbib}
\selectbiblanguage{french}
\usepackage{breakcites}	% pour couper les références en bout de ligne

%%%%% APPENDICES %%%%%%%
\usepackage[toc,page]{appendix}

%%%%%%%%%%%%%%%%%%%%%
\usepackage{url}	% gérer les adresses www.
\linespread{1.2}	% interligne

\newcommand{\ml}[1]{\textcolor{red}{ML : #1}}

\cleardoublepage