\begin{table}[h]
\centering
\begin{tabular}{m{5cm} c |m{5cm} c}
\hline
\toprule
\textbf{Classe de son} & \textbf{Nombre} & \textbf{Classe de son} & \textbf{Nombre} \\
\midrule
Aboiement de chien & 34 & Porte de voiture & 5\\ 
Balais & 6 & Roulement de valise & 5 \\ 
Bruit de chantier (marteau, perceuse \dots) & 12 & Sirène & 9 \\ 
Bruit de rue & 24 & Sonnette & 5 \\ 
Camion & 4 & Toussotement & 7\\ 
Cloches d'églises & 8 & Train & 7 \\ 
Klaxon & 24 & Tram & 7 \\
Oiseaux & 30 & Voiture à l'arrêt & 7 \\ 
Orage & 3 & Voiture Ville & 28 \\ 
Pas dans la ville & 11 & Voiture Route & 16 \\ 
Pas dans un parc & 16 & Voix (rire, 1 ou 2 mots) & 24 \\
Porte de maison & 5 &  \textbf{Total} & \textbf{321}\\ 
\bottomrule
\end{tabular}
\caption{Composition de la base de données pour les évènements sonores}
\label{tab:dataBaseEv}
\end{table}

\begin{table}[h]
\centering
\begin{tabular}{m{5cm} c |m{5cm} c}
\toprule
\textbf{Classe de son} & \textbf{Nombre} & \textbf{Classe de son} & \textbf{Nombre} \\ \midrule
Brouhaha de foule & 15 & Pluie & 14 \\ 
Brouhaha parc & 25 & Trafic routier & 9 \\
Chantier & 28 & Vent dans les arbres & 15 \\ 
Cours de récréation & 12 & Ventilation & 10 \\ 
Oiseaux & 25 & \textbf{Total} & \textbf{153} \\ 
\bottomrule
\end{tabular}
\caption{Composition de la base de données pour les bruits de fond}
\label{tab:dataBaseBcg}
\end{table}
