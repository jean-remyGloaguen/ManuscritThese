\chapter*{Conclusions générales et perspectives}
\label{chap:concl}
\addcontentsline{toc}{chapter}{Conclusions générales et perspectives}

\ml{l'objectif de cette these a été de répondre au défi de l'évaluation et de la calibration, aujourd'hui plebicisté par la communaute et considérée comme nécessaire, des outils de production de carte de bruit a ce jour uniquement basé sur .... Cette conforntation avec la mesure est a ce jour rendue possible par l'avenement des reseaux.... Repondre a ce defi posent plusieurs problméatiques scientifiques auquel nous avons apporter une premiere reponse. Comment séparer la composante traffic ? Comment evalauer la pertinence de cette spéaration ? Repondre a ces questions a necessiter la mise en place d'outils de traitement de l'information sonore issue de bla bla bla, la constitution de bases de donnees controles, la mise en place d'un protocole experimental innovant ... }

L'objectif de ces travaux de thèse a été de proposer une méthode permettant de déterminer le niveau sonore du trafic routier dans des scènes sonores urbaines mesurées. Pour cela, une technique de séparation de source a été choisie : la Factorisation en Matrices Non-négatives.
Cette méthode présente l'intérêt d'être adaptée aux capteurs monophoniques, qui équipent les capteurs des réseaux ou bien encore ceux des smartphones, et de naturellement prendre en compte le recouvrement des sources sonores, phénomène récurrent en ville.
La principale difficulté réside dans la diversité des sources sonores présentes en ville et aux différentes ambiances sonores auxquelles la méthode doit faire face : de l'ambiance \textit{Parc}, avec une prédominance de voix et d'oiseaux avec une faible présence du trafic routier, à l'ambiance \textit{Rue très bruyante}, où celui-ci est la source sonore principale.
Afin d'évaluer ces performances, l'application de la NMF a été testée non pas sur des enregistrements audio mais sur des scènes sonores simulées qui permettent de contrôler la présence des classes de sons, leurs niveaux sonores et leur occurrence. Ce choix a notamment permis de connaitre le niveau sonore du trafic exact qui est ainsi comparé au niveau sonore estimé par la NMF.

\section*{Bilan de la thèse}
Afin de trouver une forme optimale de la NMF, de nombreux \ml{aspects et formes : ne veut rien dire, aspects algorithmiques et de paramétrisation} de cette méthode ont été abordés  à travers la composition du dictionnaire (nombre d'éléments, dimension des trames temporelles) basé sur des enregistrements de passages de voitures, le choix de la $\beta$ divergence ou la forme de la NMF à travers:
\begin{itemize}
\item la NMF supervisée \cite{lee_learning_1999,fevotte_algorithms_2011}, qui est l'approche la plus simple, composée d'un dictionnaire fixe $ \mathbf{W}$ constitué d'éléments \textit{trafic} et où seul la matrice d'activation $\mathbf{H}$ est mise à jour,
\item la NMF semi-supervisée \cite{lee_semi-supervised_2010,kitamura2014music}, qui inclut dans son  dictionnaire une partie fixe $\mathbf{W_s}$ composé de spectres sonores du \textit{trafic} et d'une partie libre $\mathbf{W_r}$ mise à jour pouvant intégrer d'autres sources sonores,
\item la NMF initialisée seuillée, développée dans le cadre de ces travaux, qui consiste en l'apprentissage supervisé d'un dictionnaire initial, $\mathbf{W_0}$, composé d'éléments reliés à la source \textit{trafic}. Ce dictionnaire est ensuite mis à jour sur la scène testée. Chaque élément du dictionnaire obtenu, $\mathbf{W'}$, est alors comparé à son état initial dans $\mathbf{W_0}$. Par une technique de seuillage dur, les éléments les plus similaires à leurs états initiaux, et donc susceptibles d'être encore liés au trafic, sont alors conservés et permettent de déterminer la composante \textit{trafic} du signal.
\end{itemize}

L'ensemble de ces méthodes ont été comparés au regard d'une méthode de référence permettant de contextualiser l'intérêt de ces méthodes. Considérant que le traffic est majoritairement composé spectralement de basses fréquences, nous avons adopté une approche heuristique constituée d'un filtre passe-bas de fréquence de coupure $f_c$ \ml{donner la valeur}. %Si cette approche paraissait simple, elle s'est révélée performante sur l'ensemble des corpus.\\

L'étude de ces méthodes sur un premier corpus, le corpus \textit{Ambiance}, \ml{rappeler le caractere experimental de ce orpus} soulève la difficulté d'obtenir une méthode adaptée à l'ensemble à des différentes classes de son avec une présence de la composante trafic variable.
L'utilisation de la NMF SUP, constituée d'un dictionnaire \textit{trafic}, échoue à obtenir des estimations satisfaisantes du niveau sonore du trafic lorsque celui-ci est peu présent en confondant les classes de sons \textit{trafic} et \textit{interférante} mais elle devient performante lorsque le bruit du trafic devient prépondérant.
À l'inverse, l'approche basée sur la NMF semi-supervisée, par l'ajout d'éléments libres dans le dictionnaire, se trouve être performante quand le trafic est peu présent en y intégrant la classe de son interférant, laissant le dictionnaire fixe, composé d'éléments \textit{trafic}, modéliser cette source.
Mais, elle devient défaillante lorsque cette source devient principale en considérant dans cette partie libre, des composantes liées au trafic routier. Les degrés de libertés ajoutés sont ainsi, suivant la présence du trafic, un atout ou une faiblesse : avec peu de trafic, ces éléments intègrent facilement les composantes de la classe de son \textit{interférante} alors que face à des mixtures sonores composées principalement de la source \textit{trafic}, celle-ci est libre d'y inclure cette composante, diminuant les performances de la méthode.
La NMF initialisée seuillée semble alors être la méthode qui offre les meilleures estimations moyennes même si cette méthode se révèle n'être jamais la plus efficace selon les différentes valeurs du $TIR$ (\textit{Traffic Interfering Ratio}, le rapport des niveaux sonores de la composante \textit{trafic} et de la composante \textit{interférante}). Par la mise à jour de son dictionnaire initial $\mathbf{W_0}$ et au contrôle réalisé par le seuillage, cette méthode s'adapte à chacune des scènes et permet de résoudre la question de la généralisation du dictionnaire. De plus, cette approche permet, contrairement à la NMF SUP et SEM, de modéliser les signaux tels que captés par un microphone en raison de la mise à jour du dictionnaire. Ainsi, l'impact de l'environnement urbain sur la propagation du son peut être considéré et ainsi être plus généralisable que les deux autres méthodes qui contiennent un dictionnaire fixe. La NMF IS montre toutefois des limites face aux classes de sons ayant des allures spectrales similaires comme les évènements mécaniques (bruit de chantier, ventilation) et climatiques (pluie ou orage) où elle ne permet pas de différencier suffisamment ces évènements du trafic. Concernant les évènements climatiques, la présence de capteurs météorologiques dans les systèmes embarqués permettrait d'éviter de prendre en compte les mesures acoustiques en présence de pluie ou d'orage par exemple.

Les tests menés sur le corpus \textit{SOUR}, \ml{rappeler le caractere plus realiste de ce corpus} confirme les conclusions faites : la NMF IS est la méthode qui s'adapte le mieux aux différents ESU avec en moyenne une erreur gloable $MAE_g$ = 1,16 ($\pm$ 0,86) pour une distance euclidienne, un nombre d'éléments $K = 300$, une fenêtre temporelle $w_t$ = 1 s et un seuil dur $t_h$ = 0,35. Cette combinaison obtenue peut ainsi être considérée en vue de déterminer le niveau sonore du trafic par des mesures faites en ville grâce à la qualité des scènes sonores simulées. En effet, le test perceptif mené sur une partie des scènes du corpus sur un panel de 50 auditeurs a permis d'évaluer et de valider le critère de \og réalisme \fg{} de ces scènes simulées permettant de les assimiler à des enregistrements sonores. L'erreur réalisée par la NMF sur le copus \textit{SOUR} peut alors être celle qui serait faites sur des enregistrements sonores urbains.
L'ajout de contraintes (régularité temporelle et de parcimonie) sur la NMF IS n'a pas permis d'améliorer ces résultats. Celles-ci ont surtout montré un impact pour la NMF SEM où appliquées sur la partie fixe ou mobile du dictionnaire, elles ont pu améliorer l'estimation du niveau sonore du trafic sans toutefois surpasser la NMF IS. Enfin, l'utilisation de seuils adaptatifs à chaque ambiance sonore ne permet pas d'améliorer significativement ces estimations.
En l'état actuel, les différents résultats obtenus permettent donc de privilégier la NMF IS comme l'approche la mieux adaptée aux environnements sonores urbains en vue d'estimer le niveau sonore du trafic.


\section*{Perspectives de recherches}

Les travaux réalisés durant cette thèse ont naturellement permis d'appréhender plusieurs aspects, dont certains n'ont pas pu être approfondis ou étudiés.

\ml{plutot dire ce que ta contribution valide par rapport au defi initial, ensuite proposer des pistes pour ce qui manque encore}

\subsection*{Amélioration de la NMF}

Les pistes d'amélioration de la NMF, abordées en fin de thèse, nécessiteraient d'être approfondies notamment celles liées à la contrainte de parcimonie.
En plus de ces contraintes, la prise en compte des effets de la propagation du son par l'ajout de filtre de propagation peut être une piste d'étude intéressante. Si pour la NMF IS, ce filtre ne semble pas nécessaire, puisque cet aspect est pris en compte par la mise à jour du dictionnaire, il pourrait être considéré pour la NMF SUP et SEM en vue de leur offrir une meilleure adaptabilité aux différentes scènes sonores urbaines. Il peut être envisageable soit de dupliquer plusieurs fois un dictionnaire fixe et de les filtrer par différentes filtres de propagation afin d'obtenir plusieurs formes de dictionnaires soit de contraindre les mises à jour de $\mathbf{W}$ à suivre l'effet d'un filtre de propagation afin de conserver des éléments \textit{trafic}.
Ensuite, dans le cadre de la NMF IS, il peut être envisageable de construire cette méthode sous une forme différente qui serait basée sur la NMF SEM : le dictionnaire pourrait être composé d'un dictionnaire initial $\mathbf{W_0}$ auquel serait ajouté quelques éléments libres en plus. L'ensemble du dictionnaire serait alors mis à jour et les éléments \textit{trafic} seraient extraits par le même processus de seuillage. L'ajout de contrainte, notamment de parcimonie, sur ces éléments libres, pourrait aider la focalisation des mises à jour vers la source d'intérêt ou éviter que les deux éléments prennent la forme de spectres \textit{trafic}.

\subsection*{Validation des conclusions sur d'autres corpus de sons}
La NMF IS  est la méthode qui a montré les meilleures performances pour estimer le niveau sonore du trafic.  Cette conclusion gagnerait naturellement à être validée sur d'autres corpus de sons plus conséquents et variés. Le corpus \textit{SOUR} est représentatif d'un environnement sonore particulier, celui d'une ville occidentale, et ne comprend pas certaines sources sonores comme le bruit de fontaines, de passages de train ou de tramway par exemple. De plus, dans la conception de scènes sonores, d'autres classes de sons, relatifs au trafic routier, n'ont pas pu être enregistrées comme le passage de bus ou de deux roues (moto, scooter). La base de données élémentaire de sons gagnerait à obtenir des enregistrements de passages de ces véhicules de la même qualité que ceux des voitures afin d'apporter une plus grande diversité. L'application de la NMF IS à de nouveaux corpus de scènes sonores urbaines est donc nécessaire.

\subsection*{Création généralisée de scènes sonores urbaines}
Un autre aspect lié à la création de scènes sonores n'a également pas pu être étudié durant ces travaux :  l'utilisation dans le logiciel \textit{SimScene} des paramètres de hauts niveaux (classes de sons, niveaux sonores, occurrences de chaque classe de son par ambiance) relevés lors des annotations des enregistrements \textit{GRAFIC} (voir Tableau \ref{tab:obsScene}).
Ces paramètres peuvent être utilisés dans le simulateur \textit{SimScene} afin de générer plus aléatoirement des scènes sonores urbaines. L'utilisation de tels paramètres ouvrirait alors la possibilité de pouvoir composer de plus larges corpus de scènes sonores urbaines tout en étant basés sur des données issues d'enregistrements sonores. Leurs utilisations pour composer des scènes n'a pas pu être réalisée et validée durant cette thèse. Afin de consolider les valeurs extraites, il est nécessaire de réaliser de nouvelles annotations sur d'autres enregistrements sonores urbains, de créer des scènes sonores à partir de ces paramètres pour ensuite les soumettre à un test perceptif. Là où le corpus \textit{SOUR} possédait une structure écologiquement valide puisque basée sur des enregistrements, ce corpus serait plus construit à partir de données plus aléatoires.
Ces travaux impliquent aussi de compléter et d'enrichir la base de données élémentaires d'autres sources sonores (camion, bus, deux roues) mais surtout d'enregistrements de voix basés sur le ton de la discussion. En effet, une limite cette base de données est actuellement la composition des voix qui n'est pas suffisante pour atteindre le degré de réalisme souhaité. Celle-ci se compose de mots, de sons. Lors du test perceptif mené, les quelques évènements \textit{voix} évalués par les auditeurs ont eu un effet négatif sur l'évaluation du réalisme. Les voix présentes dans les enregistrements \textit{GRAFIC} sont composées de personnes qui discutent et rient à plusieurs, parlent au téléphone\dots{}
Aucune base de données libre proposant de tels extraits sonores n'a été trouvée, il serait donc utile d'en créer une afin de satisfaire le réalisme des scènes sonores urbaines simulées. La piste la plus rigoureuse et souhaitable serait d'enregistrer des dialogues écrits et lus par des comédiens dans des studios d'enregistrements et cela dans plusieurs langues. Malgré cela, les corpus d'évaluation \textit{Ambiance} ou \textit{SOUR}, bien que perfectibles, restent, en l'état actuel, de qualité suffisante pour être utilisés par les communautés dédiés à la création d'outils de traitement du signal pour des tâches de détection, de séparation de sources ou de classification de scènes sonores.

\subsection*{Extension à d'autres sources sonores}

Enfin, si ces travaux se sont intéressés au bruit du trafic routier, il convient de les étendre à d'autres sources comme la voix et les oiseaux, deux autres sources sonores prépondérantes dans la modélisatin de la perception des ESU. Des premières investigations ont commencé dont les premiers résultats semblent induire la nécessité d'adapter les représentations des spectrogrammes selon les sources : si la représentation en bandes de tiers d'octaves est adaptée pour le trafic routier, celle-ci semble moins l'être pour la voix et pour les oiseaux. Un représentation en bandes mel ou en bandes fines mais centrées autour des sources sonores sont des pistes à explorer. \\

L'étude entreprise durant cette thèse sur l'estimation de la contribution sonore du trafic routier en ville engage donc la réalisation d'autres travaux pouvant être menés à court et moyen termes et ouvrent ainsi des perspectives vers une connaissance plus précise des environnements sonores urbains et des sources qui les composent.
