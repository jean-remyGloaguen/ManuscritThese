\chapter*{Conclusions générales et perspectives}
\label{chap:concl}
\addcontentsline{toc}{chapter}{Conclusions générales et perspectives}


%\section{Rappel de la problématique de recherche}

L'objectif de ces travaux de thèse a été de déterminer le niveau sonore du trafic routier dans des scènes sonores urbaines. Pour cela une technique de séparation de source a été choisie : la Factorisation en Matrices Non-négatives.
Cette méthode présente l'intérêt d'être adaptée aux capteurs monophoniques qui équipent les capteurs des réseaux ou bien encore ceux des smartphones. De plus, la question du recouvrement de sources, phénomène récurrent en ville et qui complique la tâche de la séparation, est ici naturellement intégrée à la méthode NMF. 
La difficulté principale réside dans la diversité des sources sonores présentes en villes et aux différentes ambiances sonores auxquelles la méthode doit faire face : de l'ambiance \textit{Parc}, avec une prédominance de voix et d'oiseaux, à l'ambiance \textit{Rue très bruyante}, où le trafic est la source sonore principale.
Cette méthode a été appliquée, non pas sur des enregistrements audio mais sur des scènes sonores simulées qui permettent de contrôler l'ensemble des scènes sonores (classe de sons présente, niveaux sonores, occurrence dans une scène d'une classe de son \dots). Ce choix permet notamment de connaitre le niveau sonore du trafic exact qui est ainsi comparer au niveau sonore estimé par la NMF. La qualité de la base de données de son et du réalisme des scènes crées a été un point d'étude abordé notamment par la réalisation d'un test perceptif. 


%\section{Bilan et auto-évaluation}

%\subsection{sur la NMF}

Afin de trouver une forme optimale de la NMF, de nombreuses aspects et formes de cette méthode ont été abordées soit au travers de l'apprentissage supervisée ou semi-supervisée du dictionnaire, basé sur des enregistrements de passages de voitures, soit sur le choix de la $\beta$-divergence. Dans le cadre de ces travaux, une autre forme de NMF basée sur un apprentissage supervisée à été générée : la NMF \textit{initialisée seuillée}. Cette NMF consiste en l'apprentissage d'un dictionnaire initiale, $\mathbf{W_0}$, composé d'éléments reliés à la source \textit{trafic}. Ce dictionnaire est ensuite mis à jour sur la scène testée. Chaque élément du dictionnaire obtenu,$\mathbf{W'}$, est alors comparé à son état initiale dans $\mathbf{W_0}$. Par une technique de seuillage, les éléments les plus similaires à leur états initiaux, et donc susceptibles d'être encore liés au trafic, sont alors conservés et forment la composante \textit{trafic} final.
L'ensemble de ces méthodes a été comparé au regard d'une seconde méthode, une filtre passe-bas de fréquence de coupure $f_c$. Si cette approche paraissait simple, elle s'est révélée performante sur l'ensemble des corpus.

L'étude de ces méthodes sur un premier corpus, le corpus \textit{Ambiance}, soulève la difficulté d'obtenir une méthode adaptée à l'ensemble à des sons différents avec une présence de trafic variable. Dans le cas de la NMF SUP constituée d'un dictionnaire \textit{trafic}, elle échoue a obtenir des estimations satisfaisantes lorsque le trafic est peu présent mais devient performante lorsqu'il devient prépondérant. À l'inverse, la semi-supervisée, par l'ajout d'une partie du dictionnaire libre, se trouve être performante lorsque le trafic est peu présent et est défaillante lorsque cette source devient principale. Les degrés de liberté ajoutés sont ainsi suivant la présence du trafic un atout ou une faiblesse. Celle-ci se révèlera ainsi plus performante lorsqu'une contrainte de régularité temporelle est ajoutée. 
La NMF seuillé initialisée se révèle être la méthode qui, l'intégralité du corpus, donne la plus faible erreur. Selon le $TIR$, cette méthode se révèle pourtant ne jamais être la plus efficace. Mais grâce aux mises à jours du dictionnaire et à son seuil, arrive à être un équilibre qui s'adapte. Cette méthode montre toutefois des limites face aux classes de sons similaires d'un point de vue fréquentielle, notamment lors de la présence de pluie ou d'orage. Sur cette source, implémenté dans des capteurs équipée d'une station météo, il sera peut être préférable de ne pas l'utiliser lors de ces évènements climatiques.

Les tests menés sur le second corpus, confirmera les conclusions faites : la NMF IS est la méthode la plus performante qui s'adapte le mieux aux différents ESU avec en moyenne une erreur $MAE$ de 1,20 $\pm$ 0,87 pour une distance euclidienne, un nombre d'élément $K = 300$, une fenêtre temporelle $w_t$ = 1 s et un seuil de 0,35. Cette combinaison obtenue pour le corpus SOUR peut ainsi être celle retenu comme la plus adaptée pour un ESU en vue de déterminer le niveau sonore du trafic.
Cette conclusion gagnerait naturellement à être validée sur d'autres corpus de sons plus différents. Le corpus SOUR est  représentatif d'un environnement sonore particulier (celui d'une ville occidentale) et ne comprend également pas de nombreuses autres sources sonores comme le bruit de fontaines par exemple. De plus, dans la conception de scènes sonores, d'autres classe de son n'ont pas pu être enregistré comme des bus ou des deux roues (moto, scooter) qui sont des sources appartement à la classe \textit{trafic}. La base de données élémentaire de sons gagnerait à obtenir des enregistrements de passages de ces véhicules de la même qualité que ceux des voitures.

Les corpus d'évaluation \textit{Ambiance} ou \textit{SOUR}, en l'état actuel, restent toutefois suffisamment de qualité pour être utilisés par les communautés dédiés à la formation d'outils de traitement du signal pour des tâches de détection, de séparation de source ou de classification de scènes sonores.

%\section{Ouvertures et perspectives}

Naturellement, les aspects liés aux bases de sons sont  améliorable. Le corpus de SOUR a été composé à partir de l'annotation de scènes sonores enregistrés. Si leur écoute a permis de définir leur structure temporel, plusieurs paramètres de hauts niveaux ont également été extraits. Ces paramètres peuvent être utilisés dans l'outil de simulation \textit{simScene} pour générer des scènes sur des bases plus aléatoires. Ces aspects ouvrent sur la possibilités de générer des scènes sonores urbaines quasi-infini. Toutefois, la formation de ces scènes devraient être au préalable valider au travers de tests perceptifs afin de valider les valeurs de ces paramètres.
Une autre phase d'enregistrements d'ambiances sonores urbaines serait donc à réaliser et à annoter afin l'exhaustivité des classes de sons présentes et de mieux estimer leur paramètre moyens.
La base de donnée élémentaire doit ainsi être complété et enrichi en enregistrant d'autres sources sonores (camion, bus, deux roues) mais aussi en enregistrant des voix sur le ton de la discussion. Si il est possible d'enregistrer des conversations facilement par défaut, on gagnerait, pour constituer une base de données rigoureuse, à enregistrer des voies dans des studio d'enregistrement, à l'aide de comédiens lisant des dialogues écrits et cela dans plusieurs langues.

Enfin, si ces travaux se sont intéressés au bruit du trafic routier, il convient d'étendre ces travaux à d'autres source comme la voix et les oiseaux, deux autres sources sonores prépondérantes dans la perception des ESU. Ces aspects sont actuellement en cours d'études basé sur la NMF. Les premiers résultats montrent la nécessité d'adapter les représentations des spectrogrammes selon les sources : si la représentation en tiers d'octaves étaient adaptée pour le trafic routier, celle-ci semble moins l'être pour la voie et pour les oiseaux. Un représentation en bandes mel ou en linéaire mais une bande de fréquences réduites centré autour de la cible. De plus, si le trafic est une source continue, inclure de la parcimonie peut être intéressant également aussi.
Ces premiers travaux ouvrent donc la voie vers une connaissance plus précise des environnement sonores urbains et des sources qui la compose.






