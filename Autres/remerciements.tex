\chapter*{Remerciements}
\addcontentsline{toc}{chapter}{Remerciements}

%Tout d'abord, je tiens à remercier mes encadrants Arnaud Can et Mathieu Lagrange, pour m'avoir guidé, épaulé et aidé durant cette thèse. Chacun de vous aura été complémentaire et m'auront permis durant cette thèse de découvrir le monde de la recherche et de progressé tant scientifiquement qu’humainement.
% 
%Je remercie aussi Jean-François Petiot pour sa participation à ces travaux et d'avoir, par son regard extérieur, permis de nombreuses améliorations.\\
%
%Je tiens à remercier également chaque membre du jury pour avoir accepté de participer à ma soutenance et d'évaluer mon travail.\\
%
%Je suis reconnaissant à Jean-Julien Aucouturier, chercheur CNRS au sein du laboratoire Sciences et Technologies de la Musique et du Son, d'avoir accepté d'être membre de mon CSI et dont les discussions sur les tests statistiques m'ont beaucoup aidé et à Cédric Févotte, Directeur de Recherche CNRS à l'IRIT, d'avoir pris le temps de discuter avec nous sur ce sujet et d'avoir partagé son expérience et son point de vue.\\
%
%Un grand merci à Judicaël Picaut, directeur de l'UMRAE, et aux membres de cette unité pour m'avoir accueilli dans leur équipe et de m'avoir offert de si agréable conditions de travail durant les trois années de cette thèse. Une distinction particulière pour Vincent Gary, technicien, qui a participé aux enregistrements sur pistes avec nous et à Marie-Agnès Pallas, chargé de recherche, pour nous avoir transmis les enregistrements audio, même si ces derniers n'ont pas pu être utilisables. Je n'oublie pas mon camarade de bureau, Pierre Aumond, ingénieur de recherche, dont les échanges sur la science et le monde ont toujours été intéressants et agréables.\\
%
%J'ai une pensée à mes camarades de l'Ifsttar qui poursuivent ou finissent leur thèse, nos repas hebdomadaires entre doctorant m'auront toujours fait plaisir et apporté une bouffée d'air frais bien utile. Bonne continuation à eux !\\
%
%Je salue mes anciens camarades de promotions de Master dont j'ai eu plaisir à retrouver au grès des différents congrès d'acoustique. J'espère que d'autres réunions viendront à l'avenir !\\
%
%Enfin, je remercie ma famille, mes parents et ma soeur (plus le futur enfant qui ne va pas tarder) qui m'ont toujours soutenu durant mes études et qui m'ont permis d'en arriver là.\\
%
%Enfin, j'ai une grande pensée et tendresse à celle qui partage ma vie depuis plus de sept ans, celle qui m'a soutenue et réconforté tout au long de cette thèse, celle qui m'a, de nombreuses fois, écouté parler de détails techniques bien trop confus parfois pour être compréhensibles. \`A toi, Ingrid, je te remercie.


