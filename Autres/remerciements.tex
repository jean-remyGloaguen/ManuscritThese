\chapter*{Remerciements}
\addcontentsline{toc}{chapter}{Remerciements}


Durant ces 3 années de thèse, de nombreuses personnes auront été à mes côtés pour m'aider, m'épauler et me soutenir dans mes travaux. \\

C'est donc tout naturellement que je remercie, en premier lieu, mon directeur de thèse, Jean-François Petiot, Professeur des Universités qui aura apporté par son regard de précieux conseils et suggestions notamment pour la réalisation des tests statistiques du test perceptif.

Je remercie également, mes encadrants, Arnaud Can et Mathieu Lagrange, tous deux chargés de recherche, respectivement à l'UMRAE et au LS2N, pour leur précieuse aide au quotidien, leurs points de vues complémentaires et pour s'être montrés si disponibles tout au long de ces 3 années. Les nombreuses discussions menées avec vous sur mes travaux ou sur le milieu de la recherche m'auront beaucoup apporté sur ce qu'est la recherche et le métier de chercheur. \\

Je tiens à remercier également chaque membre du jury pour avoir accepté de participer à ma soutenance et d'évaluer mon travail.

Je suis reconnaissant à Jean-Julien Aucouturier, chercheur CNRS au sein du laboratoire Sciences et Technologies de la Musique et du Son, d'avoir accepté d'être membre de mon CSI et dont les discussions sur les tests statistiques m'ont beaucoup aidé ainsi qu'à Cédric Févotte, Directeur de Recherche CNRS à l'IRIT, d'avoir pris le temps de discuter avec nous sur ce sujet et d'avoir partagé son expérience et son point de vue. \\

Un grand merci à Judicaël Picaut, directeur de l'UMRAE, et aux membres de cette unité pour m'avoir accueilli dans leur équipe et de m'avoir offert de si agréable conditions de travail durant les trois années de cette thèse. Une distinction particulière pour Vincent Gary, technicien, qui a participé aux enregistrements des passages de voitures sur la piste avec nous. Je n'oublie pas mon camarade de bureau, Pierre Aumond, ingénieur de recherche, dont les échanges sur la science et le monde auront toujours été intéressants. \\

J'ai une pensée à mes camarades de l'Ifsttar qui poursuivent ou finissent leur thèse, nos repas hebdomadaires entre doctorants m'auront toujours fait plaisir et apporté une bouffée d'air frais bien utile. Bonne continuation à eux !

Je salue mes anciens camarades de promotions de Master dont j'ai eu plaisir à les retrouver au grès des différents congrès d'acoustique. J'espère que d'autres réunions viendront à l'avenir !\\

En vrac, je suis reconnaissant à ces artistes qui m'auront accompagné dans les oreilles pendant ces phases de rédaction et de codage sous Matlab : Jean-Sébastien Bach, Kevin Morby, Paul Desmond, Kimio Eto, Franz Schubert, Baden Powell, Les Parvarim, Simeon ten Holt, Father John Misty, Marissa Nadler\dots\\

Je remercie ma famille, mes parents et ma soeur qui m'ont toujours soutenu durant mes études et m'ont permis d'en arriver là.

Pour finir, j'ai une grande pensée et tendresse à celle qui partage ma vie depuis plus de sept ans, à celle qui m'a soutenue et réconforté tout au long de cette thèse, à celle qui m'a, de nombreuses fois, écouté parler de détails techniques bien trop confus parfois pour être compréhensibles. \`A toi, Ingrid, je te remercie.\\


