\chapter{Développement de la smooth-NMF pour $\beta = 0$ et $\beta = 2$}\label{annex:smoothNMF}

Inspiré par l'article \cite{essid2013smooth} et \cite{fevotte_algorithms_2011}, un algorithme de \textit{smooth NMF} est développé, basé sur l'approche de \textit{majorisation-minimisation}. D'abord développé pour la divergence K-L, elle a été étendue au cas de la distance EUC et à la divergence I-S. Les détails des calculs menant aux algorithmes sont présentés ici. Ils n'ont pas été implémentés et utilisés dans le cadre des travaux de la thèse.

\section{Cas de la distance Euclidienne}
\begin{align}
C(\mathbf{H}) &= D_2(\mathbf{V} \Vert \mathbf{WH}) + \alpha C_{MM}(\mathbf{H})\\
 &= \frac{1}{2}(\mathbf{V}-\mathbf{WH})^2+\alpha L(h_n; h_{n+1}, h_{n-1}) 
\end{align}

En suivant la méthode de \textit{majorisation-minimisation}, une fonction auxiliaire $G_{SM}(h_n\vert h_n)$ est définie : 

\begin{equation}
G_{SM}(h_n\vert h_n) = G_{2}(h_n\vert h_n)+\alpha L(h_n; h_{n+1}, h_{n-1})
\end{equation}

\begin{align}
G_{2}(h_n\vert h_n) &= \frac{1}{2}\sum_{f} \sum_{k} \frac{w_{fk}\tilde{h}_n v_f^2}{\tilde{v}_f}-2w_{fk} h_k v_f+w_{fk}\frac{h_k^2}{\tilde{h}_k} \tilde{v}_f\\
L(h_n; h_{n+1}, h_{n-1}) &= \frac{1}{2}\sum_{k}\lambda_k^2 \left[ (h_{k(n+1)}-h_{kn})^2+(h_{kn}-h_{k(n-1)}^2) \right]\\
 &= \sum_k \lambda_k^2 \left[ h_{kn}^2- h_{kn}(h_{k(n+1)}+h_{k(n-1)}+\frac{1}{2} (h_{k(n+1)}^2+h_{k(n-1)}^2)) \right]
\end{align}

La minimisation de la fonction $G_{SM}(h_n\vert h_n)$ est alors déterminée en trouvant les zéros de son gradient $\nabla_{h_k} G_{SM}(h_n\vert h_n)$ :

\begin{align}
\nabla_{h_k} G_{SM}(h_n\vert h_n) &= \nabla_{h_k} G_{2}(h_n\vert h_n) + \alpha \nabla_{h_k} L(h_n; h_{n+1}, h_{n-1})\\
 &= \frac{1}{2}\sum_{f} \sum_{k} -2 w_{fk} v_f + 2 w_{fk} v_f \frac{h_k}{\tilde{h}_k} + \alpha \lambda_k^2 \left[ 2 h_{kn} - (h_{k(n+1)}^2+h_{k(n-1)}^2)) \right]\\
 &= \left( 2\alpha \lambda_k^2 + \frac{w_{fk} v_f} {\tilde{h}_k}\right)h_k - \left(w_{fk} v_f + \alpha   \lambda_k^2 (h_{k(n+1)}^2+h_{k(n-1)}^2))\right) \label{eq:smooth_2}
\end{align}

L'équation \ref{eq:smooth_2} est alors résolue et permet d'obtenir l'expression des mises à jours de $h_n$ : 

\begin{equation}
h_n = \frac{w_{fk} v_f+\alpha   \lambda_k^2 \left(h_{k(n+1)}^2+h_{k(n-1)}^2)\right)}{2\alpha \lambda_k^2 + \frac{w_{fk} v_f} {\tilde{h}_k}}
\end{equation}

Dans le cas où $\alpha$ = 0, on retrouve bien l' algorithme \ref{eq:update_hk}.


\section{Cas de la divergence d'Itakura-Saïto}
Dans le cas d'une divergence IS, le problème est similaire avec

\begin{align}
C(\mathbf{H}) &= D_0(\mathbf{V} \Vert \mathbf{WH}) + \alpha C_{MM}(\mathbf{H})\\
 &= \frac{\mathbf{V}}{\mathbf{WH}}-\log \frac{\mathbf{V}}{\mathbf{WH}}-1+\alpha L(h_n; h_{n+1}, h_{n-1}) 
\end{align}

La fonction auxiliaire $G_0(h_n\vert h_n)$ n'est toutefois pas la même : 

\begin{equation}
G_0(h_n\vert h_n) = \sum \left[ \sum_{k = 1}^K \frac{w_{fk} \tilde{h}_k}{\tilde{v}_f}v_f \frac{\tilde{h}_k}{\tilde{v}_f h_k} + \log \tilde{v}_f+ \frac{1}{\tilde{v}_f}(w_{fk}(h_k-\tilde{h}_k) + v_f(\log v_f - 1)\right]
\end{equation}

\begin{align}
\nabla_{h_k} G_{SM}(h_n\vert h_n) &= \nabla_{h_k} G_0(h_n\vert h_n) + \alpha \nabla_{h_k} L(h_n; h_{n+1}, h_{n-1})\\
&= \sum_f \sum_k -\frac{w_{fk} \tilde{h}_k^2 v_f}{\tilde{v}_f^2 h_k^2}+\frac{w_{fk}}{\tilde{v}_f} + \alpha \lambda_k^2 \left[ 2 h_{kn} - (h_{k(n+1)}^2+h_{k(n-1)}^2)) \right]\\
 &= 2\alpha_\lambda^2h_n^3+h_{n}^2\left[\frac{w_{fk}}{\tilde{v}_f} - \alpha \lambda_k^2(h_{k(n+1)}+h_{k(n-1}) \right] - w_{fk}\frac{v_f}{\tilde{v}_f}\tilde{h}_n^2 \label{eq:eq_mimiser_smooth}
\end{align}

On obtient un problème de type équation du 3ème ordre : $ax^3+bx^2+cx+d = 0$ avec 
\begin{itemize}
\item $a = 2\alpha_\lambda^2$, 
\item $b = \frac{w_{fk}}{\tilde{v}_f} - \alpha \lambda_k^2(h_{k(n+1)}+h_{k(n-1})$, 
\item $c = 0$, 
\item $d = w_{fk}\frac{v_f}{\tilde{v}_f}\tilde{h}_n^2$
\end{itemize}

Cette polynôme d'ordre 3 se résout à l'aide de la méthode de Cardan \cite{nickalls1993new}, qui dans le cas présent, s'exprime sous la forme : 

\begin{equation}
x^3+\frac{b}{a}x^2+\frac{d}{a} = 0
\end{equation}

Dans un premier temps, un premier changement de variable est effectué $x = X = \frac{b}{3a}$ et on obtient : 

\begin{equation}\label{eq:cardan}
X^3+pX+q = 0
\end{equation}

avec $p = -\frac{b^2}{3a^2}$ et $q = \frac{2b^3}{27a^3}+\frac{d}{a}$. La variable $X$ est alors décomposé en deux variables complexes : $X = u+v$. L'équation \ref{eq:cardan} devient alors : 

\begin{equation}\label{eq:u_v}
u^3+v^3+(3uv+p)(u+v)+q = 0
\end{equation}

L'équation \ref{eq:u_v}, pour être résolue, implique alors deux conditions  : 
\begin{subequations}\label{eq:condition1}
\begin{align}
3uv &= -p,\\
u^3+v^3 &= -q.
\end{align}
\end{subequations}

Du système d'équations \ref{eq:condition1}, on en obtient un nouveau : 

\begin{subequations}\label{eq:condition2}
\begin{align}
u^3+v^3 &= -q,\\
u^3v^3 &= -\frac{p^3}{27}.
\end{align}
\end{subequations}

Si on réalise un nouveau changement de variable $u^3 = U$ et $v^3 = V$, on exprime le système d'équation \ref{eq:condition2} comme des polynômes de second degré : 

\begin{subequations}\label{eq:polyUV}
\begin{align}
U^2+qU-\frac{p^3}{27} &= 0,\\
V^2+qV-\frac{p^3}{27} &= 0.
\end{align}
\end{subequations}

Le système \ref{eq:polyUV} se résout classiquement : 

\begin{equation}
\Delta_u = \Delta_v = q^2-4\frac{p^3}{27}
\end{equation}

$U$ et $V$ ont alors deux solutions ($U_1/V_1 = \frac{-q+\sqrt{\Delta}}{2}$ et $U_2/V_2 = \frac{-q-\sqrt{\Delta}}{2}$). Seul le couple de solution  $\left[U_1-V_2\right]$ (ou $\left[V_1-U_2\right]$) est conservé afin de respecter les conditions \ref{eq:condition2}. Les valeurs de $X$ et solutions de l'équation \ref{eq:cardan} dépendent alors du signe de $\Delta$

\begin{itemize}
\item pour $\Delta >0$, 

\begin{subequations}
\begin{align}
X_1 &= \sqrt[3]{\frac{-q+i\sqrt{\Delta}}{2}}+\sqrt[3]{\frac{-q-i\sqrt{\Delta}}{2}}\\
X_2 &= j\sqrt[3]{\frac{-q+i\sqrt{\Delta}}{2}}+j\sqrt[3]{\frac{-q-i\sqrt{\Delta}}{2}}\\
X_3 &= j^2\sqrt[3]{\frac{-q+i\sqrt{\Delta}}{2}}+j^2\sqrt[3]{\frac{-q-i\sqrt{\Delta}}{2}}
\end{align}
\end{subequations}

avec $j = e^{\sfrac{2i\pi}{3}}$. Une autre forme d'écriture la solution est possible sous une forme trigonométrique : 

\begin{equation}
X_{k+1} = 2\sqrt{\frac{-p}{3}} \cos \left(\frac{1}{3} \arccos\left(\frac{-q}{2}\sqrt{\frac{27}{-p^3}}\right) + \frac{2k\pi}{3}\right)
\end{equation}

avec $k \in \left\lbrace 0,1,2 \right\rbrace$.


\item pour $\Delta = 0$, parmi les 3 solutions possibles, 1 seule solution est réelle : 

\begin{subequations}
\begin{align}
X_1 &= \frac{3q}{p}\\
X_2 &= X_3 = \frac{-3q}{2p}
\end{align}
\end{subequations}

\item pour $\Delta < 0$, l'équation possède une solution réelle et deux solutions complexes
$U =  \frac{-q+i\sqrt{\vert \Delta \vert}}{2}$ et $V = \frac{-q-i\sqrt{\vert \Delta \vert}}{2}$. L'équation possède alors une solution réelle et 2 solutions complexes qui sont 
: 

\begin{subequations}
\begin{align}
X_1 &= \sqrt[3]{U}+\sqrt[3]{V}\\
X_2 &= j\sqrt[3]{U}+\bar{j}\sqrt[3]{V}\\
X_3 &= j^2\sqrt[3]{U}+\bar{j^2}\sqrt[3]{V}.
\end{align}
\end{subequations}
\end{itemize}

La solution du problème \ref{eq:eq_mimiser_smooth} est donc la valeur $X_k$ qui est réelle et positive, définie selon le signe de $\Delta$ : 

\begin{equation}
\fbox{$
h_n = X_{k}^+-\frac{b}{3a}.
$}
\end{equation}