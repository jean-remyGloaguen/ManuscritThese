%% Unofficial SPI cover LaTeX template
%% 2017-12-07
%
% By Quentin Ayoul-Guilmard
% Contact: quentin.ayoul-guilmard@centraliens-nantes.net
%
%% DISCLAIMER
%
% This template aims at reproducing the look of SPI doctorate school's Microsoft Word template, as it is today (see date above).
% There are minute and purposeful discrepancies, which I consider to be improvements.
% I will not tell where.
%
% To use this template, follow these steps
% 1_ Copy its preamble into yours (except for document class, obviously)
% 2_ Copy the logo files into your document's folder.
% 3_ Copy the content between "FRONT COVER" and "END OF FRONT COVER" at the beginning of your document
% 4_ Copy the content between "BACK COVER" and "END OF BACK COVER" at the end of your document
% 5_ Find all the comments labelled "TODO" and follow their instructions (if relevant).
% Alternatively to steps 3 and 4, you may put the contents in two files front-cover.tex and back-cover.tex, then load them with \input.
%
% Notice: This document has been successfully compiled with pdflatex and TexLive 2016.
% I cannot guarantee that this code will work with your document.
% Furthermore, it will have to be updated as SPI changes its cover.
% I strove to make it easy to understand and modify with my comments.
% Any feedback or improvement is welcome.
% See below for some ideas.
%
% Ideas for improvement:
% _ Change language locally with babel for English abstract.
% _ Several spacing macros have to be adjusted manually (see below). It would be nice to have the corresponding spaces automatically adjusted.
% _ Two other packages (that I know of) can be used to set background pictures: background and wallpaper. Are they more suited than eso-pic here?
%
%% END OF DISCLAIMER

\documentclass[10pt]{article}

% Standard encoding and language packages.
\usepackage[utf8]{inputenc}
\usepackage[T1]{fontenc}
\usepackage[francais]{babel}
% TODO: replace the above packages with your own encoding settings.
% If you have none, you may want to consider using these.

% Required for cover
\usepackage[a4paper]{geometry} % to change margins on cover pages
\geometry{outer=2cm,inner=3cm,top=3cm}
\usepackage[scaled]{helvet} % font used on cover (Helvetica)
\usepackage{eso-pic} % to set background picture
\usepackage{multicol} % for back cover (abstracts)
\usepackage{graphicx} % to include logos
\usepackage{tikz} % to compose background picture

% Used for template demonstration only
\usepackage{lipsum} % to generate random text with \lipsum.
%TODO: You can remove this package if you use no \lipsum command.

% Colors (extracted from SPI's template)
\definecolor{boxcolor1}{rgb}{0.91373,0.92941,0.87451}
\definecolor{boxcolor2}{rgb}{0.94902,0.93333,0.91373}
\definecolor{boxcolor3}{rgb}{0.76078,0.87843,0.17647}
\definecolor{headercolor}{rgb}{0.94118,0.30980,0.17255}
\definecolor{namecolor}{rgb}{1.0,0.4,0.0}
\definecolor{titlecolor}{rgb}{0.19216,0.51765,0.60784}
% Also used: gray, teal (predefined by xcolor package, usually loaded by document class)

% Cover environment, to keep changes local
\newenvironment{cover}{%
  \fontfamily{phv}\selectfont % Select Helvetica font
  \pagestyle{empty} % No page number
}{%
  % Do not count this page (useless on back cover, obviously)
  \addtocounter{page}{-1}
  % Go to next odd page (i.e. recto, or right page)
  \cleardoublepage
  % This last command is compulsory even for back cover;
  % without it, the page is not completed and some page-wide
  % commands fail to apply.
}

% Macro for background common to front and back
\newcommand{\tikzBG}{%
  \path (0,0) rectangle (1,1);
  %TODO: You should adjust the bottom height of the following rectangle to fit your abstract's length
  \path [fill=boxcolor1] (.0571,.11) rectangle (.481,.963); 
  \path [fill=boxcolor2] (.4333,.697) rectangle (.9048,.7475);
  \path [fill=boxcolor2] (.4333,.7811) rectangle (.9048,.8316);
  \path [fill=boxcolor2] (.4333,.8687) rectangle (.9048,.9192);
  \path [fill=boxcolor3] (.0571,.7879) rectangle (.5762,.8316);
  \node[inner sep=0pt] at (0.2285,0.8788) [above left] {%
    \includegraphics[height=.0707\paperheight,keepaspectratio]{../figures/logo/logo_unb.png}};
  \node[inner sep=0pt] at (0.6667,0.8788) [above right] {%
    \includegraphics[height=.0808\paperheight,keepaspectratio]{../figures/logo/logo_ecn_color.png}};
  \node at (.0571,.8316) [above right,color=headercolor] {%
    \fontsize{29}{35}\selectfont\bfseries Th\`ese de Doctorat};
}

% Macro for repeated information (to avoid insconsistency)
%TODO: fill in with no formatting but desired case
\newcommand{\firstName}{Jean-Rémy}
\newcommand{\surname}{Gloaguen}
\newcommand{\thesisTitle}{Estimation du niveau sonore de sources d'intérêt au sein de mixtures sonores urbaines : application au trafic routier}

\begin{document}

%% FRONT COVER
%%%%%%%%%%%%%%%%%%%%%%%%%%%%%%%%%%%%%%%%
%      Fichier maître (Main.tex)      %
%%%%%%%%%%%%%%%%%%%%%%%%%%%%%%%%%%%%%%%%
% Dorian Depriester, 2014

\makeatletter
\def\@ecole{école}
\newcommand{\ecole}[1]{
  \def\@ecole{#1}
}

\def\@specialite{Spécialité}
\newcommand{\specialite}[1]{
  \def\@specialite{#1}
}

\def\@ED{\'{E}cole Doctorale}
\newcommand{\ED}[1]{
  \def\@ED{#1}
}

\def\@doctorat{Doctorat}
\newcommand{\doctorat}[1]{
  \def\@doctorat{#1}
}

\def\@adresse{Adresse}
\newcommand{\adresse}[1]{
  \def\@adresse{#1}
}

%\def\@directeur{directeur}
%\newcommand{\directeur}[1]{
%  \def\@directeur{#1}
%}
%
%\def\@encadrant{encadrant}
%\newcommand{\encadrant}[1]{
%  \def\@encadrant{#1}
%}
\def\@jurya{}{}{}
\newcommand{\jurya}[3]{
  \def\@jurya{#1,	& #2	& #3\\}
}
\def\@juryb{}{}{}
\newcommand{\juryb}[3]{
  \def\@juryb{#1,	& #2	& #3\\}
}
\def\@juryc{}{}{}
\newcommand{\juryc}[3]{
  \def\@juryc{#1,	& #2	& #3\\}
}
\def\@juryd{}{}{}
\newcommand{\juryd}[3]{
  \def\@juryd{#1,	& #2	& #3\\}
}
\def\@jurye{}{}{}
\newcommand{\jurye}[3]{
  \def\@jurye{#1,	& #2	& #3\\}
}
\def\@juryf{}{}{}
\newcommand{\juryf}[3]{
  \def\@juryf{#1,	& #2	& #3\\}
}
\def\@juryg{}{}{}
\newcommand{\juryg}[3]{
  \def\@juryg{#1,	& #2	& #3\\}
}
\def\@juryh{}{}{}
\newcommand{\juryh}[3]{
  \def\@juryh{#1,	& #2	& #3\\}
}
\def\@juryi{}{}{}
\newcommand{\juryi}[3]{
  \def\@juryi{#1,	& #2	& #3\\}
}
\def\@directeur{}{}{}
\newcommand{\directeur}[3]{
  \def\@directeur{#1,	& #2	& #3\\}
}
\def\@encadranta{}{}{}
\newcommand{\encadranta}[3]{
  \def\@encadranta{#1,	& #2	& #3\\}
}
\def\@encadrantb{}{}{}
\newcommand{\encadrantb}[3]{
  \def\@encadrantb{#1,	& #2	& #3\\}
}
\makeatother

%\newcommand\BackgroundPic{%
%	\put(0,0){%
%		\parbox[b][\paperheight]{\paperwidth}{%
%			\includegraphics[height=0.45\paperheight]{./figures/logo/barre.png}%
%			\vfill
%		}
%	}
%}

\makeatletter
\newcommand{\pagedegarde}{
\newgeometry{top=2.5cm, bottom=1cm, left=2cm, right=1cm}
%\AddToShipoutPicture*{\BackgroundPic}
%\AddToShipoutPicture*{\EtiquetteThese}
  \begin{titlepage}
  \centering
      %\includegraphics[width=0.4\textwidth]{ParisTech-Institute.pdf}
      %\hfill
      %\includegraphics[width=0.2\textwidth]{Mines.pdf}\\
    %\vspace{1cm}
      {\Large \@ED}\\
    %\vspace{1cm}
      {\huge 
      	{\bfseries \@doctorat}\\
    \vspace{0.5cm}
      	TH\`{E}SE}\\
    \vspace{1cm}
   		{\bfseries pour obtenir le grade de docteur délivré par}\\
    \vspace{1cm}
    	{\huge\bfseries \@ecole}\\
    \vspace{0.5cm}
    	{\Large{\bfseries Spécialité doctorale ``\@specialite''}}\\
    \vspace{2cm}
    	\textit{présentée et soutenue publiquement par}\\
    \vspace{0.5cm}
    	{\Large {\bfseries \@author}} \\
    \vspace{0.5cm}
    	le \@date \\
    \vfill
       {\LARGE \color[rgb]{0,0,1} \bfseries{\@title}} \\
%    \vfill
%        Directeur de thèse : {\bfseries \@directeur}\\
%        Co-encadrant de thèse : {\bfseries \@encadrant}\\
    \vfill
	\begin{tabular}{>{\bfseries}lll}
		\large Jury\\
		\@jurya
		\@juryb
		\@juryc
		\@juryd
		\@jurye
		\@juryf
		\@juryg
		\@juryh
		\@juryi
		\@directeur
		\@encadranta
		\@encadrantb
	\end{tabular}
	\vfill
	
	\@adresse
  \end{titlepage}

\restoregeometry  
}

\author{Jean-Rémy \textsc{Gloaguen}}
\title{Titre qui démontre l'intérêt et l'approche novateur de mon super travail}
\ED{\'{E}cole doctorale \no XXX : Sciences Pour l'Ingénieur}
\doctorat{Doctorat \'Ecole Centrale de Nantes}
\specialite{Acoustique}
\date{\today}
\jurya{Mme./M. Prénom Nom}{Professeur, Etablissement}{Rapporteur}
\juryb{Mme./M. Prénom Nom}{Professeur, Etablissement}{Rapporteur}
\juryc{Mme./M. Prénom Nom}{Chargé de recherche, Etablissement}{Examinateur}
\juryd{Mme./M. Prénom Nom}{Chargé de recherche, Etablissement}{Examinateur}
\directeur{M. Jean-François Petiot}{Professeur, \'Ecole Centrale de Nantes}{Directeur de thèse}
\encadranta{M. Arnaud Can}{Chargé de recherche, UMRAE}{Encadrant}
\encadrantb{M. Mathieu Lagrange}{Chargé de recherche, LS2N}{Encadrant}
\ecole{l'\'{E}cole Centrale de Nantes}
\adresse{
	\textbf{Ifsttar Nantes\\ Unité Mixte de Recherche en Acoustique Environnementale (UMRAE)}\\	UMR Nantes, France
}


\begin{cover}
	\cleardoublepage

  \AddToShipoutPictureBG*{%
    \begin{tikzpicture}[x=\paperwidth,y=\paperheight]
      % Background common to front and back:
      \tikzBG{}
      % Only on back cover:
      \path [fill=boxcolor3] (.49048,.037037) rectangle (.9048,.08081) node at (.69764,.05892) {\scshape\Large\color{white} Université Bretagne Loire};
    \end{tikzpicture}%
  }

  % Back cover margins (different from front cover)
  \newgeometry{left=1.4cm, right=2.1cm, top=6.6cm, bottom=2.6cm}

  % Name
  \noindent\textcolor{gray}{\LARGE \firstName{} \textsc{\surname}}
  
  \vspace{6pt}

  % Titles
  \begin{center}
    \begin{minipage}[c]{.75\linewidth}
      \large\color{gray}\bfseries
      \thesisTitle
      
      \vspace{18pt}

      \noindent Traffic sound level estimation of source of interests inside urban sound mixture: application to the traffic sound level

    \end{minipage}
  \end{center}
  \vspace{20pt}

  \begin{multicols}{2}
    \noindent\textbf{\large Résumé}
    \medskip
    
	texte
	
    \small\noindent 
    \bigskip

    \noindent
    \textbf{Mots-clefs}
    \smallskip

    \noindent 
    keyword1 ; keyword2 ; keyword3 ; \dots{} ; final keyword.

    % Go to next column
    \columnbreak

    \noindent
    \textbf{\large Abstract}
    \medskip
	
	text    
    
    \noindent 
    \bigskip

    \noindent
    \textbf{Keywords}
    \smallskip

    \noindent %TODO: put your English keywords here.
    % Suggested format:
    keyword1; keyword2; keyword3; \dots{}; final keyword.
    
  \end{multicols}
\end{cover}


\end{document}

