\begin{cover}
	\cleardoublepage

  \AddToShipoutPictureBG*{%
    \begin{tikzpicture}[x=\paperwidth,y=\paperheight]
      % Background common to front and back:
      \tikzBG{}
      % Only on back cover:
      \path [fill=boxcolor3] (.49048,.037037) rectangle (.9048,.08081) node at (.69764,.05892) {\scshape\Large\color{white} Université Bretagne Loire};
    \end{tikzpicture}%
  }

  % Back cover margins (different from front cover)
  \newgeometry{left=1.4cm, right=2.1cm, top=6.6cm, bottom=2.6cm}

  % Name
  \noindent\textcolor{gray}{\LARGE \firstName{} \textsc{\surname}}
  
  \vspace{6pt}

  % Titles
  \begin{center}
    \begin{minipage}[c]{.75\linewidth}
      \large\color{gray}\bfseries
      \thesisTitle
      
      \vspace{18pt}

      \noindent Estimation of the noise level of sources of interest within urban noise mixtures: application to road traffic

    \end{minipage}
  \end{center}
%  \vspace{5pt}

  \begin{multicols}{2}
    \noindent\textbf{\large Résumé}
    \medskip
	
	{\small \begin{spacing}{1.0} Des réseaux de capteurs acoustiques sont actuellement mis en place dans plusieurs grandes villes afin d'obtenir une description plus fine de lenvironnement sonore urbain. Un des défis à relever est celui de réussir, à partir d'enregistrements sonores, à estimer des indicateurs utiles tels que le niveau sonore du trafic routier. Cette tâche n'est en rien triviale en raison de la multitude de sources sonores qui composent cet environnement. Pour cela, la Factorisation en Matrices Non-négatives (NMF) est considérée et appliquée sur deux corpus de mixtures sonores urbaines simulées. L'intérêt de simuler de telles mixtures est la possibilité de connaitre toutes les caractéristiques de chaque classe de son dont le niveau sonore exact du trafic routier. 
Le premier corpus consiste en 750 scènes de 30 secondes mélangeant une composante de trafic routier dont le niveau sonore est calibré et une classe de son plus générique. Les différents résultats ont notamment permis de proposer une nouvelle approche, appelée \og NMF Seuillée Initialisée \fg{}, qui se révèle être la plus performante. Le deuxième corpus créé  permet de simuler des mixtures sonores plus représentatives des enregistrements effectués en villes, dont leur réalisme a été validé par un test perceptif. Avec une erreur moyenne d'estimation du niveau sonore inférieure à 1,3 dB, la NMF Seuillée Initialisée se révèle, là encore, la méthode la plus adaptée aux différents environnements sonores urbains.  
Ces résultats ouvrent alors la voie vers l'utilisation de cette méthode à d'autres sources sonores, telles que les voix et les sifflements d'oiseaux, qui pourront mener, à terme, à la réalisation de cartes de bruits multi-sources.
\end{spacing}}
    
    
    \noindent 
	\bigskip
	
    \noindent
    \textbf{Mots clés}
    \smallskip

\begin{spacing}{1.2}
\noindent
Acoustique urbaine; Séparation de sources sonores; Factorisation en matrices non-négatives; Environnement sonore urbain.\end{spacing}
    % Go to next column
    \columnbreak

    \noindent\textbf{\large Abstract}
    \medskip
	
	{\small \begin{spacing}{1.0}Acoustic sensor networks are being set up in several major cities in order to obtain a more detailed description of the urban sound environment. One challenge is to estimate useful indicators such as the road traffic noise level on the basis of sound recordings. This task is by no means trivial because of the multitude of sound sources that composed this environment. For this, Non-negative Matrix Factorization (NMF) is considered and applied on two corpuses of simulated urban sound mixtures. The interest of simulating such mixtures is the possibility of knowing all the characteristics of each sound class including the exact road traffic noise level. 
The first corpus consists of 750 30-second scenes mixing a road traffic component with a calibrated sound level and a more generic sound class. The various results have notably made it possible to propose a new approach, called "Thresholded Initialized NMF", which is proving to be the most effective. The second corpus created makes it possible to simulate sound mixtures more representatives of recordings made in cities whose realism has been validated by a perceptual test. With an average noise level estimation error of less than 1.3 dB, the Thresholded Initialized NMF stays the most suitable method for the different urban noise environments.  
These results open the way to the use of this method for other sound sources, such as birds' whistling and voices, which can eventually lead to the creation of multi-source noise maps.\end{spacing}}
    
    
    \noindent 
    \bigskip

    \noindent
    \textbf{Keywords}
    \smallskip
    
\begin{spacing}{1.2}
\noindent 
Urban acoustics; Sound source separation; Non-negative Matrix Factorization; Urban sound environment.
\end{spacing}
  \end{multicols}
\end{cover}
