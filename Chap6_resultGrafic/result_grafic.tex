
\chapter{\'Etude du corpus de son \textit{grafic}}

Dans ce chapitre, le corpus de SOUR est soumis à la NMF en vue de déterminer le niveau sonore du trafic routier. 

\section{Rappel des facteurs expérimentaux}

Le corpus SOUR est composé de 74 fichiers audio d'une durée totale de .... Les fichiers sont classés selon 4 ambiance sonores : \textit{parc} (P) contenant 8 fichiers audio, rue calme (C), fichiers audio, rue bruyante (Br), fichiers audio et rue Très Bruyante (Tr-Br), fichier audio. 
À l'inverse du corpus Ambiance, la répartition de scènes entre les 4 catégories sonores ne sont pas les mêmes en raison du corpus d'enregistrement récupéré. Si cette répartition peut être considéré comme représentative de l'ESU.


De nombreux facteurs expérimentaux présenté dans le chapitre précédent sont utilisés ici.
La construction du dictionnaire est similaire à l'étape précédente : les spectrogrammes des signaux trafic sont découpés en trame de largeur $w_t$ pour ensuite les soumettre à un algorithme de clustering $K$-mean. Ici comme le corpus est plus petit, on étend le nombre d'élément dans le dictionnaire à 300. Le résumé de ces paramètres se trouve dans le tableau.

La même baseline est utilisée également : un filtre passe-bas de fréquence de coupure $f_c$.
Comme les scènes sonores sont plus longues (de 1 min jusqu'à 3 minutes), le nombre d'itération est étendu à 400. 
Le nombre d'élément dans le dictionnaire $\mathbf{W_r}$, pour le NMF SS, est maintenu à $J = 2$.
Dans le cas de la NMF IS, l'influence de l'opérateur sigmoïde dans le calcul de la distance et l'utilisation du seuil \textit{firm} s'étant relevé très faible, le choix a été fait de réduire l'étude au seul cas de représentation linéaire de la distante avec un seuillage dur afin de réduire les temps de calculs. 

On résume non plus l'erreur arithmétique mais l'erreur moyenne pondérée des erreurs ? plus juste non ? 


\begin{table*}[t]
\centering
\caption{Facteurs expérimentaux et leur modalité utilisé pour le coprus SOUR.}
\begin{tabularx}{17.5cm}{L{3cm}@{}C{12cm}@{}C{2cm}@{}}
	\hline
    \textbf{\begin{tabular}[c]{@{}l@{}}facteur \\ expérimentaux \end{tabular}} & \textbf{modalités} & \begin{tabular}[c]{@{}C{2cm}@{}}\textbf{nombre de}\\ \textbf{modalité}\end{tabular}\\ \toprule
\end{tabularx}

\begin{tabularx}{17.5cm}{L{3cm}@{}C{3cm}@{}@{}C{3cm}@{}@{}C{3cm}@{}@{}C{3cm}@{}C{2cm}@{}}
    \textbf{Environnement sonore} & \begin{tabular}[c]{@{}c@{}}parc\\ 'P'\end{tabular} & \begin{tabular}[c]{@{}c@{}}rue calme \\ 'Q'\end{tabular} & \begin{tabular}[c]{@{}c@{}}rue bruyante\\ 'N' \end{tabular}& \begin{tabular}[c]{@{}c@{}}rue très bruyante\\ 'vN'\end{tabular} & 4\\
\end{tabularx}

\begin{tabularx}{17.5cm}{L{3cm}@{}C{3cm}@{}@{}C{3cm}@{}@{}C{3cm}@{}@{}C{3cm}@{}C{2cm}@{}}
	\rowcolor[HTML]{C0C0C0}
  \textbf{method} & filtre passe bas & NMF SUP & NMF SEM & NMF IS & 4\\
\end{tabularx}

\begin{tabularx}{17.5cm}{L{3cm}@{}@{}C{2cm}@{}@{}C{2cm}@{}@{}C{2cm}@{}@{}C{2cm}@{}@{}C{2cm}@{}@{}C{2cm}@{}C{2cm}@{}}
\rowcolor[HTML]{C0C0C0}
   $\mathbf{f_c}$ (kHz) & 0.5 & 1 & 2 &  5 & 10 & 20 & 6\\
   \bottomrule
\end{tabularx}

\begin{tabularx}{17.5cm}{L{3cm}@{}C{3cm}@{}@{}C{3cm}@{}@{}C{3cm}@{}@{}C{3cm}@{}C{2cm}@{}}
    $\mathbf{w_t}$ (s)& 0.5 & 1 & 2 & \textit{all} & 4\\
\end{tabularx}

\begin{tabularx}{17.5cm}{L{3cm}@{}C{3cm}@{}@{}C{3cm}@{}@{}C{3cm}@{}@{}C{3cm}@{}C{2cm}@{}}
	\rowcolor[HTML]{C0C0C0}
    $\mathbf{K}$ & 25 & 50 & 100 & 200 & 4\\
\end{tabularx}


\begin{tabularx}{17.5cm}{L{3cm}@{}C{4cm}@{}@{}C{4cm}@{}@{}C{4cm}@{}C{2cm}@{}}
   $\mathbf{\beta}$ & 0 & 1 & 2 & 3\\
\end{tabularx}

\begin{tabularx}{17.5cm}{L{3cm}@{}C{12cm}@{}C{2cm}@{}}
	\rowcolor[HTML]{C0C0C0}
   seuillage dur $\mathbf{t_h}$ & de 0.30 à 0.60 avec un pas de 0.01 & 31\\
   \bottomrule
\end{tabularx}
\label{tab:experimental_factorsNMF}
\end{table*}

\section{Résultats}

\subsection{Erreur produite par la baseline}
Dans un premier temps, les erreurs réalisées par la baseline sont observés sur l'ensemble du corpus (Tableau), puis selon chaque ambiance (Figure \Ref{}).

\begin{table}[]
\centering
\caption{My caption}
\label{my-label}
\begin{tabular}{llllll}
$f_c$ (Hz) & $MAE$ & $MAE_{P}$ & $MAE_Q$ & $MAE_{Br}$ & $MAE_{Tr-Br}$ \\
      &     &            &        &             &                \\
500   &     &            &        &             &                \\
1k    &     &            &        &             &                \\
2k    &     &            &        &             &                \\
5k    &     &            &        &             &                \\
10k   &     &            &        &             &                \\
20k   &     &            &        &             &               
\end{tabular}
\end{table}

Au final, on obtient le même comportement, un filtre avec une faible fréquence de coupure ($f_c$ = ). Il est nécessaire de l'augmenter ici puisque le corpus est composé de moins de scène où le trafic est moins présent.

\subsection{Erreur produite par la NMF}


\label{chap:grafic}
\begin{itemize}
\item rappel du corpus ambiance avec ces ambiances, le nombres total d'échantillons et la durée par ambiances
\item rappel du principe de la méthode avec la métrique en bout de course
\item paramètre de la NMF (supervisée, semi-supervisée, seuillé, beta, threshold)
\item résumé de tout les paramètres (nombre complet de combinaisons) sous forme de tableau avec le nombre total de combinaison, nombre d'itération
\item résultats
\begin{itemize}
\item mae générale (filtre VV, supervisé VV, semi-supervisé VV, seuillé VV)
\item mae pour chaque ambiance (filtre VV, supervisé VV, semi-supervisé VV, seuillé VV)
\item allure des courbes Lp,1s pour chaque ambiance
\item fonction cost (supervisé VV, semi-supervisé VV, seuillé)
\item allure des 2 éléments dans Wr
\item distance $w_0$ et $w$
\item smoothness et impact notable sur le semi-supervisée
\item optimisation de la valeur seuil en fonction d'indicateur
calibration qui fait défault donc les valeurs de référence sont à revoir mais ça signifie qu'avec des indicateurs assez simple, on peut ajuster les seuils.
\end{itemize}



\begin{table}[]
\centering
\caption{My caption}
\label{my-label}
\begin{tabular}{lcccccc}
méthode & $f_c$ & $\beta$ & K & $w_t$ & t & MAE (dB) \\
filtre &  & - & - & - & - &  \\
filtre &  & - & - & - & - &  \\
NMF SUP & - & 0 & 200 & 1000 & - & 4,17 ($\pm$ 4,81) \\
NMF SUP & - & 1 & 25 & 2000 & - & 2,78 3,32 \\
NMF SUP & - & 2 & 25 & 500 & - & 2,31 2,87 \\
NMF SEM & - & 0 & 200 & 500 & - & 2,08 0,72 \\
NMF SEM & - & 1 & 200 & 2000 & - & 1,94 0,34 \\
 & - & 2 & 200 & 2000 & - & 2,33 1,32 \\
NMF IS & - & 0 &  &  &  &  \\
NMF IS & - & 1 &  &  &  &  \\
NMF IS & - & 2 &  &  &  & 
\end{tabular}
\end{table}


\end{itemize}



%\end{document}