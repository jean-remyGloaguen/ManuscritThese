
\chapter{\'Etude du corpus de son \textit{grafic}}

Dans ce chapitre, le corpus de SOUR est soumis à la NMF en vue de déterminer le niveau sonore du trafic routier. 

\section{Rappel des facteurs expérimentaux}

Le corpus SOUR est composé de 74 fichiers audio d'une durée totale de 2h50. Les détails du corpus par ambiance sonore sont présentés dans le tableau \ref{tab:resume_sour}.

\begin{table}[]
\caption{Résumé du corpus SOUR}
\label{tab:resume_sour}
\centering
\begin{tabular}{L{3cm}C{2cm}C{2cm}}
\textbf{ambiance sonore}  & \textbf{N} & \textbf{durée (s)}  \\ \toprule
parc & 8 & 960 \\
rue calme & 35 & 4636 \\
rue bruyante & 23 & 3366 \\
rue très bruyante & 8 & 1285 \\ \midrule
\textbf{total} & 74 & 10 247 \\ \bottomrule
\end{tabular}
\end{table}

Ce corpus est plus petit que le corpus \textit{Ambiance} mais présente l'avantage de mélanger un plus grand nombre de classe et d'être assimilable à des enregistrements sonores faits en ville. 

La même baseline que précédemment est utilisée également : un filtre passe-bas de fréquence de coupure $f_c$.
La construction du dictionnaire est la même qu'à l'étape précédente : les spectrogrammes des signaux trafic sont découpés en trame de largeur $w_t$ pour ensuite les soumettre à un algorithme de clustering $K$-mean. Ici comme le corpus est plus petit, on étend le nombre d'élément dans le dictionnaire à 300. 
Le nombre d'élément dans le dictionnaire $\mathbf{W_r}$, pour le NMF SS, est maintenu à $J = 2$.
Dans le cas de la NMF IS, l'influence de l'opérateur sigmoïde dans le calcul de la distance et l'utilisation du seuil \textit{firm} s'étant relevé très faible, le choix a été fait de réduire l'étude au seul cas de représentation linéaire de la distante avec un seuillage dur afin de réduire les temps de calculs.
Comme les scènes sonores sont plus longues (de 1 min jusqu'à 3 minutes), le nombre d'itération est étendu à 400. 
Le résumé de ces paramètres se trouve dans le Tableau \ref{tab:experimental_factorsNMF}.
 

Dans le corpus, la durée des scènes n'étant pas la même, à l'inverse du corpus \textit{Ambiance}, c'est l'erreur MAE toute les 60 secondes qui est calculée, $MAE_{60}$ afin d'avoir une durée normalisée. Si la durée totale de chaque ambiance est différente, avec donc un nombre d'erreur par minute calculé différent. Le choix a été fait de ne pas pondérer les erreurs par le nombre de minute dans le calcul de l'erreur MAE globale, $MAE_{gl.}$ afin d'obtenir la combinaison la plus efficace sur l'ensemble des ambiances. Par leur durée plus importante, la combinaison trouvée serait surtout celle la plus performante dans les ambiances \textit{rue calme} et \textit{rue bruyante}.

\begin{equation}
MAE_{g, SOUR} = \frac{\sum_{i = 1}^4 MAE_i}{4}.
\end{equation}


\begin{table*}[t]
\centering
\caption{Facteurs expérimentaux et leur modalité utilisé pour le coprus SOUR.}
\begin{tabularx}{17.5cm}{L{3cm}@{}C{12cm}@{}C{2cm}@{}}
	\hline
    \textbf{\begin{tabular}[c]{@{}l@{}}facteur \\ expérimentaux \end{tabular}} & \textbf{modalités} & \begin{tabular}[c]{@{}C{2cm}@{}}\textbf{nombre de}\\ \textbf{modalité}\end{tabular}\\ \toprule
\end{tabularx}

\begin{tabularx}{17.5cm}{L{3cm}@{}C{3cm}@{}@{}C{3cm}@{}@{}C{3cm}@{}@{}C{3cm}@{}C{2cm}@{}}
    \textbf{Environnement sonore} & \begin{tabular}[c]{@{}c@{}}parc\\ 'P'\end{tabular} & \begin{tabular}[c]{@{}c@{}}rue calme \\ 'Q'\end{tabular} & \begin{tabular}[c]{@{}c@{}}rue bruyante\\ 'N' \end{tabular}& \begin{tabular}[c]{@{}c@{}}rue très bruyante\\ 'vN'\end{tabular} & 4\\
\end{tabularx}

\begin{tabularx}{17.5cm}{L{3cm}@{}C{3cm}@{}@{}C{3cm}@{}@{}C{3cm}@{}@{}C{3cm}@{}C{2cm}@{}}
	\rowcolor[HTML]{C0C0C0}
  \textbf{method} & filtre passe bas & NMF SUP & NMF SEM & NMF IS & 4\\
\end{tabularx}

\begin{tabularx}{17.5cm}{L{3cm}@{}@{}C{1.714cm}@{}@{}C{1.714cm}@{}@{}C{1.714cm}@{}@{}C{1.714cm}@{}@{}C{1.714cm}@{}@{}C{1.714cm}@{}@{}C{1.714cm}@{}C{2cm}@{}}
   $\mathbf{f_c}$ (kHz) & 1 & 0.5 & 1 & 2 &  5 & 10 & 20 & 7\\
\end{tabularx}

\begin{tabularx}{17.5cm}{L{3cm}@{}C{3cm}@{}@{}C{3cm}@{}@{}C{3cm}@{}@{}C{3cm}@{}C{2cm}@{}}
\rowcolor[HTML]{C0C0C0}
    $\mathbf{w_t}$ (s)& 0.5 & 1 & 2 & \textit{all} & 4\\
\end{tabularx}

\begin{tabularx}{17.5cm}{L{3cm}@{}C{3cm}@{}@{}C{3cm}@{}@{}C{3cm}@{}@{}C{3cm}@{}C{2cm}@{}}
    $\mathbf{K}$ & 25 & 50 & 100 & 200 & 4\\
\end{tabularx}

\begin{tabularx}{17.5cm}{L{3cm}@{}C{4cm}@{}@{}C{4cm}@{}@{}C{4cm}@{}C{2cm}@{}}
\rowcolor[HTML]{C0C0C0}
   $\mathbf{\beta}$ & 0 & 1 & 2 & 3\\
\end{tabularx}

\begin{tabularx}{17.5cm}{L{3cm}@{}C{12cm}@{}C{2cm}@{}}
   seuillage dur $\mathbf{t_h}$ & de 0.30 à 0.60 avec un pas de 0.01 & 31\\
   \bottomrule
\end{tabularx}
\label{tab:experimental_factorsNMF}
\end{table*}

\section{Résultats}

\subsection{Erreur produite par la baseline}
Dans un premier temps, les erreurs réalisées par la baseline sont observés sur l'ensemble du corpus (Tableau), puis selon chaque ambiance (Figure \ref{}).

\begin{table}[]

\caption{Erreur moyenne $MAE_{gl.,60}$ et $MAE_{60}$ pour l'estimateur \textit{baseline}.}
\label{tab:grafic_baseline}
\centering
\begin{tabular}{L{2cm}C{2.5cm}C{2cm}C{2cm}C{2cm}C{2cm}}
$f_c$ (Hz) & $MAE_{60}$ & $MAE_{P}$ & $MAE_Q$ & $MAE_{Br}$  & $MAE_{TrBr}$ \\
\toprule
100 & 2,93 $\pm$ 0,60  & \textbf{2,53} & 3,98 & 2,69  & 2,69 \\
500 & \textbf{\textcolor{red}{2,03 $\pm$ 1,43}}  & 4,00 & \textbf{2,18} & 0,93  & 1,03 \\
1k & 2,45 $\pm$ 2,62 & 6,17 & 2,48 & \textbf{0,63}  & 0,58 \\
2k & 3,00 $\pm$ 3,32 & 7,69 & 2,98 & 0,95  & \textbf{0,38} \\
5k & 3,50 $\pm$ 3,90 & 9,01 & 3,44 & 1,13  & 0,42 \\
10k & 3,61 $\pm$ 3,99 & 9,24 & 3,61 & 1,17  & 0,43 \\
20k & 3,64 $\pm$ 4,00 & 9,28 & 3,65 & 1,20  & 0,44 \\
\bottomrule         
\end{tabular}
\end{table}

Au final, on obtient le même comportement, un filtre avec une faible fréquence de coupure ($f_c$ = ). Il est nécessaire de l'augmenter ici puisque le corpus est composé de moins de scène où le trafic est moins présent.

\subsection{Erreur produite par la NMF}


\label{chap:grafic}
\begin{itemize}
\item rappel du corpus ambiance avec ces ambiances, le nombres total d'échantillons et la durée par ambiances
\item rappel du principe de la méthode avec la métrique en bout de course
\item paramètre de la NMF (supervisée, semi-supervisée, seuillé, beta, threshold)
\item résumé de tout les paramètres (nombre complet de combinaisons) sous forme de tableau avec le nombre total de combinaison, nombre d'itération
\item résultats
\begin{itemize}
\item mae générale (filtre VV, supervisé VV, semi-supervisé VV, seuillé VV)
\item mae pour chaque ambiance (filtre VV, supervisé VV, semi-supervisé VV, seuillé VV)
\item allure des courbes Lp,1s pour chaque ambiance
\item fonction cost (supervisé VV, semi-supervisé VV, seuillé)
\item allure des 2 éléments dans Wr
\item distance $w_0$ et $w$
\item smoothness VVV et impact notable sur le semi-supervisée
\item optimisation de la valeur seuil en fonction d'indicateur
calibration qui fait défault donc les valeurs de référence sont à revoir mais ça signifie qu'avec des indicateurs assez simple, on peut ajuster les seuils et optimiser tout cela !.
\end{itemize}
\end{itemize}


\begin{table}[]
\centering
\caption{My caption}
\label{my-label}
\begin{tabular}{L{2cm}C{1.2cm}C{1.2cm}C{1.2cm}C{1.2cm}C{1.2cm}C{2.5cm}}
méthode & $f_c$ (kHz) & $\beta$ & K & $w_t$ & $t_h$ & $MAE_{60}$ \\ \toprule
\multirow{2}{*}{filtre PB} & 20 & - & - & - & - &  3,64 ($\pm$ 4,00)\\
 & 0,5 & - & - & - & - & 2,03 ($\pm$ 1,43) \\ \midrule
\multirow{3}{*}{NMF SUP} & - & 0 & 200 & 1000 & - & 4,03 ($\pm$ 4,47) \\
 & - & 1 & 25 & 2000 & - & 2,69 ($\pm$ 2,90) \\
 & - & 2 & 25 & 500 & - & 2,22 ($\pm$ 2,33) \\ \midrule
\multirow{3}{*}{NMF SEM} & - & 0 & 300 & 1000 & - & 2,16 ($\pm$ 0,67) \\
 & - & 1 & 300 & 500 & - & 1,94 ($\pm$ 0,53) \\
 & - & 2 & 300 & 2000 & - & 2,23 ($\pm$ 1,41) \\ \midrule
\multirow{3}{*}{NMF IS} & - & 0 & 25 & 1000 & 0,32 & 1,43 ($\pm$ 0,73) \\
 & - & 1 & 50 & 0 & 0,34 &  1,32 ($\pm$ 1,15)\\
 & - & \textbf{2} & \textbf{300} & \textbf{1000} & \textbf{0,35} & \textbf{\textcolor{red}{1,20 ($\pm$ 0,87)}}\\
 \bottomrule
\end{tabular}
\end{table}


\begin{table}[]
\centering
\caption{My caption}
\label{my-label}
\resizebox{\textwidth}{!}{%
\begin{tabular}{L{2cm}C{1.5cm}C{1.5cm}C{1.5cm}C{1.5cm}C{1.5cm}C{1.5cm}C{1.5cm}C{1.5cm}C{1.5cm}}
méthode & $f_c$ (kHz) & $K$ & $w_t$ & $\beta$ & $t_h$ & P & C & Br & Tr-Br \\ \toprule
filtre PB & 20 & - & - & - &  - & 9,28 & 3,65 & 1,20 & 0,44\\
filtre PB & 0,5 & -  & - & - & - & 4,00 & 2,18 & 0,93 & 1,03 \\ \midrule
NMF SUP & - & 25 & 500 & 2 & - & 5,71 & 2,05 & 0,63 & 0,53\\
NMF SEM & - & 300 & 500 &  & - & 2,37 & 2,36 & 1,76 & 1,27\\
NMF IS & - & 300 & 1000 & 1 & 0,35 & 2,21 & 1,64 & 0,61 & 0,34\\ \bottomrule
\end{tabular}}
\end{table}

\subsection{NMF contraintes Smoothness et parcimonie}

\subsection{Optimation par les seuils}




%\end{document}