%%\documentclass[twoside,openright,a4paper,11pt]{book}
%%
%%
\usepackage[utf8]{inputenc}
\usepackage[francais]{babel}
\usepackage[T1]{fontenc}

\addto\captionsfrench{\def\tablename{\textsc{Tableau}}}% pour avoir TABLEAU et pas TABLE dans les légendes des tableaux
%%%%%%%% MISE EN PAGES %%%%%%
\usepackage{geometry}
\geometry{outer=2cm,inner=3cm,top=3cm}

\setcounter{tocdepth}{3}     % Dans la table des matieres
\setcounter{secnumdepth}{3}  % Avec un numero.
\usepackage{setspace}

\usepackage{fancyhdr}	% marge en haut et en bas
\pagestyle{fancy}

\fancyhead{}	% vide l'entête
\fancyfoot{} % vide le pied~de~page

\fancyhead[RO]{\leftmark}
\fancyhead[LE]{\rightmark}
\fancyfoot[C]{\thepage}	% numéro de page en bas au centre

\renewcommand{\headrulewidth}{0.4pt} % épaisseur du trait en haut
\renewcommand{\footrulewidth}{0.4pt} % épaisseur du trait en bas

\fancypagestyle{mypagestyle}{%
    \fancyhead{}	
    \fancyfoot{} 
    \fancyfoot[C]{\thepage}
    \renewcommand{\headrulewidth}{0.4pt} 
	\renewcommand{\footrulewidth}{0.4pt} 
}

\fancypagestyle{couvertureAbstract}{%
    \fancyhead{}	
    \fancyfoot{} 
    \fancyfoot[C]{}
	\renewcommand{\headrulewidth}{0pt} 
	\renewcommand{\footrulewidth}{0pt} 
}
%
\usepackage{layout}
\usepackage{tocbibind} % include tableofcontent in itself

%%%%%% PAGE DE GARDE %%%%%%

\geometry{outer=2cm,inner=3cm,top=3cm}
\usepackage[scaled]{helvet} % font used on cover (Helvetica)
\usepackage{eso-pic} % to set background picture
\usepackage{multicol} % for back cover (abstracts)
\usepackage{graphicx} % to include logos
\usepackage{tikz} % to compose background picture

% Colors (extracted from SPI's template)
\definecolor{boxcolor1}{rgb}{0.91373,0.92941,0.87451}
\definecolor{boxcolor2}{rgb}{0.94902,0.93333,0.91373}
\definecolor{boxcolor3}{rgb}{0.76078,0.87843,0.17647}
\definecolor{headercolor}{rgb}{0.94118,0.30980,0.17255}
\definecolor{namecolor}{rgb}{1.0,0.4,0.0}
\definecolor{titlecolor}{rgb}{0.19216,0.51765,0.60784}
% Also used: gray, teal (predefined by xcolor package, usually loaded by document class)

% Cover environment, to keep changes local
\newenvironment{cover}{%
  \fontfamily{phv}\selectfont % Select Helvetica font
  \pagestyle{empty} % No page number
}{
  \addtocounter{page}{-1}
  \cleardoublepage
}

% Macro for background common to front and back
\newcommand{\tikzBG}{%
  \path (0,0) rectangle (1,1);
  %TODO: You should adjust the bottom height of the following rectangle to fit your abstract's length
  \path [fill=boxcolor1] (.0571,.11) rectangle (.481,.963); 
  \path [fill=boxcolor2] (.4333,.697) rectangle (.9048,.7475);
  \path [fill=boxcolor2] (.4333,.7811) rectangle (.9048,.8316);
  \path [fill=boxcolor2] (.4333,.8687) rectangle (.9048,.9192);
  \path [fill=boxcolor3] (.0571,.7879) rectangle (.5762,.8316);
  \node[inner sep=0pt] at (0.2285,0.8788) [above left] {%
    \includegraphics[height=.0707\paperheight,keepaspectratio]{./figures/logo/logo_unb.png}};
  \node[inner sep=0pt] at (0.6667,0.8788) [above right] {%
    \includegraphics[height=.0808\paperheight,keepaspectratio]{./figures/logo/logo_ecn_color.png}};
  \node at (.0571,.8316) [above right,color=headercolor] {%
    \fontsize{29}{35}\selectfont\bfseries Th\`ese de Doctorat};
}

% Macro for repeated information (to avoid insconsistency)
%TODO: fill in with no formatting but desired case
\newcommand{\firstName}{Jean-Rémy}
\newcommand{\surname}{Gloaguen}
\newcommand{\thesisTitle}{Estimation du niveau sonore de sources d'intérêts au sein de mixtures sonores urbaines : application au trafic routier}

%%%%%%% SYMBOLES %%%%%
\usepackage{tipa}	% pour avoir l'accent concave
\usepackage{lmodern}	% pour les guillemets
\usepackage{gensymb}	% pour les degrés
\usepackage{enumitem}	% pour changer le symbole de l'item (\begin{itemize}[label=$\bullet$])

%%%%%%% EQUATION %%%%%%
\usepackage{amssymb}
\usepackage{amsmath}
\usepackage{fancybox}
\usepackage{xfrac}	% fraction de type "1/4"
\usepackage{cases}	% système équation
\usepackage[overload]{empheq}
\usepackage{bm}		% pour mettre en gras .
\usepackage{units} 	% x/y barre latérale pour les fractions
%
%%%%%%% FIGURE %%%%%%
\usepackage{subfigure}	% utiliser subfigure
\usepackage{float}	% utiliser H dans les figures
%
%%%%%% TABLEAUX %%%%%%
\usepackage{array,multirow,makecell}
%\addto\captionsfrench{\def\tablename{\textsc{Tableau}}}% pour avoir TABLEAU et pas TABLE dans les légendes des tableaux
\usepackage{colortbl} % pour avoir des lignes colorées dans les tableau
%\usepackage{slashbox} % pour les \backslashbox
%\usepackage{subcaption}
\usepackage{hhline}	% pour les lignes horizontales 
\usepackage{tabularx} % permet itemize dans les cellules
\usepackage{booktabs}
\usepackage{longtable}	% pour les tableaux longs

\newcolumntype{L}[1]{>{\raggedright\let\newline\\\arraybackslash\hspace{0pt}}m{#1}}
\newcolumntype{C}[1]{>{\centering\let\newline\\\arraybackslash\hspace{0pt}}m{#1}}
\newcolumntype{R}[1]{>{\raggedleft\let\newline\\\arraybackslash\hspace{0pt}}m{#1}}

%%%%% ALGORITHME %%%%%
\usepackage{algorithm}
\usepackage{algorithmic}

%%%%% BIBLIO %%%%%
\usepackage[fixlanguage]{babelbib}
\selectbiblanguage{french}
\usepackage{breakcites}	% pour couper les références en bout de ligne

%%%%% APPENDICES %%%%%%%
\usepackage[toc,page]{appendix}

%%%%%%%%%%%%%%%%%%%%%
\usepackage{url}	% gérer les adresses www.
\linespread{1.2}	% interligne

\newcommand{\ml}[1]{\textcolor{red}{ML : #1}}

\cleardoublepage
%%
%%\begin{document}

\chapter{Connaitre l'environnement sonore urbain : de la prédiction à la mesure}\label{chap:modele}
\thispagestyle{empty}

Dans ce chapitre, une présentation des méthodes utilisées pour connaitre l'environnement sonore urbain est réalisée. Dans une première partie, le problème générale est posé formellement, puis l'utilisation de modèles prédictifs et les éléments de cartographie de bruit en ville sont exposés et enfin la réalisation de mesures et l'installation de réseaux de capteurs sont présentées. Enfin, la problématique générale est énoncée et une solution est proposée.

\section{Définition formelle du problème}

Soit $M_{i}(t)$, un environnement sonore urbain (abrégé ESU) défini dans un espace $\Omega$, capté en un point donné $i_{\in \Omega}$, reçu à un instant $t$. L'ESU se décompose alors comme la somme de $N$ différentes contributions $S_i(t)$ reçues au point $i$. Chacune de ces contributions, est le résultat de l'émission sonore d'une source acoustique $s_j$, de puissance sonore $L_{w,j}$, située à la position $x_j$ et émise à un instant $t$,  qui s'est propagée dans l'environnement urbain. L'ESU s'exprime alors, mathématiquement, dans le domaine temporel comme : 

\begin{align}
M_i(t) &= \sum_{j = 1}^{N}S_j(t), \\
 & = \sum_{j = 1}^{N} s_j(t) \ast \delta_{ij}(t) \label{eq:convolution_ESU}\\
 & = \sum_{j = 1}^{N} \sum_{k = 1}^{+\infty} s_j(t-\tau_{ijk}) \delta_{ijk}\label{eq:propagation}
\end{align}

Dans l'équation \ref{eq:convolution_ESU}, le produit de convolution de la source $s_j(t)$ par la variable $\delta_{ij}(t)$ traduit l'intégralité des effets de propagation généré par la diffusion du son dans l'environnement $\Omega$ jusqu'au récepteur $M_i(t)$. Ils incluent les phénomènes d'atténuation géométrique de l'onde sonore ainsi que ceux de diffusion et d'absorption provoquées par les réflexions de l'onde sonore sur les parois des bâtiments et sur le sol.
Dans l'équation \ref{eq:propagation}, $\delta_{ij}(t)$ est décomposé afin d'exprimer l'ensemble des chemins de propagation $k$ parcourus par l'onde sonore. Chaque voie possède alors une atténuation $\delta_{ijk}$ et un déphasage temporel $\tau_{ijk}$ qui lui est propre. 
La mixture $M_{i}(t)$ peut alors soit être captée par un microphone installé en ville (comme dans l'exemple en Figure \ref{fig:schema_ville}), ce qui permet notamment d'obtenir des indicateurs physiques (niveau sonore en dB SPL ou pondéré A), soit être perçue par les citadins où les aspects perceptifs entre alors en jeux. Dans ce cas, la mixture $M_{i}(t)$ est évaluée à travers des indicateurs perceptifs comme l'\textit{agrément sonore} qui dépend notamment de la prédominance de certaines sources sonore et du cadre environnemental dans lequel elles sont perçues. L'ensemble de ces variables est représenté dans la Figure \ref{fig:schema_ville}. 

\begin{figure}[hbtp]
\centering
\includegraphics[width=.9\linewidth]{./figures/autres/schema_ville_propa.pdf}
\caption{Schéma du problème considéré en ESU pour un signal capté par un microphone au point $\mathbf{x}$. Deux sources sonores émise à l'instant $t-\tau_{ijk}$ à la position $x_j$ sont présentes : une voiture, $s_{1}$ (résumée en une source ponctuelle symbolisée par un cercle rouge), et un piéton, $s_2$, (résumée en une source ponctuelle symbolisée par un cercle bleu). Chaque source se propage jusqu'au récepteur selon 2 chemins (champ direct $\delta_{ij1}$ et champ réfléchi $\delta_{ij2}$). Le signal capté $M_{i}(t)$ correspond alors à la somme des 2 sources sonores filtrés.}
\label{fig:schema_ville}
\end{figure}

Les sources $s_j(t)$ incluent les différentes sources sonores présentes en ville comme le trafic aérien ou ferroviaire mais aussi les voix, les bruits de pas, celui d'une valise à roulettes, les sifflements d'oiseaux, les aboiement de chiens \dots{} Chacune de ces sources présente des allures temporelles et fréquentielles différentes. La source sonore principale en ville est alors le trafic routier qui est considéré comme la somme des contributions des $M$ véhicules présents en ville et dont la somme globale s'exprime :

\begin{subequations}
\begin{align}
S_{tr}(t) &= \sum_{j = 1}^M S_{v_j}(t)\\
 & = \sum_{j = 1}^M s_{v_j}(t) \ast \delta_{j}(t)
\end{align}
\end{subequations}

où $s_{v_j}(t)$ correspond à l'émission sonore du véhicule $j$. L'émission sonore global d'une voiture a plusieurs origines : bruit du moteur thermique, aérodynamique et de roulement. Ici, puisque les citadins ne réalisent pas cette séparation mais considèrent l'ensemble de comme un tout, on résume la source \textit{voiture} $s_{v_j}(t)$ comme l'ensemble de ses bruits. Précisons que le son émis par le klaxon du véhicule n'est pas considéré dans la source \textit{voiture} car il appartient à la catégorie des avertisseurs sonores.
L'ESU peut ainsi s'exprimer : 

\begin{equation}
M_i(t) = S_{tr}(t)+\sum_{j = 1}^{N-M}S_j(t)
\end{equation}
\\

et est donc le résultat d'un ensemble de sources variées, ayant une allure temporelle, fréquentielle ainsi qu'un niveau sonore propre, et qui sont émises dans un environnement où des phénomènes de propagation complexes ont lieux. Plusieurs approches sont possibles pour les définir selon les choix de représentations voulues (physique ou perceptive). Les questions soulevées sont alors : 
 
\begin{itemize}
\item \textbf{Quels sont les moyens disponibles pour déterminer les ESU dans leur globalité ?}
\item \textbf{Comment estimer la présence et le niveau sonore du trafic routier ? Peut-on connaitre la contribution des autres sources sonores ?}\\
\end{itemize}

Avant de répondre à ces questions, il nécessaire, dans un premier temps, d'en présenter les motivations.

\section{De l'intérêt d'étudier les environnements sonores urbains}

Au sein de l'Union Européenne, 70 $\%$ de la population, soit quasiment 340 millions d'habitants, vivent dans des zones urbaines \cite{europ-commission_data_2017}. 486 villes concentrent, chacune, plus de 100 000 habitants. En France, selon l'INSEE, c'est même plus de 84 $\%$ de la population qui vivent dans une zone urbaine, soit plus de 55 millions d'habitants. Cette concentration soulève de grandes questions autour de l'organisation de l'espace urbain afin d'offrir une qualité de vie acceptable aux citadins. En effet, avec de telles densités (environ 3000 habitants/km$^2$ et jusqu'à plus de 21 000  habitants/km$^2$ pour la ville de Paris, la plus dense de l'UE), plusieurs formes de pollutions viennent dégrader l'environnement urbain. Des sources de désagrément perçues par le citadin, le bruit est le phénomène qui provoque le plus de gène après la pollution de l'air. Ce bruit est le fruit des activités humaines, provenant essentiellement du transport qu'il soit routier, ferroviaire ou aérien \cite{zannin_characterization_2013}.\\

%\ml{je suis pas trop pour la mise en capitale pour introduire les acronymes}

Selon un rapport de l'Organisation mondiale de la Santé (OMS) \cite{who_burden_2017}, en Europe, près de 200 millions de personnes sont exposées quotidiennement à des niveaux sonores supérieurs à 55 dB($A$), soit 40$\%$ de la population. Près de 20 $\%$ atteignent même plus de 65 dB($A$) en journée et plus de 30 $\%$ sont touchées par un niveau sonore excédant 55 dB($A$) la nuit. En France, selon un rapport de l'ADEME \cite{europeens2016analyse}, ce sont 52 millions de personnes qui se disent affectées par le bruit et principalement le bruit issu du trafic routier. Plus de 7 millions d'individus y sont alors exposés à des niveaux supérieurs à 65 dB($A$) au quotidien et à plus de 55 dB($A$) la nuit.
Cette exposition quotidienne, à de tels niveaux, n'est pas sans conséquence pour l'être humain. L'impact sur l'organisme humain dû à l'exposition du bruit est observé et étudié depuis de nombreuses années \cite{ising1980health}. Parmi les effets possibles, les plus couramment relevés sont des troubles du sommeil \cite{pirrera2010nocturnal}, de la vigilance et de la concentration, l'augmentation du stress, de la pression artérielle et du rythme cardiaque \cite{babisch2008road, babisch2005traffic}. Selon le rapport de l'OMS, ce sont près de 8 millions de personnes en Europe qui sont affectées par des troubles du sommeil mais aussi 900 000 touchées par de l'hypertension. On estime aussi que 43 000 hospitalisations sont imputables au bruit dues à des pertes de vigilance et de concentration et jusqu'à 10 000 cas de morts prématurées par an. Cet impact sur la santé a alors un coût financier pour la société : en France, ce coût est estimé à plus de 11,5 milliard d'euros par an dont une grande partie (89 $\%$) est imputable au bruit du trafic routier \cite{europeens2016analyse}. De plus, si le bruit en ville impacte la vie des citadins, celui-ci se fait également ressentir auprès de la faune sauvage \cite{dutilleux_anthropogenic_2012, francis2009noise} leur causant également du stress ou en compliquant la communication entre les individus et leur reproduction.\\

Le bruit, et une trop grande exposition à celui-ci, a donc un impact négatif sur les individus et sur leur environnement. C'est donc un enjeu de société dont les français ont pleinement conscience \cite{JNA2016etude}.
Il est donc nécessaire et utile de s'intéresser aux environnements sonores urbains (ESU) afin d'estimer les sources sonores présentes, leurs niveaux sonores, leurs répartitions et ainsi réduire et limiter leurs impacts sur les populations urbaines.

\section{Utilisation de modèles prédictifs pour estimer l'ESU}

L'une des premières pistes pour évaluer les ESU est l'utilisation de modèles prédictifs de bruits. L'objectif est alors de générer des lois d'émission sonores afin de déterminer les sources $s_i$ et des lois de propagation $\delta(t)$. De nombreux modèles d'émission existent afin de prédire les niveaux sonores émis par le trafic routier \cite{quartieri2009review}, ferroviaire \cite{van2000railway} et aérien \cite{zaporozhets1998aircraft}. Étant la source sonore la plus présente et la plus gênante, ce sont les modèles dédiés au trafic routier qui sont ici présentés.


\subsection{Modèle d'émission du trafic routier}

Plusieurs modèles visant à estimer la puissance acoustique, $L_w$, émise par les véhicules et leur niveaux sonores, $L_p$, reçu en un point ont été développés depuis plus de 30 ans, plusieurs pays ayant alors leur propre modèle (RLS-90 en Allemagne, CNR en Italie, NMPB-Routes en France, CoRTN au Royaume-Uni, Nord 2000 pour les pays Scandinaves). On se propose tout d'abord de décrire succinctement 3 modèles : HARMONOISE/Imagine, NMPB-routes-2008) et CNOSSOS-EU, le modèle européen le plus récent. 

\begin{itemize}
\item Le modèle HARMONOISE/Imagine \cite{jonasson2004source} a été développé afin d'offrir un premier modèle d'émission harmonisé à l'échelle des pays membres de l'UE. 5 catégories de véhicules sont considérés (véhicule légers, médium,  lourd, poids lourds et deux roues) pour des vitesses constantes, en phase d'accélération et de décélération. Chaque véhicule est décrit en deux sources ponctuelles, chacune ayant une puissance sonore qui se décompose en deux contributions : \og bruit de roulement \fg{}, $L_{w_R}$, et \og bruit de moteur \fg{}, $L_{w_M}$,  

\begin{align}
L_{w_R}(HARMONOISE) &= a_R(f)+b_R(f)\log\left(\frac{v}{v_{ref}}\right),\\
L_{w_M}(HARMONOISE) &= a_M(f)+b_M(f)\log\left(\frac{v-v_{ref}}{v_{ref}}\right).
\end{align}

où $v_{ref}$ est une vitesse de référence fixée à 70 km/h. Les valeurs des coefficients $a_R(f)$, $b_R(f)$, $a_M(f)$ et $b_M(f)$ sont données selon les bandes de tiers d'octave (entre 25 Hz et 10 kHz) et la catégorie du véhicule. La propagation du son est réalisée en considérant plusieurs approches (résolution de l'équation parabolique, méthode des éléments de frontières ou tir de rayons), afin de s'adapter à différentes configurations, en considérant des conditions atmosphériques homogènes. Enfin, l'influence de la route est pris en compte selon son revêtement (température, âge, humidité). Cette approche est reconnu pour être plus \og physique \fg{} mais avoir un temps de calcul plus long que les autres modèles (jusqu'à 50 fois plus long) \cite{probst2011comparison}.

\item La NMPB-routes-2008 est un modèle français développé à partir des années 90 \cite{setra_prevision_2009-1, setra_prevision_2009-2}. 2 catégories sont, ici, considérés (véhicules légers et poids lourds) pour des vitesses constantes et en phase d'accélération et de décélération également.
La puissance de la source sonore est considérée par mètre et par véhicule sur une ligne source et se divise en deux composantes là aussi : une composante \og bruit de roulement \fg{} et une composante \og bruit de moteur \fg{}.

\begin{equation}\label{eq:puissance_NMPB}
L_w(NMPB) = 10\times \log \left(10^{L_{w_r}/10}+10^{L_{w_m}/10}\right).
\end{equation} 

exprimées en bandes d'octaves entre 125 Hz et 4000 Hz. Les valeurs sont déduites en fonction du type de revêtement et de véhicule. L'estimation de ces composantes est réalisée à partir de mesures moyennées et divisées selon l'allure du véhicule (vitesse constante, accélération, décélération). La méthode de propagation choisie est celle des tirs de rayons où les chemins directs, réfléchis et diffractés sont considérés entre la source et le récepteur. En fonction des conditions atmosphériques relevées (température, vent), les atténuations dans les conditions favorables (à la propagation) et homogènes sont considérées. Les effets de sol (3 types de routes considérés, dont les propriétés varient selon leur ancienneté et la température ambiante de l'air), la divergence géométrique et l'absorption atmosphérique sont ensuite pris en compte.

\item Le modèle CNOSSOS-EU \cite{CNOSSOS} est le modèle développé le plus récent, là aussi afin d'harmoniser le modèle d'émission sonore et de propagation à l'échelle de l'UE, inspiré par les modèles précédents. L'ensemble du parc automobile est classé selon 5 catégories  dont une permettant d'inclure les véhicules électriques. Chaque véhicule est résumé résumé en une source équivalente ponctuelle dont la puissance acoustique est :  
 
\begin{equation}\label{eq:puissance_CNOSSOS}
L_w(CNOSSOS) = L_w + \log\left(\frac{Q_m}{1000\times v_m} \right)
\end{equation}

où $Q_m$ est le débit véhicule de la catégorie m par heure et $v_m$ est la vitesse moyenne (km/h) calculé en bande d'octave entre 125 Hz et 4000 Hz. $L_w$ est défini selon l'équation \ref{eq:puissance_NMPB}. La relation \ref{eq:puissance_CNOSSOS} est définie pour une vitesse constante mais des corrections peuvent être apportées afin de prendre en compte les effets de l'accélération et de décélération. Enfin, les aspects liés à la propagation du son sont les mêmes que dans la NMPB-routes-2008. 
\end{itemize}

Une comparaison exhaustive de 8 modèles (FHWA, CoRTN, RLS-90, ASJ, HARMONOISE/Imagine, Son Road, Nord 2000 et NMPB-Routes-2008) a été réalisé par Garg et Maji \cite{garg_critical_2014} selon de nombreux critères :

\begin{itemize}
\item modélisation des sources sonores (trafic, ferroviaire),
\item condition de trafic (constantes, accélération/décélération, intersection\dots),
\item modèle de propagation,
\item prise en compte de la divergence géométrique,
\item données d'entrées prise en compte,
\item modélisation des effets de sol,
\item effets météorologiques,
\item effets de diffraction,
\item \dots  \\
\end{itemize}

À travers leur comparaison, les auteurs relèvent de nombreuses différences notamment auprès des modèles de propagations du son. 
L'ensemble des divergences amènent donc à les auteurs à considérer à utiliser le modèle harmonisé CNOSSOS-EU, souhaité par Steele, comme une solution satisfaisante. Les auteurs de l'étude précise tout de même qu'il est difficile de déterminer un \og meilleur \fg{} modèle par rapport aux autres, chacun ayant une approche différente. De plus, même si les modèles ont été validé sur des cas simple, leur validation par des mesures \textit{in situ} reste compliquer à réaliser en ville en raison des nombreuses autres sources sonores présentes en villes. 


\subsection{Réaliser des cartes du bruit de trafic en ville}

À partir de ces modèles, est instaurée en 2002, la directive européenne 2002/49/DE \textit{relative à l'évaluation et à la gestion du bruit dans l'environnement} dont le but est de mieux connaitre la répartition des niveaux sonores généré par le trafic routier, ferroviaire et aérien ainsi que par les Installations Classées pour la Protection de l'Environnement (ICPE) dans les agglomérations de plus de 100 000 habitants. Cette directive prévoit :

\begin{itemize}
	\item d'évaluer l'exposition au bruit des populations basée sur des méthodes communes aux pays européens,
	\item d'informer les populations sur leur niveau sonore d'exposition et sur les effets du bruit sur la santé,
	\item de connaitre et de délimiter les zones bruyantes et les zones calmes.\\
\end{itemize}

Cette directive se traduit notamment par la production de cartes de bruits stratégiques, pour chacune des 4 sources sonores ciblées, afin de déterminer les endroits où les niveaux sonores sont élevés. Un résumé des étapes permettant la réalisation des cartes de bruit du trafic routier est présenté en Figure \ref{fig:cartographie}.\\

% Ces cartes aident ainsi à la mise en place d'aménagements qui permettront de les réduire (construction d'un mur anti-bruit, réduction de la vitesse, mise en place d'un revêtement sol particulier \dots). L'intérêt des modèles prédictifs est alors de pouvoir prédire l'impact de ces aménagements sur l'évolution des niveaux sonores.

\begin{figure}[h]
\centering
\includegraphics[width=.85\linewidth]{./figures/cartographie/cartographie.pdf}
\caption{Résumé des étapes menant aux cartes de bruit de trafic routier.}
\label{fig:cartographie}
\end{figure}

Dans un premier temps, plusieurs indicateurs en données d'entrée des modèles sont relevés \textit{in situ} :  

\begin{itemize}
\item vitesses moyennes des véhicules sur les portions de routes principales,
\item débits de véhicules (nombre de véhicules par tranche horaire),
\item composition du trafic (nombre de véhicules légers et de poids lourds).
\item architecture et topographie de la ville (revêtement au sol), 
\item conditions météorologiques (températures, vent).\\
\end{itemize}

\begin{figure}[t]
\centering
\subfigure[\label{fig:lden}]{\includegraphics[width=0.85\linewidth]{./figures/cartographie/Lden_ile_Nantes.PNG}}
\subfigure[\label{fig:ln}]{\includegraphics [width=0.7\linewidth]{./figures/cartographie/Ln_ile_Nantes.PNG}}
\caption{$L_{DEN}$ \subref{fig:lden} et $L_N$ \subref{fig:ln} de l'île de Nantes pour le trafic routier \cite{nantes_carte}.}
\label{fig:carto_nantes}
\end{figure}

Puis à partir d'un modèle d'émission sonore et de ces données d'entrée, il est possible d'établir la puissance émise par chaque véhicule, $\hat{s}_{v}(f)$.
Enfin, l'influence de l'environnement $\hat{\delta}(t)$ sur la répartition des niveaux sonores dans la ville est calculé à l'aide d'un logiciel numérique (Mithra, Immi, CadnAa \dots). Les logiciels accompagnés d'un Système d'Information Géographique (SIG) sont ceux qui offrent actuellement le plus de possibilités. Un SIG est un outil informatique conçu pour stocker, analyser et manipuler plusieurs type de données spatiales et géographiques comme l'architecture des villes ou le nombre d'habitants présents. Son utilisation permet alors de connaitre plus facilement le nombre de citadins exposés à des forts niveaux sonores. \\

Les cartes de bruits produites résument les niveaux sonores équivalent pondérés $A$ sur 24h ($L_{DEN}$ pour \textit{Day-Evening-Night} (\textit{Jour-Soir-Nuit} en français)) et durant la nuit ($L_N$) :

\begin{equation}
L_{DEN} = 10\times\log \left(\frac{1}{24} \left(12\times10^{\frac{L_D}{10}}+4\times10^{\frac{L_E+5}{10}}+8\times10^{\frac{L_N+10}{10}} \right)\right)
\end{equation}

avec $L_D$, $L_E$ et $L_N$, les niveaux sonores équivalent pondéré A pour les périodes respectives 6h-18h, 18h-22h, 22h-6h (pouvant être changées suivant le rythme de vie des habitants de la zone considérée),

\begin{subequations}
\begin{align}
L_D &= 10\times\log\left(\frac{1}{T} \sum_{t = 1}^{T}10^{\frac{L_{D_t}}{10}}\right),\\
L_E &= 10\times\log\left(\frac{1}{T} \sum_{t = 1}^{T}10^{\frac{L_{E_t}}{10}}\right),\\
L_N &= 10\times\log\left(\frac{1}{T} \sum_{t = 1}^{T}10^{\frac{L_{N_t}}{10}}\right).
\end{align}
\end{subequations}

Les niveaux $L_E$ et $L_N$ sont majorés respectivement de 5 dB($A$) et de 10 dB($A$) afin de pénaliser les plages horaires où la gêne occasionnée par le trafic est plus importante. La Figure \ref{fig:carto_nantes} résume, par exemple, le $L_{DEN}$ et le $L_N$ pour le bruit du trafic routier dans un quartier de la ville de Nantes.
La réalisation de ces cartes permet alors d'estimer le nombre  de personnes touchées par ces niveaux sonores élevés et facilite ainsi la mise en oeuvre de trvaux d'aménagement (construction de mur anti-bruit, chanegement de revêtement, diminution des vitesses \dots) si ceux-ci sont trop élevé selon la réglementation en vigeur. L'utilisation de modèle prédictif présente alors l'intérêt de pouvoir simuler l'effet de ces dispositifs sur les niveaux sonores émis \cite{murphy2011scenario,guedes2011influence}. Les cartes produites doivent être ensuite mises à jours tous les 5 ans.


\subsection{Vers la modélisation d'autres sources sonores ?}

Le trafic routier, aérien et ferroviaire et les ICPE sont actuellement les seules sources sonore à faire l'objet de cartes de bruit. Mais il est tout à fait possible d'ouvrir ces outils à d'autres sources sonores. 
En effet si le trafic est la source de bruit principale dont les citadins se plaignent le plus, ces derniers sont tout de même sensible à la présence d'autres sources sonores comme la voix, le chant d'oiseaux ou le bruit d'une fontaine et que ces sources participent à l'agrément sonore des ESU. Il peut donc être intéressant et utile de savoir représenter leur présence dans les villes. 
Certaines sources sonores ont déjà fait l'objet de modélisation (dans le cadre d'autres études ou applications) comme les oiseaux \cite{nemeth2013bird}, les voix \cite{hayne2011prediction} ou les fontaines \cite{watts2009measurement}, mais ne sont, pour l'instant pas assez étudiées pour offrir des modèles validés. 
Une des difficultés est la localisation de certaines de ces sources dans l'espace urbain. Si les sources sonores fixes comme les fontaines ou les cloches d'une église sont faciles à localiser, d'autres, comme la foule et les oiseaux, sont plus difficiles à déterminer car plus mobiles et parcimonieuses. Dans \cite{aumond2018probabilistic}, c'est par une approche statistique que la position des sources est déterminée : les piétons sont ainsi plus susceptible de se trouver sur des places ou le long des trottoirs, dans les parcs ou auprès des arbres pour les oiseaux. En combinant l'ensemble de ces sources, il deviendrait envisageable de générer des cartes dites multi-sources d'une ville. 
De plus, leur prise en compte de l'ensemble de ces sources ouvre la voie vers des cartes descriptives de l'ESU comme dans \cite{can_describing_2015} où des indicateurs physiques classifient les ambiances sonores ou bien encore selon une représentation perceptive de l'ESU. Dans \cite{aumond2017modeling}, l'évaluation perceptive d'ESU, menée auprès de citadins, y est corrélée à des indicateurs physiques (niveau sonore fractile $L_{50}$ dans la bande de tiers d'octave de 1 kHz ainsi que la variation normalisée en temps et en fréquence dans les bandes de 500 Hz et de 4 kHz). Ces indicateurs physiques peuvent alors être estimés à partir de l'ensemble des sources modélisés et ainsi être lié à la perception du citadin.


%De plus, à partir des études perceptives réalisées, il deviendrait possible de lier les niveaux physiques prédit à des indicateurs perceptifs et ainsi générer des cartes de bruits perceptives qui serait lié à la perception du citadin de son environnement. Dans \cite{aumond2017modeling}, l'évaluation perceptive de citadins d'ESU est corrélé à des indicateurs physiques deux modèles d'agrément sonore sont proposés issus d'enregistrements sonores et des évaluations perceptives de citadins. L'un se base sur des grandeurs physiques tels que le niveau sonore fractile $L_{50}$ dans la bande de tiers d'octave de 1 kHz ainsi que la variation normalisée en temps et en fréquence des bandes de 500 Hz et de 4 kHz, le second est établi à partir du niveau sonore globale et du temps de présence de plusieurs source sonore spécifiques : trafic, voix, et oiseaux.
%Ce modèle perceptif est intéressant car il ne lie pas la perception du citadins qu'à des indicateurs acoustiques mais à certaines sources sonores et à leur prépondérance. dans l'environnement.\\


\subsection{Limitations des modèles prédictifs}

L'utilisation de modèles d'émission sonores présente donc plusieurs intérêts : prédiction des niveaux sonores dans une situation donnée, possibilité de tester différents scénarios d'aménagements, création de cartes de bruit de trafic. Leur utilisation s'est répandue et démocratisée en raison de l'introduction de la directive européenne. Les premières études réalisées suivant les recommandations de la directive ont permis de soulever plusieurs limites à cette méthode comme celles lié au choix du modèle parmi ceux existants, aux incertitudes liées aux estimations des données d'entrées.

\subsubsection{Limites liées à la modélisation de la source et de la propagation}

Ces modèles sont basés sur des mesures et des études physique du trafic. Comme tout modèle simulant la réponse d'un système physique, il n'est pas possible d'en générer un universel qui puisse s'adapter à l'ensemble des cas et scénarios possibles. Souhaitant généraliser un ensemble de situation possible, des simplifications sont réalisées par exemple en classifiant les routes et du parc automobile en un nombre réduit de catégorie ou bien en considérant soit des conditions atmosphérique homogène soit des conditions homogènes et favorables. Ce sont d'autant de simplifications qui, certes, facilite l'implémentation et l'utilisation des modèles mais qui sont aussi vecteurs d'incertitudes. De plus, dans dans les cas où un élément ne correspond pas à ceux définis, ces modèles échouent à prédire des niveaux sonore corrects.

La complexité des modèles existant basés sur des hypothèses différentes a permis la formation de nombreux modèles selon les pays. Avant même la mise en place de la directive, \cite{steele_critical_2001} les avait comparé selon différents aspects (données d'entrée, type de cartographie, méthode de propagation des différents logiciels). Parmi cette diversité d'outils, l'auteur met en avant le problème, soulevé également par \cite{king_implementation_2011}, de la diversité des outils et des méthodes de calculs qui peuvent être employées : quelle méthode doit-on utiliser ? Les premières cartes de bruits générées dans le cadre de la directive européenne ont donc été établies sur des modèles différents : dans truc muche c'est telle méthode qui est employé, alors que dans truc chouette c'est cette méthode.  \cite{murphy_environmental_2006, murphy_estimating_2009}.
Afin de résoudre ces problèmes, la méthode CNOSSOS-EU \cite{CNOSSOS,kephalopoulos} a été développée basé sur les méthodes déjà existantes permettant d'harmoniser la construction des cartes de bruit des villes à l'échelle européenne permettant plus facilement leur comparaison. 

Si l'harmonisation des méthodes est bénéfique, la confrontation des modèles face à des mesures faites en villes. L'ensemble de ces outils ont été développé et validé à partir de mesure faites dans des conditions optimales afin de déterminer des lois d'émission sonore et de propagation générique et adapté à la ville. Toutefois, la comparaison entre les niveaux sonores calculés et mesuré est délicate. Premièrement les mesures présentent l'inconvénient d'être soumises à d'autres sources sonores qui ne sont pas lié au trafic et qui viennent donc détériorer les mesures. De plus, il faut s'assurer que les données d'entrée fournit aux modèles correspondent bien aux conditions expérimentales ce qui n'est pas facile.

Enfin, ces modèles dépendent des données d'entrée relevés \textit{in situ} qui s'exprime sous la forme de moyennes qui induisent donc des écarts-types qui se propagent dans les étapes suivantes. L'ensemble de ces écarts-types se propage alors sur l'ensemble de la chaine de calculs. La propagation des incertitudes a été étudié dans \cite{probst2005uncertainties}, l'auteur y concède qu'estimer l'incertitude $\sigma$ à un niveau sonore à un point donné est un processus long et difficile.

\subsubsection{Limites liées à la simulation et à la représentation}

Dans le cas de la cartographie de bruit, les modèles de sources et de propagation sont implémentés dans des logiciels pour déterminer, sur l'ensemble d'une ville ou d'un quartier, les niveaux sonores liés au trafic. Cette tâche peut être très lourde en coût et en temps de calculs (de quelques heures à plusieurs jours). Pour cela, l'environnement urbain est simplifié en le discrétisant, le plus souvent par un maillage régulier de 10 mètres par 10 mètres. À l'échelle d'une ville, c'est ainsi plusieurs millions de points qui peuvent être définis. Le calcul consiste ensuite à estimer la propagation acoustique entre les différentes sources et les points de ce maillage, selon les différents chemins possibles, pour chaque bande d'octave ou de tiers d'octave.
Cette étape de discrétisation, si elle simplifie la forme de l'environnement urbain, présente l'inconvénient de ne pas prendre en compte les multiples variations géométriques des façades ou bien l'ensemble des petits mobiliers urbains qui ont un rôle dans la diffusion du son. 
De plus, en raison de la discrétisation du milieu, des méthodes d'interpolation (méthode linéaire, de krigeage) sont utilisées pour calculer les niveaux sonores entre ces points, ce qui est également source d'approximation qui peuvent mener à de mauvaises interprétations \cite{van_leeuwen_noise_2015}.

Enfin, une fois la carte générée, on peut noter que seulement 2 niveaux sonores par source de trafic sont obtenus, $L_{DEN}$ et $L_N$, et mis à jour tous les 5 ans seulement. C'est donc une information statique et restreinte que la directive européenne impose Cependant, le trafic routier, ferroviaire et aérien varient aussi bien à l'échelle de l'année, d'une journée ou même d'une heure \cite{lv2015traffic}. Il est envisageable de calculer des cartes de bruit pour, par exemple, chaque heure de la journée, en adaptant les données d'entrée, mais c'est une opération qui reste très longue et couteuses à réaliser.
L'utilisation de modèles dynamiques de trafic, couplés aux modèles d'émissions sonore  \cite{can2010traffic}, n'est actuellement pas destiné à la cartographie des ESU mais est une piste envisageable pour lier les interactions entre les véhicules et leur cinématique à l'échelle d'une rue ou d'un quartier.

En conclusion, l'utilisation de modèles prédictifs est une approche utile pour réaliser une première estimation de la répartition du bruit de trafic en ville. Si elle présente des avantages (estimation d'un niveau physique à l'échelle de la ville, possibilité de tester des scénarios d'aménagement), elle présente plusieurs limites : 

\begin{itemize}
\item le nombres de sources est, pour l'instant, trop restreint et ne permet pas de considérer l'ensemble des ESU et la perception qu'en ont les citadins,
\item l'utilisation d'un modèle d'émission et de propagation et la calcul numérique entraine de nombreuses approximations qui engendre des incertitudes qui sont difficiles à estimer, 
\item l'accès à leur évolution temporelle des sources sonores n'est pas possible.
\end{itemize}

Afin de considérer l'ensemble des sources et des évènements sonores présents en ville et de compléter l'apport des modèles prédictifs, une autre approche est envisagée, basée sur la réalisation de mesures et d'enregistrements sonores.

\section{Utilisation de mesures acoustiques}

À la différence des modèles prédictifs qui déterminent l'émission sonore des sources $s_i$ et leur propagation $\delta_i$ pour déterminer le niveau sonore en un point $\mathbf{x}$, la réalisation de mesures donne directement accès à la mixture globale $M_{\Omega}(t,\mathbf{x})$. Les limites liées à la modélisation des sources, de la propagation du son et de l'ensemble des environnements urbain dans toute leur complexité n'interviennent alors plus dans la mesure faite \textit{in situ}.

De nombreuses études ont été menées en ville faisant intervenir des mesures pour des durées plus ou moins longues (de quelques jours \cite{romeu2011street} à plusieurs années \cite{gaja2003sampling}) avec des microphones de mesures de hautes qualités. Dans une première étude \cite{zannin2002environmental} a réalisé une série de mesures sur près de 1000 positions dans la ville de Curitiba, au Brésil, pour étudier l'ESU dans sa globalité afin d'évaluer l'exposition aux bruits dans les différents quartier de la ville, puis a réduit la surface d'étude dans \cite{zannin_characterization_2013} où 58 points de mesures sont déployés dans le campus universitaire de la ville. 
Dans \cite{Mioduszewski}, 40 microphones sont placés isolément à travers la ville de Gdansk en Pologne pour une durée d'un an afin pour mesurer le niveau sonore du trafic et valider la cartographie de bruit. Il y relève notamment la différence entre les niveaux calculé et mesurés qu'il impute à la non prises en compte de certain types de bruits dans les modèles et à l'intégration de sources sonores qui ne sont pas du trafic dans l'estimation des niveaux sonores.
En parallèle, des outils ont été développés en vue de mieux considérer ces mesures dans la cartographie du bruit de trafic. \cite{makarewicz_empirical_2011} propose une estimation des niveaux sonore $L_{DEN}$ et $L_N$ directement à partir des mesures dont l'incertitude est estimé en fonction de la durée des mesures. Dans \cite{wei_dynamic_2016}, la correction dynamiques de cartes de bruit basées des mesures est proposée où les niveaux de puissance et l'effet de propagation de chaque source y sont alors modifiés par des termes correctifs obtenues en minimisant l'erreur au carré entre les niveaux prédits et les niveaux mesurés.

L'utilisation de mesure en vue de classer les ESU selon différents indicateurs physique a également été étudié comme dans \cite{rychtarikova2013soundscape} où le niveau sonore, la dureté, la fluctuation, le centre de gravité et la \textit{différence de niveau interaural} leur permettent d'obtenir 20 catégorie d'ESU à partir de 370 enregistrements 15-20 minutes réalisés dans 4 villes de Belgique. Dans \cite{can_describing_2015}, une série de mesures réalisée dans la ville de Marseille en vue de classer des ambiances sonores en fonction du niveau sonore pondéré $A$, de l'écart type de ce niveau et du centre de gravité spectrale. 

Enfin la réalisation de mesures est notamment utile pour des études perceptives de l'ESU par des citadins lors de \textit{soundwalks}. Soumis à un questionnaire sur les sons qui les entoure, leur réponses peuvent alors corrélées avec les enregistrements sonores \cite{brocolini_measurements_2013, hong2013designing}) permettant de relier leur perception avec des indicateurs physiques.\\ 

L'utilisation de mesures et d'enregistrements faits en ville permet donc une description différente des ESU par rapports au modèles prédictifs. À l'heure de l'émergence de l'\textit{IoT} (\textit{Internet of Things} ou l'internet des objets en anglais) \cite{zanella2014internet} et de la ville intelligente (\textit{Smart City} en anglais) \cite{chourabi2012understanding}, de nombreuses villes s'équipent actuellement en réseaux de différents types de capteurs disséminés dans le milieu urbain afin d'en contrôler, en temps réel, de nombreux aspects : distribution d'énergie, gestion des transports, de l'eau ou des déchets. L'objectif étant alors d'optimiser le fonctionnement de la ville afin d'améliorer la qualité de vie des citadins. L'arrivée sur le marché de capteurs acoustiques à bas cout (microphone MEMS)  \cite{van2010use} rend alors possible l'intégration de capteurs acoustiques dans ces réseaux pour ainsi étudier les ESU. La mise en place de tels réseaux en  villes a fait lieux de plusieurs projets où différentes approches sont étudiées : nombre de capteurs installé, surface couverte par ce réseau, mesures fixes ou mobiles \dots

\subsection{Déploiement de réseaux de capteurs fixes}

Un des limites des campagnes de mesures mené en ville était leur difficulté à être déployé sur l'ensemble d'une ville en raison de leur coût. Grâce à la diminution des coûts des microphones (3 €), leur déploiement en ville est plus facilement réalisable. La réalisation de réseaux de capteurs fixes, en réalisant des mesures constamment, donne la possibilité d'avoir accès à des variations à long terme des niveaux sonores à l'emplacement des microphones. 

\begin{figure}[t]
\centering
\includegraphics[width=0.8\linewidth]{./figures/cartographie/reseau_mesure.png}
\caption[Schéma d'un réseau de capteurs fixes]{Schéma d'un réseau de capteurs fixes\protect\footnotemark}
\label{fig:reseau_capteur}
\end{figure}

\footnotetext{\url{http://cense.ifsttar.fr/}}

Un réseau de capteurs développé depuis plusieurs année est celui de la ville de Paris, géré par BruitParif, à travers le projet RUMEUR \cite{mietlicki2012innovative}, en région parisienne, qui existe déjà depuis plusieurs années où des réseaux de microphones sont déployés afin d'évaluer l'ESU en région parisienne. Le réseau est comprend des stations fixes pour évaluer les évolutions à long terme mais intègre aussi des campagnes de mesures plus courtes (de plusieurs heures à plusieurs années) pour voir l'impact acoustique d'un évènement (journée sans voiture) ou d'un aménagement (modifications de voirie). Un site internet\footnote{\url{http://rumeur.bruitparif.fr/}} est mis à disposition pour avoir un aperçu complet des mesures réalisées sur les nombreux emplacements.

\`A l'heure actuelle, plusieurs projets étudient la mise en place de réseaux plus important comme le projet européen DYNAMAP \footnote{\url{http://www.life-dynamap.eu/}} \cite{dynamap_2016}. Ce projet a pour objectif de développer un système de cartographie de bruit dynamique basé sur des réseaux de capteurs à bas coûts installés en ville. Une application de ce projet a déjà été réalisée dans deux villes tests, Milan et Rome \cite{bellucci_life_2017}.
Le principe de leur approche est d'ajuster les cartes de bruits simulées à partir des différences obtenues entre les niveaux sonores mesurés aux stations et les niveaux sonores calculés à ce même point par les modèles prédictifs. Pour limiter le coût d'un tel déploiement, le nombre de microphones est réduit en les installant à des emplacements spécifiques représentatifs des différents scénarios possibles de trafic routier (homogénéité du trafic, type de revêtement, type de trafic\dots) \cite{zambon2017life}.
Le projet SONYC \footnote{\url{https://wp.nyu.edu/sonyc/}} à New-York dédie son réseau de capteurs à la surveillance de la pollution sonore et au développement d'outils de traitement du signal afin de décrire l'ESU par l'étude des sources présentes \cite{mydlarz2017noise}. 
Enfin, le projet CENSE\footnote{\url{http://cense.ifsttar.fr/}} vise à développer un réseau de capteurs dans la ville test de Lorient afin là encore d'améliorer la cartographie du bruit de trafic en agrégeant les données simulées des niveaux sonores du trafic avec les mesures réalisées en ville par ce réseau. L'approche est différente de DYNAMAP, puisqu'ici l'étude se restreint à l'échelle de plusieurs quartiers de la ville afin d'avoir un réseau de capteurs dense. La mise à jour des cartes est faite à l'aide de techniques d'assimilation de données en vue de compléter les cartes de bruits prédites avec les mesures.
Ces méthodes d'assimiliation sont notamment utilisés dans le domaine des sciences géophysiques et consistent à modifier une estimation émises par un modèle prédictif à partir de données mesurées \cite{wu2008comparison}.
Le projet s'intéresse également à la perception des citadins des ESU aux travers de questionnaires et des mesures réalisées par ce réseau de capteurs.

L'apport de ces mesures faites en villes est donc significatif (cartographie dynamique du bruit de trafic, études perceptives, détection d'évènements anormaux ou dont le niveau sonore est supérieur à ceux défini par des réglementations). Toutefois, l'installation de tels réseaux de capteurs nécessite de gérer de nombreuses problématiques techniques comme la disposition des microphones, leur maintenance, la transmission et le stockage des mesures, leur alimentation électrique\dots{} Une des première limite technique sont les perforamnce des capteurs. En effet, la réductin de leur coût s'est faite en diminuant les performances  individuelles de chaque capteur (dynamique énergétique et fréquentiel) \cite{aumond2017study}. Il est donc nécessaire de caractériser chacun des microphones en vue de connaitre leurs performances et leur limites.
Mais une des points cruciaux est celui de la position des microphones des mesures (quelle hauteur ? quelle position par rapport à des sources sonores qui seraient digne d'intérêts ?) et de la surface couverte par ces mesures. Un réseau distribué selon un maillage dense permettra une bonne représentation de l'espace mais coutera cher à installer et à maintenir alors qu'une faible densité de capteurs sera moins onéreuse mais apportera moins d'information et nécessitera des interpolations entre les mesures, sources d'incertitudes.
Toutefois, la réalisation de mesures acoustique en ville n'est pas nécessairement obligée d'être réalisée via des réseaux de capteurs fixes. D'autres pistes ont également été explorées.

\subsection{Mesures mobiles}
En parallèle des réseaux fixes, la mesure mobile est une voie envisagée. Elle consiste à réaliser des mesures acoustiques en plaçant le microphone sur un support mobile (piéton, cycliste, voiture, bus). L'avantage de cette méthode par rapports aux capteurs fixes est sa capacité à pouvoir couvrir plus facilement une plus grande surface urbaine à moindre coût. Les mesures mobiles sous-entendent deux manières d'être réalisées : soit le microphone réalise sa mesure sur un support mobile qui se déplace en même temps \cite{alsina-pages_design_2016}. Dans ce cas, un traitement du signal doit être effectué pour prendre en compte le bruit émis par ce support, soit le support permet de déplacer le microphone pour faire ensuite des mesures fixes \cite{manvell2004sadmam} ce qui simplifie la tâche mais qui nécessite plus de temps pour couvrir une surface similaire par rapport aux mesures faites sur un support mobile.
L'inconvénient de ces méthodes est qu'elle ne permettent pas la réalisation de mesures à long terme et donc de ne pas pouvoir estimer l'évolution temporelle des niveaux sonores en un point donné au cours du temps.
En conséquence, plusieurs travaux se sont intéressés à l'agrégation des mesures mobiles à des mesures réalisées par des stations fixes.
\cite{morillas2014uncertainty} s'intéresse aux incertitudes sur l'estimation des niveaux sonores estimés suivant le nombre de points ou de jours de mesures. Dans \cite{can_measurement_2014}, la prise en compte de mesures mobiles pour compléter des stations fixes est comparé à des méthodes d'interpolation (méthode de Kriging , pondération inverse de la distance). Il en résulte que l'apport des mesures mobiles diminue l'erreur produite par rapport aux méthodes d'interpolation en cela qu'elles permettent d'apporter plus d'informations quant aux variations spatiales du niveau sonore (rues calmes peu fréquentées, rues très passantes, aux abords d'intersections\dots) ce que ne permet pas une méthode d'interpolation numérique \cite{aumond2018kriging}.

\begin{figure}[t]
\begin{center}
    \begin{minipage}[t]{0.3\textwidth}
        \centering
        \includegraphics[width=0.9\textwidth]{./figures/autres/noiseCapture1.png}
    \end{minipage}
    \begin{minipage}[t]{0.3\textwidth}
        \centering
        \includegraphics[width=0.9\textwidth]{./figures/autres/noiseCapture3.png}
    \end{minipage}
    \begin{minipage}[t]{0.3\textwidth}
        \centering
        \includegraphics[width=0.9\textwidth]{./figures/autres/noiseCapture2.png}
    \end{minipage}
    \caption{Captures d'écran de l'application \textit{NoiseCapture}}
\end{center}
\end{figure}

\subsection{Mesures participatives}

Enfin, la participation des citadins peut être sollicitée aux travers de mesures participatives. Celles-ci peuvent se réaliser en équipant les personnes de dispositifs spécifiques \cite{aumond2017study} ou bien à partir d'applications développées pour smartphones. Profitant de la démocratisation de ces appareils et de l'augmentation de leurs performances, ces applications leur permettent d'avoir un dispositif suffisamment performant pour mesurer les niveaux sonores. Cette approche permet surtout d'obtenir un plus grand nombre de mesures qui ont le plus souvent une distribution spatiale et temporelle plus aléatoire mais qui sont aussi effectuées moins régulièrement. L'utilisation de ces mesures est toutefois encore sujet à caution puisque de nombreux problèmes sont encore à résoudre comme la calibration et la prise en compte des performances des microphones dans les faibles et forts niveaux sonores ou bien qualité de la réalisation de la mesure faite par l'utilisateur\dots{} Dans ce cas, le traitement statistique des résultats est primordial afin de détecter les mesures incongrues pour ne pas les considérer \cite{guillaume2016noise}. Plusieurs applications ont été dévelopées comme \textit{NoiseSpy} \cite{kanjo_noisespy_2010} ou \textit{Ambicity} \cite{ventura2017estimation}. On peut également relever le projet \textit{Noise Planet}\footnote{\url{http://noise-planet.org}} qui a pour objectif de proposer un outil libre et gratuit pour évaluer le bruit de l'environnement sonore. Il comprend une application pour smartphone, \textit{NoiseCapture} \cite{guillaume2016noise}, qui permet, là aussi, à l'utilisateur d'évaluer les niveaux sonores l'entourant tout en ayant la possibilité de décrire, à l'aide de mots-clés prédéfinis, les sons présents et l'ambiance sonore de la scène. La géo-localisation et les mesures sont ensuites collectées puis traitées pour produire des cartes de bruits, publiées en ligne (voir Figure \ref{fig:carte_noiseModelling}). En plus de collecter plus de données, ces applications permettent également de sensibiliser le citadin à son environnement sonore.\\

\begin{figure}[t]
\centering
\includegraphics[width=0.7\linewidth]{./figures/cartographie/noise_modelling.PNG}
\caption{Carte de l'ESU de l'île de Nantes mesurée par l'application \textit{NoiseCapture}  (relevée le 22/03/2018)}
\label{fig:carte_noiseModelling}
\end{figure}

\section{Intérêts et limites des mesures faites en villes}

L'ensemble de ces dispositifs (mesures pas des réseaux de capteurs fixes, mesures mobiles et participative) offrent ainsi de nombreuses possibilités, similaires à celle faites et envisagé au travers l'utilisation des modèles prédictifs : 
\begin{itemize}
\item estimation les niveaux sonores du trafic, 
\item évaluation et classification des ESU au travers d'indicateurs physique
\item représentation des ESU selon la perception des citadins.
\end{itemize}
L'intérêt est que ces applications se basent alors directement sur mesures \textit{in situ}, toutes sources confondues. Les limites des modèles liées à la simplification des sources sonores ou l'approximation de l'environnement lors de la discrétisation du milieu dans les outils numériques disparaissent alors. De plus, la réalisation de mesures donne plus facilement accès aux variations temporelles du trafic. 
Ces approchent ne sont toutefois pas exempt de limite. Dans le cas des réseaux de capteurs, leur installation et leur entretien a un coût important et sont des dispositifs lourd à gérer. De plus, la question de l'interpolation entre les points de mesures reste une source d'approximation. L'apport de mesures mobiles, si elles permettent à l'inverse de mieux estimer les variations spatiales n'offrent pas d'indications sur les variations à long-terme et peuvent être très couteuse en temps à réaliser à l'échelle d'une ville. Enfin les mesures participatives par les particuliers présentent donc de nombreuses incertitudes qui nécessitent un traitement du signal important. 

Toutefois, sur l'ensemble des dispositif et des projet, un aspect est, à l'heure actuelle, peu étudié : comme tout signal mesuré, il est nécessaire de disposer d'outils de traitements  adaptés afin d'y extraire les informations utiles et recherché. L'ESU est un milieu complexe, composé d'une multitude de sources variées (trafic routier, voix, oiseaux, klaxon, bruit de pas\dots) dont leurs allures temporelles diffèrent (parfois brèves pour le retentissement d'un klaxon ou longues pour le passage d'une voiture) ainsi que des allures spectrales variées (dans les basses fréquences pour le trafic, dans les hautes fréquences pour le sifflement des oiseaux), voir Figure \ref{fig:sourceUrbain}. L'ensemble de ces sources est aussi susceptible d'être généré simultanément. La création d'outils adaptés à cet environnement pour reconnaitre ou détecter des sources spécifiques n'est donc pas triviale. Par exemple, dans le cas du trafic routier, s'il existe des endroits où celui-ci est prépondérant sur les autres sources sonores (périphérique, grand boulevard), il peut l'être moins dans d'autres lieux (dans des rues calmes, au niveau de parc) où ce sont d'autres sources sonores qui sont majoritairement présentes (voix, oiseaux \dots). 
\cite{Mioduszewski} dans sa comparaison des niveaux sonores prédits et mesurés relève que les niveaux sonores mesurés sont supérieurs à ceux estimés par les modèles prédictifs de bruit de trafic routier. Une part de cette surestimation provient de la prise en compte des bruits qui ne sont pas relatifs au trafic  mais aussi des modèles prédictifs qui ne permettent pas de considérer l'ensemble des variations sonores des véhicules. Ses travaux révèlent donc la nécessité de générer un outil adapté à l'étude des mesures et enregistrements faits en villes pour y extraire les contributions et les niveaux sonores des sources présentes en villes. Considérer l'ensemble des mesures et des enregistrements réalisés en ville sans distinction particulière entre les sources peut donc mener à de mauvaises estimations du temps de présence ou de son niveau sonore et donc à de mauvaises interprétations. Cet aspect est donc une limite à l'utilisation de ces méthodes qu'il est nécessaire de résoudre afin de permettre la démocratisation de la mesure acoustique en ville.\\

\begin{figure}[t]
\centering
\subfigure[\label{fig:sourceUrb1}]{\includegraphics[width=0.4\linewidth]{./figures/autres/sourcesUrbainesCar.pdf}}
\subfigure[\label{fig:sourceUrb2}]{\includegraphics [width=0.4\linewidth]{./figures/autres/sourcesUrbainesBird.pdf}}
\subfigure[\label{fig:sourceUrb3}]{\includegraphics [width=0.4\linewidth]{./figures/autres/sourcesUrbainesCarHorn.pdf}}
\subfigure[\label{fig:sourceUrb4}]{\includegraphics [width=0.4\linewidth]{./figures/autres/sourcesUrbainesFootStep.pdf}}
\caption{Spectrogrammes d'un passage d'une voiture \subref{fig:sourceUrb1}, d'un sifflement d'oiseaux \subref{fig:sourceUrb2}, d'un klaxon \subref{fig:sourceUrb3} et d'un bruit de pas \subref{fig:sourceUrb4}.}
\label{fig:sourceUrbain}
\end{figure}

En conséquence, les travaux de cette thèse cherchent à répondre à ces questions :
\begin{itemize}
\item \textbf{Comment déterminer le niveau sonore du trafic routier en ville ? est-il possible de déterminer ceux d'autres sources ?}
\item \textbf{Quelles sont les méthodes disponibles pour réaliser cette tâche ? Quel est l'outil le plus adapté à cet environnement parmi ces méthodes ?}
\item \textbf{Quel protocole expérimental mettre en oeuvre pour tester et valider les performances des implémentations alternatives de cet outil ?}
\end{itemize}


\section{Estimation du niveau sonore du trafic routier à partir de mesures} \label{part:cachier_charges}

\subsection{Objectifs}
Étant la source principale de bruit en ville ainsi que la plus gênante, le trafic routier sera la source d'intérêt qui sera principalement étudiée dans ce document. Le principe général est résumé en Figure \ref{fig:estimateur0} : à partir d'un enregistrement audio (en format wav et de fréquence d'échantillonnage 44,1 kHz), un outil, appelé \textit{estimateur}, détermine le niveau sonore du trafic routier $\tilde{L}_{eq,trafic}$.

\begin{figure}[t]
\centering
\includegraphics[width=0.7\linewidth]{./figures/NMF/bloc_diagram_estimateur0.pdf}
\caption{Vue synthétique de l'approche proposée.}
\label{fig:estimateur0}
\end{figure}

L'objectif est donc de construire cet estimateur et un protocole expérimental adéquat afin d'obtenir la meilleure estimation possible du niveau sonore du trafic.

\subsection{Protocole expérimental}

Si plusieurs études se sont déjà intéressées au trafic routier au sein d'environnements sonores urbains (reconnaissance du bruit de véhicule \cite{defreville_automatic_2006}, estimation du débit véhicule \cite{torija2012using}, détection des accidents \cite{harlow2001automated}, estimation des trajectoires \cite{leiba2017large}), la détermination de son niveau sonore a pour l'instant été très peu étudiée par le passé. On peut citer les travaux réalisés récemment au sein du projet DYNAMAP \cite{socoro2017anomalous}. Leur approche consiste à entrainer une méthode de détection en vue d'estimer les trames temporelles où la classe de son \textit{trafic} n'est pas présente afin de les rejeter lors de l'estimation des niveaux sonores. Ici, l'approche choisie est différente : l'estimateur du niveau sonore du trafic routier s'appuie sur une méthode de séparation de sources afin d'extraire l'intégralité de la composante \textit{trafic} des enregistrements audio parmi les autres sources sonores présentes (voir Figure \ref{fig:separation_source}). De cette extraction, le niveau sonore trafic $\tilde{L}_{eq,trafic}$ est calculé.

\ml{disctuer un peu plus les avantages et defaut des deux approches, sous section a part entiere ?}

\begin{figure}[t]
\centering
\includegraphics[width=0.7\linewidth]{./figures/NMF/bloc_diagram_source_separation.pdf}
\caption{Diagramme en bloc de l'approche par séparation de sources.}
\label{fig:separation_source}
\end{figure}

Une question reste toutefois à résoudre : comment, à partir d'enregistrements audio, être certain que l'estimateur donne une valeur correcte du niveau sonore du trafic ? Dans le cas où il n'y a que du trafic, l'estimation fournie par la méthode peut être facilement comparée mais quid des scènes sonores où le trafic n'est pas prépondérant et est recouvert par d'autres sources sonores ? La valeur exacte du trafic, $L_{eq,trafic}$, est l'inconnue qu'on cherche justement à déterminer. Sans cette valeur de référence, il est impossible de comparer la valeur déterminée et ainsi la validité et les performances de l'estimateur.
Le choix est donc fait ici d'utiliser des corpus de scènes sonores issus d'un processus de simulation où un contrôle complet des classes sonores présentes, ainsi que de leur niveau sonore, est alors possible. La valeur exacte, $L_{eq,trafic}$, est alors obtenue. La figure \ref{fig:diagramBlocProtocol} résume le schéma global du procédé suivi.

\begin{figure}[t]
\centering
\includegraphics[width=0.7\linewidth]{./figures/NMF/Bloc_diagram_estimateur_FR.pdf}
\caption{Diagramme bloc du protocole expérimental.}
\label{fig:diagramBlocProtocol}
\end{figure}

Se pose alors la question du niveau sonore calculé par scène : choisit-on un niveau sonore à court-terme toutes les 125 ms ? toutes les secondes ? toutes les minutes ? pondéré $A$ ? Aux vues de l'utilisation de cet estimateur qui est faite (meilleur estimation de la contribution du trafic, amélioration de la cartographie des cartes de bruits en ville), le choix est fait de choisir une durée adaptée à cet outil et ainsi de calculer le niveau sonore du trafic chaque minute, $\L_{eq,trafic,60s}$. Enfin, on ne  considère aucune pondération afin de se focaliser sur une estimation physique du niveau \textit{trafic}.\\

Les niveaux sonores exacts et estimés, produits sur un ensemble de $M$ scènes testés, sont ensuite comparés à travers un calcul de métrique. Parmi les différentes métriques possibles (somme des carrés des résidus, la racine de l'erreur quadratique moyenne $RMSE$ \dots), l'erreur absolue moyenne, $MAE$ (pour \textit{Mean Absolute Error}) est retenue :

\begin{equation}
MAE = \frac{\sum_{i = 1}^{M} \vert L_{eq, trafic, 60s}^i - \tilde{L}_{eq, trafic, 60s}^i \vert}{M}.
\end{equation}

Contrairement à l'erreur RMSE, qui revient à la racine carré de la moyenne du carré des différences entre les données observées et réelles, qui pénalise plus les valeurs qui dévie fortement, l'erreur $MAE$ présente l'intérêt de considérer un poids identique entre chaque différence et ainsi de gagner en interprétabilité.\\

Les questions à résoudre sont alors :
\begin{itemize}
\item \textbf{Quelle méthode de séparation de sources choisir comme estimateur ?}
\item \textbf{Comment construire des corpus de scènes sonores urbaines pour tester l'approche proposée ?}
\end{itemize}

%
%%\bibliographystyle{unsrt}
%%\bibliography{../bibliographie}
%%
%%\end{document}

%
%Dans le domaine fréquentiel, le produit de convolution s'exprime sous la forme : 
%
%\begin{equation}
%\hat{M}_{i}(f) = \hat{s}_j(f)\hat{\delta}_{ij}(f)e^{i2\pi f \tau_{ij}} 
%\end{equation}
%
%avec $\hat{g}(f)$, la transformée de Fourier de la fonction $g(t)$, $\hat{g}(f) = \frac{1}{\sqrt{2\pi}}\int_{-\infty}^{+\infty}g(t)e^{-i2\pi ft} dt$. $\hat{\delta}_{ij}(f)$ s'apparente alors à un filtre de propagation.