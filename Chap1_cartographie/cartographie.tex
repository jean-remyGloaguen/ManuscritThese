%%\documentclass[twoside,openright,a4paper,11pt]{book}
%%
%%
\usepackage[utf8]{inputenc}
\usepackage[francais]{babel}
\usepackage[T1]{fontenc}

\addto\captionsfrench{\def\tablename{\textsc{Tableau}}}% pour avoir TABLEAU et pas TABLE dans les légendes des tableaux
%%%%%%%% MISE EN PAGES %%%%%%
\usepackage{geometry}
\geometry{outer=2cm,inner=3cm,top=3cm}

\setcounter{tocdepth}{3}     % Dans la table des matieres
\setcounter{secnumdepth}{3}  % Avec un numero.
\usepackage{setspace}

\usepackage{fancyhdr}	% marge en haut et en bas
\pagestyle{fancy}

\fancyhead{}	% vide l'entête
\fancyfoot{} % vide le pied~de~page

\fancyhead[RO]{\leftmark}
\fancyhead[LE]{\rightmark}
\fancyfoot[C]{\thepage}	% numéro de page en bas au centre

\renewcommand{\headrulewidth}{0.4pt} % épaisseur du trait en haut
\renewcommand{\footrulewidth}{0.4pt} % épaisseur du trait en bas

\fancypagestyle{mypagestyle}{%
    \fancyhead{}	
    \fancyfoot{} 
    \fancyfoot[C]{\thepage}
    \renewcommand{\headrulewidth}{0.4pt} 
	\renewcommand{\footrulewidth}{0.4pt} 
}

\fancypagestyle{couvertureAbstract}{%
    \fancyhead{}	
    \fancyfoot{} 
    \fancyfoot[C]{}
	\renewcommand{\headrulewidth}{0pt} 
	\renewcommand{\footrulewidth}{0pt} 
}
%
\usepackage{layout}
\usepackage{tocbibind} % include tableofcontent in itself

%%%%%% PAGE DE GARDE %%%%%%

\geometry{outer=2cm,inner=3cm,top=3cm}
\usepackage[scaled]{helvet} % font used on cover (Helvetica)
\usepackage{eso-pic} % to set background picture
\usepackage{multicol} % for back cover (abstracts)
\usepackage{graphicx} % to include logos
\usepackage{tikz} % to compose background picture

% Colors (extracted from SPI's template)
\definecolor{boxcolor1}{rgb}{0.91373,0.92941,0.87451}
\definecolor{boxcolor2}{rgb}{0.94902,0.93333,0.91373}
\definecolor{boxcolor3}{rgb}{0.76078,0.87843,0.17647}
\definecolor{headercolor}{rgb}{0.94118,0.30980,0.17255}
\definecolor{namecolor}{rgb}{1.0,0.4,0.0}
\definecolor{titlecolor}{rgb}{0.19216,0.51765,0.60784}
% Also used: gray, teal (predefined by xcolor package, usually loaded by document class)

% Cover environment, to keep changes local
\newenvironment{cover}{%
  \fontfamily{phv}\selectfont % Select Helvetica font
  \pagestyle{empty} % No page number
}{
  \addtocounter{page}{-1}
  \cleardoublepage
}

% Macro for background common to front and back
\newcommand{\tikzBG}{%
  \path (0,0) rectangle (1,1);
  %TODO: You should adjust the bottom height of the following rectangle to fit your abstract's length
  \path [fill=boxcolor1] (.0571,.11) rectangle (.481,.963); 
  \path [fill=boxcolor2] (.4333,.697) rectangle (.9048,.7475);
  \path [fill=boxcolor2] (.4333,.7811) rectangle (.9048,.8316);
  \path [fill=boxcolor2] (.4333,.8687) rectangle (.9048,.9192);
  \path [fill=boxcolor3] (.0571,.7879) rectangle (.5762,.8316);
  \node[inner sep=0pt] at (0.2285,0.8788) [above left] {%
    \includegraphics[height=.0707\paperheight,keepaspectratio]{./figures/logo/logo_unb.png}};
  \node[inner sep=0pt] at (0.6667,0.8788) [above right] {%
    \includegraphics[height=.0808\paperheight,keepaspectratio]{./figures/logo/logo_ecn_color.png}};
  \node at (.0571,.8316) [above right,color=headercolor] {%
    \fontsize{29}{35}\selectfont\bfseries Th\`ese de Doctorat};
}

% Macro for repeated information (to avoid insconsistency)
%TODO: fill in with no formatting but desired case
\newcommand{\firstName}{Jean-Rémy}
\newcommand{\surname}{Gloaguen}
\newcommand{\thesisTitle}{Estimation du niveau sonore de sources d'intérêts au sein de mixtures sonores urbaines : application au trafic routier}

%%%%%%% SYMBOLES %%%%%
\usepackage{tipa}	% pour avoir l'accent concave
\usepackage{lmodern}	% pour les guillemets
\usepackage{gensymb}	% pour les degrés
\usepackage{enumitem}	% pour changer le symbole de l'item (\begin{itemize}[label=$\bullet$])

%%%%%%% EQUATION %%%%%%
\usepackage{amssymb}
\usepackage{amsmath}
\usepackage{fancybox}
\usepackage{xfrac}	% fraction de type "1/4"
\usepackage{cases}	% système équation
\usepackage[overload]{empheq}
\usepackage{bm}		% pour mettre en gras .
\usepackage{units} 	% x/y barre latérale pour les fractions
%
%%%%%%% FIGURE %%%%%%
\usepackage{subfigure}	% utiliser subfigure
\usepackage{float}	% utiliser H dans les figures
%
%%%%%% TABLEAUX %%%%%%
\usepackage{array,multirow,makecell}
%\addto\captionsfrench{\def\tablename{\textsc{Tableau}}}% pour avoir TABLEAU et pas TABLE dans les légendes des tableaux
\usepackage{colortbl} % pour avoir des lignes colorées dans les tableau
%\usepackage{slashbox} % pour les \backslashbox
%\usepackage{subcaption}
\usepackage{hhline}	% pour les lignes horizontales 
\usepackage{tabularx} % permet itemize dans les cellules
\usepackage{booktabs}
\usepackage{longtable}	% pour les tableaux longs

\newcolumntype{L}[1]{>{\raggedright\let\newline\\\arraybackslash\hspace{0pt}}m{#1}}
\newcolumntype{C}[1]{>{\centering\let\newline\\\arraybackslash\hspace{0pt}}m{#1}}
\newcolumntype{R}[1]{>{\raggedleft\let\newline\\\arraybackslash\hspace{0pt}}m{#1}}

%%%%% ALGORITHME %%%%%
\usepackage{algorithm}
\usepackage{algorithmic}

%%%%% BIBLIO %%%%%
\usepackage[fixlanguage]{babelbib}
\selectbiblanguage{french}
\usepackage{breakcites}	% pour couper les références en bout de ligne

%%%%% APPENDICES %%%%%%%
\usepackage[toc,page]{appendix}

%%%%%%%%%%%%%%%%%%%%%
\usepackage{url}	% gérer les adresses www.
\linespread{1.2}	% interligne

\newcommand{\ml}[1]{\textcolor{red}{ML : #1}}

\cleardoublepage
%%
%%\begin{document}

\chapter{Connaitre l'environnement sonore urbain : de la prédiction à la mesure}\label{chap:modele}
\thispagestyle{empty}

Dans ce chapitre, une présentation des méthodes utilisées pour caractériser l'environnement sonore urbain est réalisée. Dans une première partie, le problème général est posé formellement, puis l'utilisation de modèles prédictifs et les éléments de cartographie de bruit en ville sont exposés et enfin la réalisation de mesures en milieu urbain est présentée. Enfin, la problématique générale est énoncée et une solution est proposée.

\section{Définition formelle du problème}

Soit $M_{i}(t)$, un environnement sonore urbain (abrégé ESU) défini dans un espace $\Omega$, capté en un point donné $i_{\in \Omega}$, reçu à un instant $t$. L'ESU se décompose alors comme la somme de $N$ différentes contributions sonores $S_j(t)$ reçues à ce point $i$. Chacune de ces contributions, est le résultat de l'émission sonore d'une source acoustique $s_j$, de puissance sonore $L_{w,j}$, située à la position $j_{\in \Omega}$ et émise à un instant $t-\tau_j$,  qui s'est propagée dans l'environnement urbain le long de $k$ chemins. L'ESU s'exprime alors, mathématiquement, dans le domaine temporel comme :

\begin{subequations}\label{eq:esu_formel}
\begin{align}
M_i(t) &= \sum_{j = 1}^{N}S_j(t), \\
 & = \sum_{j = 1}^{N} s_j(t-\tau_j) \ast \delta_{ij}(t), \label{eq:convolution_ESU}\\
 & = \sum_{j = 1}^{N} \sum_{k = 1}^{+\infty} s_j(t-\tau_{ijk}) \delta_{ijk}.\label{eq:propagation}
\end{align}
\end{subequations}

Dans l'équation \ref{eq:convolution_ESU}, le produit de convolution de la source $s_j(t-\tau_j)$ par la variable $\delta_{ij}(t)$ traduit l'intégralité des effets de propagation générés par la diffusion du son dans l'environnement $\Omega$ jusqu'au récepteur $M_i(t)$. Ils incluent les phénomènes d'atténuation géométrique de l'onde sonore ainsi que ceux de diffusion et d'absorption provoqués par les réflexions de l'onde sonore sur les parois des bâtiments et sur le sol.
En conséquence, l'équation \ref{eq:propagation} décompose, pour chaque source, l'impact de chaque chemin de propagation de l'onde sonore. Ces chemins incluent le champ direct, le champ réfléchi par une réflexion, deux réflexions \dots{} Pour chaque champ, comme la distance de propagation est différente, le temps de propagation entre la source et le récepteur varie créant un déphasage temporel $\tau_{ijk}$ et une atténuation $\delta_{ijk}$ spécifique.
La mixture $M_{i}(t)$ peut alors soit être captée par un microphone installé en ville (comme dans l'exemple en Figure \ref{fig:schema_ville}), ce qui permet notamment d'obtenir des indicateurs physiques (niveau sonore en dB SPL ou pondéré A), soit être perçue par les citadins où les aspects perceptifs entrent alors en jeux. Dans ce cas, la mixture $M_{i}(t)$ est évaluée à travers des indicateurs perceptifs comme l'\textit{agrément sonore} qui dépend notamment de la prédominance de certaines sources sonores et du cadre environnemental dans lequel elles sont perçues. L'ensemble de ces variables est représenté dans la Figure \ref{fig:schema_ville}.

\begin{figure}[hbtp]
\centering
\includegraphics[width=.9\linewidth]{./figures/autres/schema_ville_propa.pdf}
\caption{Schéma du problème considéré en ESU pour un signal capté par un microphone au point $\mathbf{x}$. Deux sources sonores émises à l'instant $t-\tau_{j}$ à la position $x_j$ sont présentes : une voiture, $s_{1}$ (résumée en une source ponctuelle symbolisée par un cercle rouge), et un piéton, $s_2$, (résumée en une source ponctuelle symbolisée par un cercle bleu). Chaque source se propage jusqu'au récepteur selon 2 chemins de propagation (champ direct $\delta_{ij1}$ et champ réfléchi $\delta_{ij2}$).}
\label{fig:schema_ville}
\end{figure}

Les sources $s_j(t)$ expriment les différentes sources sonores présentes en ville comme le trafic aérien ou ferroviaire mais aussi les voix, les bruits de pas, celui d'une valise à roulettes, les sifflements d'oiseaux, les aboiement de chiens \dots{} La source sonore principale en ville est celle du trafic routier, qui est considéré comme la somme des contributions des $M$ véhicules présents dans l'environnement du récepteur et dont la somme globale s'exprime :

\begin{subequations}
\begin{align}
S_{tr}(t) &= \sum_{j = 1}^M S_{v_j}(t),\\
 & = \sum_{j = 1}^M s_{v_j}(t-\tau_j) \ast \delta_{j}(t),
\end{align}
\end{subequations}

où $s_{v_j}(t)$ correspond à l'émission sonore globale du véhicule $j$. La variable $s_{v_j}(t)$ a plusieurs origines : bruit du moteur thermique, aérodynamique et de roulement. Ici, puisque les citadins ne réalisent pas cette séparation mais considèrent l'ensemble de ces sources comme un tout, on résume la source \textit{voiture} $s_{v_j}(t)$ comme l'ensemble de ses bruits. Précisons que le son émis par le klaxon du véhicule n'est pas considéré dans la source \textit{voiture} car il appartient à la catégorie des avertisseurs sonores.
L'ESU peut ainsi s'exprimer comme

\begin{equation}
M_i(t) = S_{tr}(t)+\sum_{j = 1}^{N-M}S_j(t),
\end{equation}
\\

qui est un ensemble de sources variées, ayant une allure temporelle, fréquentielle ainsi qu'un niveau sonore propre, et qui sont émises dans un environnement où des phénomènes de propagation complexes ont lieu. Les questions soulevées sont alors :

\begin{itemize}
\item \textbf{Comment caractériser les ESU ? Quels sont les moyens disponibles pour cela ?}
\item \textbf{Comment estimer la présence et le niveau sonore du trafic routier ? Peut-on déterminer la contribution des autres sources sonores ?}\\
\end{itemize}

%Avant de répondre à ces questions, il est nécessaire, dans un premier temps, d'en présenter les motivations.

%\section{De l'intérêt d'étudier les environnements sonores urbains}
%
%Au sein de l'Union Européenne, 70 $\%$ de la population, soit quasiment 340 millions d'habitants, vivent dans des zones urbaines \cite{europ-commission_data_2017}. 486 villes concentrent, chacune, plus de 100 000 habitants. En France, selon l'INSEE, c'est même plus de 84 $\%$ de la population qui vit dans une zone urbaine, soit plus de 55 millions d'habitants. Cette concentration soulève de grandes questions autour de l'organisation de l'espace urbain afin d'offrir une qualité de vie acceptable aux citadins. En effet, avec de telles densités (environ 3000 habitants/km$^2$ et jusqu'à plus de 21 000  habitants/km$^2$ pour la ville de Paris, la plus dense de l'Union Européenne (UE)), plusieurs formes de pollutions viennent dégrader l'environnement urbain. Des sources de désagrément perçues par le citadin, le bruit est le phénomène qui provoque le plus de gêne après la pollution de l'air. Ce bruit est le fruit des activités humaines, provenant essentiellement du transport qu'il soit routier, ferroviaire ou aérien \cite{zannin_characterization_2013}.\\
%
%%\ml{je suis pas trop pour la mise en capitale pour introduire les acronymes}
%
%Selon un rapport de l'Organisation mondiale de la Santé (OMS) \cite{who_burden_2017}, en Europe, près de 200 millions de personnes sont exposées quotidiennement à des niveaux sonores équivalent supérieurs à 55 dB($A$), soit 40$\%$ de la population. Près de 20 $\%$ atteignent même plus de 65 dB($A$) en journée et plus de 30 $\%$ sont touchées par un niveau sonore excédant 55 dB($A$) la nuit. En France, selon un rapport de l'ADEME \cite{europeens2016analyse}, ce sont 52 millions de personnes qui se disent affectées par le bruit et principalement le bruit issu du trafic routier. Plus de 7 millions d'individus sont exposés à des niveaux supérieurs à 65 dB($A$) au quotidien et à plus de 55 dB($A$) la nuit.
%Cette exposition quotidienne, à de tels niveaux, n'est pas sans conséquence pour l'être humain. L'impact sur l'organisme humain dû à l'exposition du bruit est observé et étudié depuis de nombreuses années \cite{ising1980health}. Parmi les effets possibles, les plus couramment relevés sont des troubles du sommeil \cite{pirrera2010nocturnal}, de la vigilance et de la concentration, l'augmentation du stress, de la pression artérielle et du rythme cardiaque \cite{babisch2005traffic, babisch2008road}. Selon le rapport de l'OMS, ce sont près de 8 millions de personnes en Europe qui sont affectées par des troubles du sommeil mais aussi 900 000 touchées par de l'hypertension. On estime aussi que 43 000 hospitalisations sont imputables au bruit dues à des pertes de vigilance et de concentration et jusqu'à 10 000 cas de morts prématurées par an. Cet impact sur la santé a également un coût financier pour la société : en France, ce coût est estimé à plus de 11,5 milliard d'euros par an dont une grande partie (89 $\%$) est imputable au bruit du trafic routier \cite{europeens2016analyse}. De plus, si le bruit en ville impacte la vie des citadins, celui-ci se fait également ressentir auprès de la faune sauvage \cite{dutilleux_anthropogenic_2012, francis2009noise} leur causant également du stress ou en compliquant la communication entre les individus et leur reproduction.\\
%
%Une trop grande exposition au bruit a ainsi un impact négatif sur les individus et sur leur environnement. Enjeu de société dont les français ont pleinement conscience \cite{JNA2016etude}, il est nécessaire et utile de caractériser les environnements sonores urbains (ESU) afin d'estimer les sources sonores présentes, leurs niveaux sonores, leurs répartitions et ainsi réduire et limiter leurs impacts sur les populations urbaines.

\section{Utilisation de modèles prédictifs}

L'une des premières pistes pour évaluer les ESU est l'utilisation de modèles prédictifs de bruits de trafic. L'objectif est alors de générer des lois d'émissions sonores afin de déterminer les sources $s_j$ et des lois de propagation $\delta_{ij}(t)$. De nombreux modèles d'émission existent afin de prédire les niveaux sonores émis par le trafic routier \cite{quartieri2009review}, ferroviaire \cite{van2000railway} et aérien \cite{zaporozhets1998aircraft}. Étant la source sonore la plus présente et la plus gênante, ce sont les modèles dédiés au trafic routier qui sont ici présentés.

Plusieurs modèles visant à estimer la puissance acoustique, $L_w$, émise par les véhicules et la propagation des ondes sonores dans le milieu urbain ont été développés depuis plus de 30 ans. Plusieurs pays ayant alors leur propre modèle (RLS-90 en Allemagne, CNR en Italie, NMPB-Routes en France, CoRTN au Royaume-Uni, Nord 2000 pour les pays Scandinaves). On se propose tout d'abord de décrire succinctement 3 modèles selon plusieurs paramètres de leur modèle d'émission et de propagation :

\begin{itemize}
\item Le modèle \bsc{Harmonoise}/Imagine \cite{jonasson2004source}, développé afin d'offrir un premier modèle d'émission harmonisé à l'échelle des pays membres de l'UE.
\item La NMPB-routes-2008, un modèle français développé à partir des années 90 \cite{setra_prevision_2009-1, setra_prevision_2009-2}.
\item Le modèle \bsc{Cnossos-Eu}  \cite{CNOSSOS}, le modèle développé le plus récent, là aussi afin d'harmoniser le modèle d'émission sonore et de propagation à l'échelle de l'UE, inspiré par les modèles précédents.
\end{itemize}

\subsection{Modèle d'émission du trafic routier}\label{part:modele_emission}

Dans ces trois modèles, la puissance émise par une portion de route est liée à la puissance acoustique émise par un véhicule $L_{w,veh}$ et au débit des véhicules $Q$:

\begin{equation}
L_w = L_{w,veh} + f(Q).
\end{equation}

La puissance acoustique émise par le véhicule est décomposée en deux parties : une composante \og bruit de roulement \fg{}, $L_{w_r}$, et \og bruit de moteur \fg{}, $L_{w_m}$ qui est définie comme

\begin{equation}
L_{w,veh} = 10\times \log_{10} \left(10^{L_{w_r}/10}+10^{L_{w_m}/10}\right).
\end{equation}

Selon le modèle choisi, l'estimation de $L_{w,veh}$ est alors différente. Dans le Tableau \ref{tab:modèle_emission}, les niveaux de puissances $L_{w_r}$ et $L_{w_m}$ et la fonction $f(Q)$, exprimés en fonction de la vitesse $v$ du véhicule, ainsi que plusieurs paramètres et catégories pris en considération sont détaillés.

\begin{table}[ht]
\centering
\caption{Paramètre d'estimation de la puissance acoustique selon 3 modèles d'émission sonores}
\label{tab:modèle_emission}
\begin{tabular}{L{2cm}C{4cm}C{4cm}C{4cm}}
 & \bsc{Harmonoise} & NMPB & \bsc{Cnossos-Eu} \\ \toprule
\rowcolor[HTML]{C0C0C0} $L_{w,veh}$ & \begin{tabular}[l]{L{3.5cm}}$L_{w_r} = a_r+b_r\log\left(\frac{v}{70}\right)$ \\ $L_{w_m} = a_m+b_m\left(\frac{v-70}{70}\right)$\end{tabular} &  \begin{tabular}[l]{L{3.5cm}}$L_{w_r} = \alpha_r+\beta_r\log\left(\frac{v}{90} \right)$\\ $L_{w_m} = \alpha_m+\beta_m \log\left(\frac{v}{90} \right)$\end{tabular} & \begin{tabular}[l]{L{3.5cm}}$L_{w_r} = a_r+b_r\log\left(\frac{v}{70}\right)$\\ $L_{w_m} = a_m+b_m\left(\frac{v-70}{70}\right)$\end{tabular} \\
$f(Q)$ & $\log\left(\frac{Q}{1000\times v} \right)$ & $\log(Q)$ & $\log\left(\frac{Q}{1000\times v} \right)$ \\
\rowcolor[HTML]{C0C0C0}bande de fréquences & 25 Hz - 10 kHz & 100 Hz - 5 kHz & 125 Hz - 4 kHz\\
 description de la source & 2 sources équivalentes ponctuelles & 1 source équivalente ponctuelle & 1 source équivalente ponctuelle \\
\rowcolor[HTML]{C0C0C0} nombre de catégorie de véhicule & 5 & 2  & 5 \\
nombre de revêtement & 1 & 3 & 1 \\
\bottomrule
\end{tabular}
\end{table}

Les valeurs des coefficients dans les modèles \bsc{Harmonoise} et \bsc{Cnossos-Eu}, $a_r$ ,  $b_r$ ,  $a_m$  et $b_m$,  sont données selon les bandes de tiers d'octave et la catégorie du véhicule. Dans la NMPB, $\alpha_r$, $\beta_r$, $\alpha_m$ et $\beta_m$ sont données en fonction du véhicule, de sa vitesse et du revêtement du sol.
On observe que ces trois modèles, bien que similaires dans leur manière de décomposer la puissance acoustique, présentent plusieurs aspects divergents : nombre de catégories de véhicules, nombre de revêtement des sols, nombre de bandes de tiers d'octave, calcul d'un niveau en dB pour le modèle \bsc{Harmonoise} et \bsc{Cnossos-Eu} et d'un calcul en dB par mètre pour le modèle NMPB. Ces 3 modèles considèrent des vitesses stables en accélération ou en décélération au travers de corrections qui cependant diffèrent entre les modèles. Si les modèles \bsc{Harmonoise} et \bsc{Cnossos-Eu} sont similaires sur l'estimation des niveaux de puissance $L_{w_r}$ et $L_{w_m}$, le premier considère deux sources ponctuelles pour modéliser un véhicule alors que le second n'en considère qu'une. De plus, dans ces deux modèles si le même nombre de catégories de véhicule est identique, leur classification est différente puisque \bsc{Harmonoise} considère une 5\ieme{} catégorie pour les camions de type tracteur, camion de chantier alors que celle de \bsc{Cnossos-Eu} permet d'inclure les véhicules électriques.
Enfin, le temps de calcul associé à la modélisation des sources sonores sonores, sur tous les modèles, est quasi instantané. 

\subsection{Modèle de propagation}\label{part:modele_propa}
Chaque méthode possède également son modèle de propagation acoustique qui simule le rayonnement acoustique de la source. C'est cette étape qui est la plus longue en raison du volume du domaine $\Omega$ à calculer, du nombre de chemin de propagation $k$ possible et du nombre $M$ de sources sonores à considérer.
Dans le cas du modèle d'\bsc{Harmonoise}, la propagation du son est réalisée en considérant plusieurs approches (résolution de l'équation parabolique, méthode des éléments de frontières ou tir de rayons), afin de s'adapter à différentes configurations, en y considérant des conditions atmosphériques homogènes. L'influence de la route est prise en compte selon son revêtement (température, âge, humidité). Cette approche est reconnue pour être plus \og physique \fg{} mais ausi pour avoir un temps de calcul plus long que les autres modèles (jusqu'à 50 fois plus long) \cite{probst2011comparison}. Pour la NMPB, la méthode de propagation choisie est celle des tirs de rayons où les chemins directs, réfléchis et diffractés sont considérés entre la source et le récepteur. En fonction des conditions atmosphériques relevées (température, vent), les atténuations dans les conditions favorables (à la propagation) et homogènes sont considérées. Les effets de sol (3 types de routes considérés, dont les propriétés varient selon leur ancienneté et la température ambiante de l'air), la divergence géométrique et l'absorption atmosphérique sont aussi prises en compte.
Enfin, les aspects, liés à la propagation du son du modèle NMPB ont été repris dans le modèle \bsc{Cnossos-Eu}. \\
Là encore, les modèles de propagation diffèrent selon les méthodes choisies. Une comparaison plus détaillée entre plusieurs de ces modèles, et sur de nombreux autres points, peut être trouvée dans \cite{steele_critical_2001} et dans \cite{garg_critical_2014}.

\subsection{Réaliser des cartes du bruit de trafic en ville}

À partir de 2002, est instaurée la directive européenne 2002/49/CE \textit{relative à l'évaluation et à la gestion du bruit dans l'environnement} \cite{directive} dont le but est de mieux connaitre la répartition des niveaux sonores générés par le trafic routier, ferroviaire et aérien ainsi que par les Installations Classées pour la Protection de l'Environnement (ICPE) dans les agglomérations de plus de 100 000 habitants. Cette directive prévoit :

\begin{itemize}
	\item d'évaluer l'exposition au bruit des populations basée sur des méthodes communes aux pays européens,
	\item d'informer les populations sur leur niveau sonore d'exposition et sur les effets du bruit sur la santé,
	\item de connaitre et de délimiter les zones bruyantes et les zones calmes.\\
\end{itemize}

Cette directive se traduit notamment par la production de cartes de bruits stratégiques, pour chacune des 4 sources sonores ciblées, afin de déterminer les endroits où les niveaux sonores sont élevés. Un résumé des étapes permettant la réalisation des cartes de bruit du trafic routier est présenté en Figure \ref{fig:cartographie}.\\


\begin{figure}[h]
\centering
\includegraphics[width=.85\linewidth]{./figures/cartographie/cartographie.pdf}
\caption{Résumé des étapes menant aux cartes de bruit de trafic routier.}
\label{fig:cartographie}
\end{figure}

Dans un premier temps, plusieurs indicateurs en données d'entrée des modèles sont relevés \textit{in situ} :

\begin{itemize}
\item vitesses moyennes des véhicules sur les portions de routes principales,
\item débits de véhicules (nombre de véhicules par tranche horaire),
\item composition du trafic (nombre de véhicules légers et de poids lourds),
\item architecture et topographie de la ville (revêtement au sol),
\item conditions météorologiques (températures, vent).\\
\end{itemize}

\begin{figure}[t]
\centering
\subfigure[\label{fig:lden}]{\includegraphics[width=0.85\linewidth]{./figures/cartographie/Lden_ile_Nantes.PNG}}
\subfigure[\label{fig:ln}]{\includegraphics [width=0.7\linewidth]{./figures/cartographie/Ln_ile_Nantes.PNG}}
\caption{$L_{DEN}$ \subref{fig:lden} et $L_N$ \subref{fig:ln} de l'île de Nantes pour le trafic routier \cite{nantes_carte}.}
\label{fig:carto_nantes}
\end{figure}

Puis à partir d'un modèle d'émission sonore et de ces données d'entrée, il est possible d'établir la puissance émise par chaque véhicule, $s_{v}(f)$.
Enfin, l'influence de l'environnement $\delta(t)$ sur la répartition des niveaux sonores dans la ville est calculé à l'aide d'un logiciel numérique (Mithra, Immi, CadnAa \dots). Les logiciels accompagnés d'un Système d'Information Géographique (SIG) sont ceux qui offrent actuellement le plus de possibilités. Un SIG est un outil informatique conçu pour stocker, analyser et manipuler plusieurs types de données spatiales et géographiques comme l'architecture des villes ou le nombre d'habitants présents.
Son utilisation permet alors de connaitre plus facilement le nombre de citadins exposés à des forts niveaux sonores.
Par exemple, le logiciel OrbisGIS\footnote{\url{http://orbisgis.org/}}, destiné à représenter de données spatiales, permet la réalisation de cartes de bruit à l'aide de l'ajout d'un plugin, \textit{NoiseModelling}, développé par \cite{fortin}. \\

Les cartes de bruits produites résument les niveaux sonores équivalent pondérés $A$ sur 24h ($L_{DEN}$ pour \textit{Day-Evening-Night} (\textit{Jour-Soir-Nuit} en français)) et durant la nuit ($L_N$) :

\begin{equation}
L_{DEN} = 10\times\log_{10} \left(\frac{1}{24} \left(12\times10^{\frac{L_D}{10}}+4\times10^{\frac{L_E+5}{10}}+8\times10^{\frac{L_N+10}{10}} \right)\right)
\end{equation}

avec $L_D$, $L_E$ et $L_N$, les niveaux sonores équivalent pondéré $A$ pour les périodes respectives 6h-18h, 18h-22h, 22h-6h (horaires pouvant être changées suivant le rythme de vie des habitants de la zone considérée),

\begin{subequations}
\begin{align}
L_D &= 10\times\log_{10}\left(\frac{1}{T} \sum_{t = 1}^{T}10^{\frac{L_{D_t}}{10}}\right),\\
L_E &= 10\times\log_{10}\left(\frac{1}{T} \sum_{t = 1}^{T}10^{\frac{L_{E_t}}{10}}\right),\\
L_N &= 10\times\log_{10}\left(\frac{1}{T} \sum_{t = 1}^{T}10^{\frac{L_{N_t}}{10}}\right),
\end{align}
\end{subequations}

avec $L_{X_t}$, le niveau sonore dans la tranche horaire $t$. Les niveaux $L_E$ et $L_N$ sont majorés respectivement de 5 dB($A$) et de 10 dB($A$) afin de pénaliser les plages horaires où la gêne occasionnée par le trafic est plus importante. La Figure \ref{fig:carto_nantes} résume, par exemple, le $L_{DEN}$ et le $L_N$ pour le bruit du trafic routier dans un quartier de la ville de Nantes.
La réalisation de ces cartes permet alors d'estimer le nombre  de personnes touchées par ces niveaux sonores élevés et facilite ainsi la mise en oeuvre de travaux d'aménagement (construction de mur anti-bruit, changement de revêtement, diminution des vitesses \dots) si les niveaux sonores estimés sont trop élevés selon la réglementation en vigueur. L'utilisation de modèles prédictifs présente alors l'intérêt de pouvoir simuler l'effet de ces dispositifs sur les niveaux sonores émis et d'en mesurer l'impact \cite{murphy2011scenario,guedes2011influence}. Les cartes produites doivent ensuite être mises à jours tous les 5 ans.


\subsection{Vers la modélisation d'autres sources sonores ?}

Le trafic routier, aérien et ferroviaire et les ICPE sont actuellement les seules sources sonores à faire l'objet de cartes de bruit. Mais il est tout à fait possible d'ouvrir ces outils à d'autres sources sonores.
En effet, les cartes de bruits, si elles permettent de mieux identifier la présence de bruits en ville, ne permettent pas de représenter au mieux la perception des citadins de l'ESU \cite{brown2012review}. À travers plusieurs études perceptives \cite{lavandier2006contribution,hong2013designing}, plusieurs sources sonores comme la voix, le chant d'oiseaux, le bruit des fontaines ont montré avoir une influence dans la perception de l'ESU.
Il peut donc être intéressant et utile de savoir estimer leur présence dans les villes afin de permettre une représentation des ESU non plus juste à partir d'indicateurs physiques de sources sonores connotées négativement (comme le trafic routier) mais également des sources plus appréciées.
Certaines sources sonores ont déjà fait l'objet de modélisation (dans le cadre d'autres études ou applications) comme les oiseaux \cite{nemeth2013bird}, les voix \cite{hayne2011prediction} ou les fontaines \cite{watts2009measurement}, mais ne sont, pour l'instant pas assez étudiées pour offrir des modèles validés.
Une des difficultés est la localisation de certaines de ces sources dans l'espace urbain. Si les sources sonores fixes comme les fontaines ou les cloches d'une église sont faciles à localiser, d'autres, comme la foule et les oiseaux, sont plus difficiles à déterminer car plus mobiles et parcimonieuses. Dans \cite{aumond2018probabilistic}, c'est par une approche statistique que la position des sources est déterminée : les piétons sont ainsi plus susceptibles de se trouver sur des places ou le long des trottoirs, alors que pour les oiseaux, ce sont dans les parcs ou auprès des arbres qu'on les trouve. Il deviendrait alors envisageable de générer des cartes dites multi-sources d'une ville où une représentation des niveaux sonores émis par chaque source pourrait être générée et ainsi s'orienter vers des cartes de bruits perceptives.

%De plus, à partir des études perceptives réalisées, il deviendrait possible de lier les niveaux physiques prédit à des indicateurs perceptifs et ainsi générer des cartes de bruits perceptives qui serait lié à la perception du citadin de son environnement. Dans \cite{aumond2017modeling}, l'évaluation perceptive de citadins d'ESU est corrélé à des indicateurs physiques deux modèles d'agrément sonore sont proposés issus d'enregistrements sonores et des évaluations perceptives de citadins. L'un se base sur des grandeurs physiques tels que le niveau sonore fractile $L_{50}$ dans la bande de tiers d'octave de 1 kHz ainsi que la variation normalisée en temps et en fréquence des bandes de 500 Hz et de 4 kHz
%Dans \cite{aumond2017modeling}, cet agrément sonore, déterminée par une marche sonore (ou \textit{soundwalk}) auprès de citadins dans la ville de Paris, y est corrélée au niveau sonore globale de la scène sonore et au temps de présence du trafic, de la voix et des oiseaux. \\


\subsection{Limitations des modèles prédictifs}

L'utilisation de modèles d'émissions sonores présente donc plusieurs intérêts : prédiction des niveaux sonores dans une situation donnée, création de cartes de bruit de trafic et possibilité de tester différents scénarios d'aménagements. Leur utilisation s'est répandue et démocratisée en raison de l'introduction de la directive européenne. Les premières études réalisées suivant les recommandations de la directive ont permis de soulever plusieurs limites à ces méthodes comme celles liées au choix du modèle parmi ceux existants ou aux incertitudes liées aux estimations des données d'entrées.

\subsubsection{Limites liées à la modélisation de la source et de la propagation}

Ces modèles sont basés sur des mesures et des études physiques du trafic. Comme tout modèle simulant la réponse d'un système physique, il n'est pas possible d'en générer un universel qui puisse s'adapter à l'ensemble des scénarios possibles. Des simplifications sont ainsi réalisées, par exemple en classifiant les routes et le parc automobile en un nombre réduit de catégories ou bien en considérant (ou non), en plus des conditions atmosphériques homogènes, des conditions favorables à la propagation. Ce sont d'autant plus de simplifications qui, certes, facilite l'implémentation et l'utilisation des modèles mais qui sont aussi vecteurs d'incertitudes. De plus, ces catégorisations prennent le risque de mal prendre en compte certains cas limites qui ne correspondent pas spécifiquement à ceux définis.

Un second aspect, évoqué dans les parties \ref{part:modele_emission} et \ref{part:modele_propa} est celui de l'existence de plusieurs modèles d'émission et de propagation au sein de plusieurs pays européens. Avant même la mise en place de la directive, \cite{steele_critical_2001} en avait comparé plusieurs selon différents aspects (données d'entrée, type de cartographie, méthode de propagation des différents logiciels).
Parmi ces différents outils, l'auteur met en avant le problème, soulevé également par \cite{king_implementation_2011}, de la diversité des méthodes de calculs qui peuvent être employées : quelle méthode, parmi celles existantes, doit-être utilisée ?
Dans un premier temps, ce choix a été laissé libre par la directive européenne. Les premières cartes de bruits ont donc été établies sur des modèles différents : par exemple pour la même année, dans \cite{kliuvcininkas2006noise}, le modèle RLS-90 est employé pour calculer la carte de bruit dans le centre-ville de Kaunas, en Lituanie,  alors que dans \cite{murphy_environmental_2006}, la carte de bruit de trafic dans la ville de Dublin, en Irlande, est construite sur la base du modèle \bsc{Harmonoise}.
Une comparaison exhaustive de 8 modèles (FHWA, CoRTN, RLS-90, ASJ, \bsc{Harmonoise}/Imagine, Son Road, Nord 2000 et NMPB-Routes-2008) a également été réalisée par Garg et Maji \cite{garg_critical_2014} selon un plus grand nombre de critères (modélisation des sources sonores, vitesse des véhicules (constantes, accélération/décélération, intersection\dots), modèle de propagation, modélisation des effets de sol, effets météorologiques \dots). À travers leur comparaison, les auteurs relèvent ainsi les nombreuses différences notamment entre les modèles de propagation du son. Les auteurs de l'étude précisent tout de même qu'il est difficile de déterminer un \og meilleur \fg{} modèle par rapport aux autres, chacun ayant une approche différente.
Afin de résoudre ces problèmes, la méthode \bsc{Cnossos-Eu} \cite{CNOSSOS,kephalopoulos} a ainsi été développée, basée sur les méthodes déjà existantes en vue d'harmoniser la construction des cartes de bruit des villes à l'échelle européenne permettant plus facilement leur comparaison.

Toutefois, quelque soit le modèle choisi, la confrontation des niveaux sonores prédits face à des mesures faites en villes reste  à réaliser. L'ensemble de ces modèles d'émission et de propagation a été développé et validé à partir de mesures faites dans des conditions optimales. Toutefois, la comparaison entre les niveaux sonores calculés et mesurés reste délicate. Premièrement, les mesures présentent l'inconvénient d'être soumises à d'autres sources sonores qui ne sont pas liées au trafic et qui viennent donc fausser les estimations des niveaux sonores. De plus, il faut s'assurer, lors des mesures, que les données d'entrée des modèles correspondent bien aux conditions expérimentales, ce qui n'est pas facile.

Enfin, ces modèles dépendent des données d'entrée relevées \textit{in situ} s'exprimant sous la forme de moyennes et qui induisent donc des écarts-types qui se propagent dans les étapes suivantes du calcul. \cite{van_leeuwen_noise_2015} proposent un résumé et un schéma détaillé de la propagation de ces erreurs sur l'ensemble du modèle. La propagation de ces écarts-types a été étudiée dans \cite{probst2005uncertainties}, l'auteur y concède qu'estimer une incertitude $\sigma$ à un niveau sonore à un point donné est un processus long et difficile.

%\ml{a re formuler de manière rigoureuse}

\subsubsection{Limites liées à la simulation et à la représentation}

Cette tâche peut être très lourde en coût et en temps de calculs (de quelques heures à plusieurs jours).


Dans le cas de la cartographie de bruit, les modèles de sources $s_j(t)$ et de propagation $\delta_{ij}(t)$ sont implémentés dans des logiciels numériques pour déterminer, sur l'ensemble d'une ville ou d'un quartier, les niveaux sonores liés au trafic. Cette numérisation implique alors la discrétisation de l'environnement urbain. Cette discrétisation permet de décrire l'espace urbain $\Omega$ en un ensemble de points qui forme un maillage. L'environnement urbain est alors réduit en un espace discret $\Omega_m$. Le plus souvent, c'est un maillage régulier de 10 mètres par 10 mètres qui est choisi. À l'échelle d'une ville, c'est ainsi plusieurs millions de points qui peuvent être définis. Chaque source est alors rattachée à un point du maillage ($s_{j}(t)$ pour $j\in\Omega_m$). Le calcul consiste ensuite à estimer la propagation acoustique entre les sources et les points de ce maillage, selon les différents chemins de propagation possibles, pour chaque bande de fréquences ($\delta_{ij,\Omega_m}$). Plus le nombre de points dans $\Omega_{m}$ est élevé, plus le temps nécessaire pour calculer le niveau sonore augmente.  
Également, le nombre de chemin de propagation d'une onde acoustique est également une variable que contrôle l'utilisateur : d'une valeur théorique infinie, celle-ci se réduit à un nombre $K$. Ce nombre influe alors directement sur la quantité d'énergie finale présente en un point $i$. L'équation \ref{eq:propagation} devient alors : 

\begin{equation}
 M_{i,\Omega_m}(t)= \sum_{j = 1}^{N} \sum_{k = 1}^{K} s_{j}(t-\tau_{ijk}) \delta_{ijk}
 \end{equation}


avec $\lbrace i,~j \rbrace \in \Omega_m$. Le choix de ces paramètres revient donc à un compromis à faire, par l'utilisateur, entre la précision des résultats souhaitée et le temps de calcul alloué. 
L'étape de discrétisation du milieu, si elle simplifie la forme de l'environnement urbain, présente l'inconvénient de ne pas prendre en compte les multiples variations géométriques des façades ou bien l'ensemble des petits mobiliers urbains qui ont un rôle dans la diffusion du son.
De plus, en raison de la discrétisation du milieu, des méthodes d'interpolation (méthode linéaire, de krigeage) sont utilisées pour calculer les niveaux sonores entre ces points, ce qui est également source d'approximation pouvant mener à de mauvaises interprétations \cite{van_leeuwen_noise_2015}.

Enfin, une fois la carte générée, on peut noter que seulement 2 niveaux sonores par source de trafic sont obtenus, $L_{DEN}$ et $L_N$, et mis à jour tous les 5 ans seulement. C'est donc une information statique et restreinte que la directive européenne impose. Cependant, le trafic routier, ferroviaire et aérien varient aussi bien à l'échelle de l'année, d'une journée ou même d'une heure \cite{lv2015traffic}. S'il parait envisageable de calculer des cartes de bruit pour, par exemple, chaque heure de la journée, cela reste une opération très longue et coûteuse à réaliser.
L'utilisation de modèles dynamiques de trafic, couplés aux modèles d'émissions sonores \cite{can2010traffic}, n'est actuellement pas destiné à la cartographie des ESU mais est une piste envisageable pour modéliser l'impact de l'écoulement du trafic routier et de la cinématique des véhicules sur l'ESU.

En conclusion, l'utilisation de modèles prédictifs est une approche utile pour réaliser une première estimation de la répartition du bruit de trafic en ville. Si elle présente des avantages (estimation d'un niveau physique à l'échelle de la ville, possibilité de tester des scénarios d'aménagement), elle présente plusieurs limites :

\begin{itemize}
\item le nombres de type de sources est, pour l'instant, trop restreint (trafic routier, aérien, ferroviaire, ICPE) et ne permet pas de considérer l'ensemble des ESU et la perception qu'en ont les citadins,
\item l'utilisation d'un modèle d'émission et de propagation et la calcul numérique entraine de nombreuses incertitudes qui sont difficiles à estimer,
\item l'accès à l'évolution temporelle des sources sonores n'est pas possible.
\end{itemize}

Afin de considérer l'ensemble des sources et des évènements sonores présents en ville et de compléter les estimations générées par les modèles prédictifs, une autre approche est envisagée, basée sur la réalisation de mesures et d'enregistrements sonores réalisés directement dans la ville.

\section{Utilisation de mesures acoustiques}

À la différence des modèles prédictifs qui déterminent l'émission sonore des sources $s_j(t)$ et leur propagation $\delta_{ij}(t)$ pour en déterminer le niveau sonore en un point $i$, la réalisation de mesures donne directement accès à la mixture globale $M_{i}(t)$. Les limites liées à la modélisation des sources, de la propagation du son et de l'ensemble des environnements urbains dans toute leur complexité n'interviennent alors plus dans les mesures faites \textit{in situ}.

De nombreuses études ont déjà été menées en ville en faisant intervenir des mesures pour des durées plus ou moins longues (de quelques jours \cite{romeu2011street} à plusieurs années \cite{gaja2003sampling}) avec des microphones de mesures de hautes qualités. Dans une première étude \cite{zannin2002environmental} a réalisé une série de mesures sur près de 1000 positions dans la ville de Curitiba, au Brésil, pour étudier l'ESU dans sa globalité afin d'évaluer l'exposition aux bruits dans les différents quartier de la ville, puis a réduit la surface d'étude dans \cite{zannin_characterization_2013} où 58 points de mesures sont déployés dans le campus universitaire de la ville.
Dans \cite{Mioduszewski}, 40 microphones sont placés isolément à travers la ville de Gdansk en Pologne pour une durée d'un an afin d'y mesurer le niveau sonore du trafic et de valider la cartographie de bruit.

Il y est donc tout à fait possible d'étudier les ESU et les sources qui les composent à l'aide d'enregistrements et de mesures acoustiques faites directement dans la ville. Plusieurs approches sont possibles aux travers de mesures faites par des réseaux de capteurs, par des mesures mobiles et participatives, chacune ayant déjà fait l'objet d'études et d'applications. Celles-ci sont, dans un premier temps, successivement présentées, pour ensuite être interrogées au regard de notre problématique.

\subsection{Déploiement de réseaux de capteurs fixes}

Les approches évoquées en introduction de cette partie ont consisté en des mesures limitées dans le temps, or la pérennisation de l'installation de capteurs en ville est de plus en plus envisagée à l'heure de l'émergence de l'\textit{IoT} (\textit{Internet of Things} ou l'internet des objets en anglais) \cite{zanella2014internet} et de la ville intelligente (\textit{Smart City} en anglais) \cite{chourabi2012understanding}. De nombreuses villes s'équipent actuellement en réseaux de différents types de capteurs disséminés dans le milieu urbain afin d'en contrôler, en temps réel, de nombreux aspects : distribution d'énergie, gestion des transports, de l'eau ou des déchets. L'objectif étant alors d'optimiser le fonctionnement de la ville afin d'améliorer la qualité de vie des citadins.
Ces réseaux sont constitués d'un ensemble de capteurs fixes alimentés et connectés à un réseau de télécommunication. Leurs mesures sont alors stockées par des serveurs pour ensuite être post-traitées. L'intérêt principal d'un tel réseau est la possibilité d'avoir accès à des variations à long-terme des niveaux sonores à l'emplacement des microphones. Un schéma d'un tel réseau est résumé en Figure \ref{fig:reseau_capteur}.

\begin{figure}[t]
\centering
\includegraphics[width=0.8\linewidth]{./figures/cartographie/reseau_mesure.png}
\caption[Schéma d'un réseau de capteurs fixes]{Schéma d'un réseau de capteurs fixes\protect\footnotemark}
\label{fig:reseau_capteur}
\end{figure}

\footnotetext{\url{http://cense.ifsttar.fr/}}

Un premier réseau de capteurs développé depuis plusieurs années est celui de la ville de Paris, géré par BruitParif, à travers le projet RUMEUR \cite{mietlicki2012innovative}, où des réseaux de microphones sont déployés afin d'évaluer l'ESU en région parisienne. Le réseau comprend des stations fixes mais intègre aussi des campagnes de mesures plus courtes (de plusieurs heures à plusieurs années) pour voir l'impact acoustique d'un évènement (journée sans voiture) ou d'un aménagement (modifications de voirie). Un site internet\footnote{\url{http://rumeur.bruitparif.fr/}} est mis à disposition pour avoir un aperçu complet des mesures réalisées sur les nombreux emplacements.
En parallèle, plusieurs études se sont intéressées à la prise en compte de mesures pour caractériser le bruit du trafic comme dans \cite{makarewicz_empirical_2011} qui propose une estimation des niveaux sonores $L_{DEN}$ et $L_N$ directement à partir des mesures et dont l'incertitude est estimée en fonction de la durée d'acquisition des mesures. Dans \cite{wei_dynamic_2016}, une technique d'assimilation de données est proposée afin de réaliser des cartes de bruit dynamiques basées sur des mesures. Les niveaux de puissances et les effets de propagation de chaque source sont ici modifiés par des termes correctifs obtenus en minimisant l'erreur au carré entre les niveaux prédits et les niveaux mesurés.
L'arrivée, depuis plusieurs années, sur le marché de capteurs acoustiques à bas coût (moins de 10 € pour un capteur microphone MEMS) \cite{van2010use} rend possible le déploiement d'un plus grand nombre de capteurs acoustiques à travers les villes. La mise en place de tels réseaux donne lieu à plusieurs projets où les questions qui émanent de tels projets sont étudiées (nombre de capteurs installés, surface couverte par ce réseau, mesures fixes ou mobiles \dots comme le projet européen DYNAMAP \footnote{\url{http://www.life-dynamap.eu/}} \cite{dynamap_2016}. DYNAMAP a pour objectif de développer un système de cartographie de bruit dynamique basé sur des réseaux de capteurs à bas coûts installés en ville. Une application de ce projet a déjà été réalisée dans deux villes tests, Milan et Rome \cite{bellucci_life_2017}.
Le principe de leur approche est d'ajuster les cartes de bruits simulées à partir des différences obtenues entre les niveaux sonores mesurés aux stations et les niveaux sonores calculés à ce même point par les modèles prédictifs. Pour limiter le coût d'un tel déploiement, le nombre de microphones est réduit en les installant à des emplacements spécifiques représentatifs des différents scénarios possibles de trafic routier (homogénéité du trafic, type de revêtement) \cite{zambon2017life}.
Le projet SONYC \footnote{\url{https://wp.nyu.edu/sonyc/}} à New-York dédie son réseau de capteurs à la surveillance de la pollution sonore et au développement d'outils de traitement du signal afin de décrire l'ESU par l'étude des sources présentes \cite{mydlarz2017noise}.
Enfin, le projet CENSE\footnote{\url{http://cense.ifsttar.fr/}} vise à développer un réseau de capteurs dans la ville test de Lorient afin là encore d'améliorer la cartographie du bruit de trafic en agrégeant les données simulées des niveaux sonores du trafic avec les mesures réalisées en ville par ce réseau. L'approche est différente de DYNAMAP, puisqu'ici l'étude se restreint à l'échelle de plusieurs quartiers de la ville afin d'avoir un réseau de capteurs dense. La mise à jour des cartes est faite à l'aide de techniques d'assimilation de données en vue de compléter les cartes de bruits prédites avec les mesures.
Ces méthodes d'assimiliation sont notamment utilisées dans le domaine des sciences géophysiques et consistent à modifier une estimation émises par un modèle prédictif à partir de données mesurées \cite{wu2008comparison}.
Le projet s'intéresse également à la perception des citadins des ESU aux travers de questionnaires qui leur sont envoyés et de mesures réalisées par ce réseau de capteurs.

Toutefois, l'installation de tels réseaux de capteurs nécessite de gérer de nombreuses problématiques techniques comme la disposition des microphones, leur maintenance, la transmission et le stockage des mesures, leur alimentation électrique\dots{} Une des premières limites techniques est la performance individuelle des capteurs. En effet, la réduction de leur coût s'est faite en diminuant leur performance  individuelle (dynamique énergétique et fréquentielle) \cite{mydlarz2015design}. Il est donc nécessaire de caractériser chacun des microphones en vue de connaitre leurs performances et leur limites. Une des points cruciaux est celui de la position des microphones des mesures (quelle hauteur ? quelle position par rapport à des sources sonores qui seraient dignes d'intérêt ?) et de la surface couverte par ces mesures. Un réseau distribué selon un maillage dense permettra une bonne représentation de l'espace mais coûtera cher à installer et à maintenir alors qu'une faible densité de capteurs sera moins onéreuse mais apportera moins d'information et nécessitera des interpolations entre les mesures, ce qui reste une source d'incertitudes. Toutefois, la réalisation de mesures acoustique en ville n'est pas nécessairement obligée d'être réalisée via des réseaux de capteurs fixes. D'autres pistes sont également explorées.

%\ml{On se perd un peu la, les questions cruciales c'est bien, mais je pense qu'il faut que tu ramène a ta problématique, fais moi penser a te parler du discours en arete de poisson}

\subsection{Mesures mobiles}

En parallèle aux réseaux fixes, la mesure mobile est une voie envisagée. Elle consiste à réaliser des mesures acoustiques en plaçant le microphone sur un support mobile (piéton, cycliste, voiture, bus). Ce type de mesure correspond aux mesures faites par des professionnels avec du matériel de haute qualité. 
%Les mesures participatives réalisées par des citadins avec leur smartphones (voir partie \ref{part:mesures_participatives}), et les questions liées à la calibration des microphones notamment, n'interviennent donc ici pas.

Dans le cadre des études des ESU et du bruit de trafic, l'avantage de cette méthode par rapports aux capteurs fixes est sa capacité à pouvoir couvrir plus facilement une plus grande surface urbaine à moindre coût. Les mesures mobiles sous-entendent deux manières d'être réalisées : soit le microphone réalise sa mesure sur un support mobile qui se déplace en même temps \cite{alsina-pages_design_2016}. Dans ce cas, un traitement du signal doit être effectué pour prendre en compte le bruit émis par ce support. 
Le microphone peut également être placé sur un support mobile afin de le déplacer pour faire ensuite des mesures fixes \cite{manvell2004sadmam} ce qui simplifie la tâche mais nécessite plus de temps pour couvrir une surface similaire par rapport aux mesures faites sur un support mobile. 
L'inconvénient de ces méthodes est qu'elles ne permettent pas la réalisation de mesures à long terme et donc d'estimer l'évolution temporelle des niveaux sonores en un point donné au cours du temps.
Ainsi, plusieurs travaux se sont intéressés à l'agrégation des mesures mobiles à des mesures réalisées par des stations fixes.
\cite{morillas2014uncertainty} s'intéressent aux incertitudes sur l'estimation des niveaux sonores estimés suivant le nombre de points ou le nombre de jours de mesures. Dans \cite{can_measurement_2014}, la prise en compte de mesures mobiles pour compléter des stations fixes est comparée à des méthodes d'interpolation (méthode de Kriging , pondération inverse de la distance). Il en résulte que l'apport des mesures mobiles diminue l'erreur produite par rapport aux méthodes d'interpolation en cela qu'elles permettent d'apporter plus d'informations quant aux variations spatiales du niveau sonore (rues calmes peu fréquentées, rues très passantes, aux abords d'intersections\dots), ce que ne permet pas une méthode d'interpolation numérique \cite{aumond2018kriging}.

La réalisation de mesures mobiles est notamment très courante pour des études perceptives de l'ESU par des citadins lors de \textit{soundwalks} (\textit{marches sonores} en français). Cette méthode consiste à réaliser un parcours en ville et à soumettre, à un panel d'auditeur, un questionnaire sur les sons qui les entourent. L'enregistrement de l'ESU durant cette marche permet alors de corréler leurs réponses et leur perception avec des indicateurs (niveau sonore en dB($A$), présence de certaines sources sonores \dots{}) extraits de ces enregistrements \cite{brocolini_measurements_2013, hong2013designing}). Dans \cite{aumond2017modeling}, l'évaluation perceptive d'ESU par des citadins est corrélée à des indicateurs physiques tels que le niveau sonore fractile $L_{50}$ dans la bande de tiers d'octave de 1 kHz ainsi que la variation normalisée en temps et en fréquence des bandes de 500 Hz et de 4 kHz.
Enfin, les mesures mobiles peuvent servir à des études plus ponctuelles par exemple en vue de classer les ESU selon différents indicateurs physiques comme dans \cite{rychtarikova2013soundscape}, où 370 enregistrements de 15 à 20 minutes réalisés dans 4 villes de Belgique ont été réalisés lors de \textit{marche sonore}, et dans \cite{can_describing_2015}, où des enregistrements mobiles ont été réalisés dans la ville de Marseille.


\begin{figure}[t]
\begin{center}
    \begin{minipage}[t]{0.3\textwidth}
        \centering
        \includegraphics[width=0.9\textwidth]{./figures/autres/noiseCapture1.png}
    \end{minipage}
    \begin{minipage}[t]{0.3\textwidth}
        \centering
        \includegraphics[width=0.9\textwidth]{./figures/autres/noiseCapture3.png}
    \end{minipage}
    \begin{minipage}[t]{0.3\textwidth}
        \centering
        \includegraphics[width=0.9\textwidth]{./figures/autres/noiseCapture2.png}
    \end{minipage}
    \caption{Captures d'écran de l'application \textit{NoiseCapture}}
\end{center}
    \label{fig:noiseCapture}
\end{figure}

\subsection{Mesures participatives}\label{part:mesures_participatives}

Une dernière voie sollicite la participation des citadins. Ces mesures participatives peuvent se réaliser en équipant les citadins de dispositifs spécifiques \cite{delaitre2014influence} ou bien à partir d'applications développées pour smartphones qu'ils peuvent télécharger eux-mêmes. Profitant de la démocratisation de ces appareils et de l'augmentation de leurs performances, ces applications leurs permettent d'avoir un dispositif suffisamment performant pour mesurer les niveaux sonores autour d'eux. Cette approche permet surtout d'obtenir un plus grand nombre de mesures qui ont le plus souvent une distribution spatiale et temporelle plus aléatoire mais qui sont aussi effectuées moins régulièrement. L'utilisation de ces mesures est toutefois encore sujette à caution puisque de nombreux problèmes sont encore à résoudre comme la calibration et la prise en compte des performances des microphones dans les faibles et forts niveaux sonores \cite{aumond2017study} ou bien encore la qualité de la réalisation de la mesure faite par l'utilisateur\dots{} Dans ce cas, le traitement statistique des résultats est primordial afin de détecter les mesures incongrues et ne pas les considérer \cite{guillaume2016noise}. Plusieurs applications ont été dévelopées comme \textit{NoiseSpy} \cite{kanjo_noisespy_2010} ou \textit{Ambicity} \cite{ventura2017estimation}. On peut également relever dans le projet \textit{Noise Planet}\footnote{\url{http://noise-planet.org}} l'application pour smartphone, \textit{NoiseCapture} \cite{guillaume2016noise} (Figure \ref{fig:noiseCapture}), qui permet, là aussi, à l'utilisateur d'évaluer les niveaux sonores l'entourant tout en ayant la possibilité de décrire, à l'aide de mots-clés prédéfinis, les sons présents et l'ambiance sonore de la scène. La géo-localisation et les mesures sont ensuite collectées puis traitées pour produire des cartes de bruits, publiées en ligne (voir Figure \ref{fig:carte_noiseModelling}). 
Les mesures réalisées par ces dispositifs permettent d'évaluer les ESU toutes sources confondues sans considérer leur influence individuelle. 
En outre, un des intérêts de ces applications, en plus de sensibiliser le citadin à son environnement sonore, est de le rendre producteur et utilisateur de données environnementales. Les informations récoltées sur ces trajets permettent de calculer son exposition au bruit ou bien de le guider vers des itinéraires secondaires où son exposition au bruit serait plus faible \cite{aumond2016sound}.\\


\begin{figure}[t]
\centering
\includegraphics[width=0.7\linewidth]{./figures/cartographie/noise_modelling.PNG}
\caption{Carte de l'ESU de l'île de Nantes mesurée par l'application \textit{NoiseCapture}  (relevée le 22/03/2018)}
\label{fig:carte_noiseModelling}
\end{figure}

\subsection{Intérêts et limites des mesures faites en villes}

L'ensemble de ces dispositifs permet d'aborder l'ESU par une nouvelle approche en s'affranchissant des limitation liées à la modélisation des sources et de leur propagation dans l'environnement urbain. 
Si les mesures participatives permettent d'estimer des niveaux sonores toutes sources confondues, les réseaux de capteurs et les mesures mobiles permettent une meilleure description dynamique et spatiale des ESU qui sont impossibles à obtenir avec les modèles prédictifs. Leur utilisation offrent donc une représentation globale des ESU et ouvrent donc la voie vers de nombreuses applications :

\begin{itemize}
\item estimation des niveaux sonores du trafic et amélioration de la cartographie de bruit,
\item identification et détection des sources sonores spécifiques, 
\item évaluation et classification plus complète des ESU au travers d'indicateurs physiques,
\item et représentation possible des ESU selon la perception des citadins.
\end{itemize}

Ces méthodes ne sont toutefois pas exemptes de défauts.
Les réseaux de capteurs sont des systèmes complexes à gérer par leur installation et leur entretien. De plus, la question de l'interpolation entre les points de mesures reste une source d'approximation.
À l'inverse, les mesures mobiles permettent de mieux estimer les variations spatiales aux dépends des variations à long-terme. Mais elles restent très couteuses en temps à réaliser à l'échelle d'une ville. Enfin les mesures participatives présentent de nombreuses incertitudes quant à la qualité de la mesure dues aux performances des capteurs des smartphones ou de la mesure réalisée qui nécessitent un traitement du signal important.

Toutefois, s'il existe déjà des outils destinés à évaluer les ambiances sonores des l'ESU ou à lier leur perception à des indicateurs physiques, la prédiction des niveaux de bruits du trafic ou la description des ESU selon les différentes sources sonores présentes nécessite de disposer d'outils de traitements du signal adaptés afin d'y extraire leur contributions. 
Hors, l'ESU est un milieu complexe, composé d'une multitude de sources variées (trafic routier, voix, oiseaux, klaxon, bruit de pas\dots) dont leurs allures temporelles (parfois brèves pour le retentissement d'un klaxon ou longues pour le passage d'une voiture) et fréquentielles (dans les basses fréquences pour le trafic, dans les hautes fréquences pour le sifflement des oiseaux) diffèrent, voir Figure \ref{fig:sourceUrbain}.
L'ensemble de ces sources est aussi susceptible d'être généré simultanément. La création d'outils adaptés à cet environnement n'est donc pas triviale.

Des outil d'identification ou de détection ont déjà été  développés pour des sons environnementaux \cite{mesaros_sound_2015, chachada2014environmental, cakir2015polyphonic}, mais la tâche de séparation de tels signaux au sein de mixtures sonores urbaines reste quant à elle, pour l'instant peu étudiée. 
Développer de tels outils serait pourtant nécessaire et utile, pour l'amélioration de la cartographie du bruit de trafic par exemple. Car, s'il existe des endroits où celui-ci est prépondérant sur les autres sources sonores (périphérique, grand boulevard) et donc que son niveau sonore peut être estimé facilement, il existe de nombreux autres lieux (dans des rues calmes, au niveau de parc) où ce sont d'autres sources sonores qui sont majoritairement présentes (voix, oiseaux \dots). Ne pas réussir à isoler la contribution du trafic routier des autres sources dans ces environnements risque alors de mener à de mauvaises estimations de son niveau sonore et de son temps de présence. Dans le cas d'étude perceptive, il deviendrait possible de relier l'évaluation perceptive des citadins réalisé lors de marches sonores au niveau sonore de certaines sources et non plus juste à partir de leur temps de présence. Ce constat peut s'étendre à chaque source sonore qu'on souhaiteraient écouter.\\

Il est donc nécessaire de générer un outil adapté à l'ESU afin de pouvoir extraire les contributions et les niveaux sonores des sources présentes en villes à partir de mesures et d'enregistrements. En conséquence, les travaux de cette thèse cherchent à répondre à ces questions :

\begin{itemize}
\item \textbf{Comment déterminer le niveau sonore du trafic routier en ville ? est-il possible de déterminer d'autres sources sonores ?}
\item \textbf{Quelles sont les méthodes disponibles pour réaliser cette tâche ? Quel est l'outil le plus adapté à notre  environnement d'étude parmi ces méthodes ?}
\item \textbf{Quel protocole expérimental mettre en place pour tester et valider les performances de cet outil ?}
\end{itemize}

\begin{figure}[t]
\centering
\subfigure[\label{fig:sourceUrb1}]{\includegraphics[width=0.4\linewidth]{./figures/autres/sourcesUrbainesCar.pdf}}
\subfigure[\label{fig:sourceUrb2}]{\includegraphics [width=0.4\linewidth]{./figures/autres/sourcesUrbainesBird.pdf}}
\subfigure[\label{fig:sourceUrb3}]{\includegraphics [width=0.4\linewidth]{./figures/autres/sourcesUrbainesCarHorn.pdf}}
\subfigure[\label{fig:sourceUrb4}]{\includegraphics [width=0.4\linewidth]{./figures/autres/sourcesUrbainesFootStep.pdf}}
\caption{Spectrogrammes d'un passage d'une voiture \subref{fig:sourceUrb1}, d'un sifflement d'oiseaux \subref{fig:sourceUrb2}, d'un klaxon \subref{fig:sourceUrb3} et d'un bruit de pas \subref{fig:sourceUrb4}.}
\label{fig:sourceUrbain}
\end{figure}

\section{Estimation du niveau sonore du trafic routier} \label{part:cachier_charges}

Étant la source principale de bruit en ville ainsi que la plus gênante \cite{noauthor_noise_nodate}, le trafic routier sera la source d'intérêt qui sera principalement étudiée dans ce document. Le principe général de la méthode proposée est résumé en Figure \ref{fig:estimateur0} : à partir d'un enregistrement audio monophonique (réalisé en format wav et de fréquence d'échantillonnage 44,1 kHz), un outil, appelé \textit{estimateur}, détermine le niveau sonore estimé du trafic routier.
L'objectif est donc de construire cet estimateur et un protocole expérimental adéquat afin de générer l'erreur d'estimation la plus faible possible du niveau sonore du trafic.\\

\begin{figure}[ht]
\centering
\includegraphics[width=0.7\linewidth]{./figures/NMF/bloc_diagram_estimateur0.pdf}
\caption{Vue synthétique de l'approche proposée.}
\label{fig:estimateur0}
\end{figure}

Parmi les travaux qui s'intéressent au trafic routier à partir d'enregistrements sonores d'ESU,  des outils de reconnaissance du bruit \cite{defreville_automatic_2006}, d'estimation du débit véhicule \cite{torija2012using}, de détection d'accidents de la route \cite{harlow2001automated} ou bien d'estimation des trajectoires  basé sur l'antennerie de microphones \cite{leiba2017large} ont déjà été développés.
La détermination du niveau sonore du trafic routier en tenant compte des autres sources sonores présentes parmi ces enregistrements a pour l'instant été très peu étudiée. On peut citer les travaux réalisés récemment au sein du projet DYNAMAP \cite{socoro2017anomalous}. Leur approche consiste à entrainer une méthode de détection en vue d'estimer les trames temporelles où la classe de son \textit{trafic} n'est pas présente afin de les rejeter lors de l'estimation des niveaux sonores. Une limite de cette approche est la possibilité de considérer des faux-positifs et de ne pas répondre à la question du recouvrement avec les autres sources sonores.

\section{Méthode proposée}

Ici, l'approche choisie est différente : l'estimateur du niveau sonore du trafic routier s'appuie sur une méthode de séparation de sources (voir chapitre \ref{chap:methode_separation_source}) afin d'extraire l'intégralité de la composante \textit{trafic} des enregistrements audio parmi les autres sources sonores présentes (voir Figure \ref{fig:separation_source}). L'un des intérêts est notamment la prise en compte naturelle du phénomène de recouvrement temporel. 
Cette méthode, en isolant la contribution \textit{trafic} d'une mixture sonore permet de déterminer plusieurs de ces caractéristiques dont le niveau sonore. Cette estimation dépend de deux grandeurs : son unité et sa durée d'acquisition $\delta_t$.

\begin{figure}[ht]
\centering
\includegraphics[width=0.9\linewidth]{./figures/autres/schema_source_separation_FR.pdf}
\caption{Diagramme en bloc de l'approche par séparation de sources d'une mixture sonore mélangeant une composante \textit{trafic} avec des sons de klaxons.}
\label{fig:separation_source}
\end{figure}

Ce niveau sonore du trafic peut s'exprimer soit en grandeur linéaire (équivalent à la pression $p_{rms}$ en Pascal), soit en grandeur logarithmique (en dB), tel que

\begin{equation}
L_{p} = 20 \times \log_{10} \left( \frac{p_{rms}}{2\times 10^{-5}} \right)
\end{equation}

avec $p_0 = 2\times 10^{-5}$, la pression acoustique de référence. Pour ces travaux, le choix est fait d'exprimer les niveaux sonores en dB afin d'éviter de manipuler des valeurs dont les ordres de grandeurs seraient trop différentes et de calibrer l'échelle de grandeur. Enfin, le dB étant une grandeur couramment utilisée dans le domaine de l'acoustique urbaine, c'en est une représentation adaptée. La pondération $A$ n'est pas considérée afin de se focaliser sur une estimation physique du niveau \textit{trafic}. De plus, celle-ci diminue la présence des basses fréquences où se trouve les composantes du trafic. 

La durée d'intégration $\delta_t$ sert ensuite à déterminer un niveau sonore équivalent, $L_{eq}$, sur une plage de temps,

\begin{equation}
L_{eq} = 10 \times \log_{10}\left(\frac{1}{\delta_t} \int_t^{t+\delta_t} 10^{Lp/10} dt \right).
\end{equation}

Quelle valeur choisir pour $\delta_t$ ? Choisit-on un niveau sonore exprimé toutes les 125 ms ? toutes les secondes ? toutes les minutes ? Aux vues de l'utilisation de cet estimateur qui est faite (meilleur estimation de la contribution du trafic, amélioration des cartes de bruits en ville), le choix est fait de considérer une temps d'intégration long de l'ordre de la minute. Il n'est en effet pas pertinent de choisir un pas d'intégration trop faible (de 125 ms par exemple) qui serait alors une précision trop exigeante par rapport aux applications souhaitées. 
Pour le premier corpus (voir partie \ref{part:corpus_ambiance}), la durée des scènes sonores est la même (30 secondes). La durée d'intégration $\delta_t$  sera alors définie selon leur longueur ($\delta_t$ = 30 secondes). Puis, dans le second corpus (partie \ref{part:corpus_grafic}), la durée des scènes sonores étant variables, on choisira un temps d'intégration ramené à la minute ($\delta_t$ = 60 secondes).\\

Il faut ensuite pouvoir comparer le niveau sonore estimé du trafic, $\tilde{L}_{eq,trafic, \delta_t}$, à sa valeur exacte, , $L_{eq,trafic, \delta_t}$, afin de pouvoir évaluer la justesse de l'estimateur. En se basant sur des enregistrements sonores, dans le cas où il n'y a que du trafic, l'estimation fournie par la méthode peut être facilement comparée mais quid des scènes sonores où le trafic n'est pas prépondérant et est capté avec d'autres sources sonores ? La valeur exacte du trafic est l'inconnue qu'on cherche justement à déterminer. Sans cette référence, il est impossible de comparer la valeur estimée du trafic et ainsi la validité et les performances de l'estimateur.
Le choix est donc fait d'utiliser non pas des enregistrements sonores mais des scènes sonores issues d'un processus de simulation où un contrôle complet des classes sonores présentes, ainsi que de leur niveau sonore, est alors possible. Grâce à ce procédé, la valeur exacte, $L_{eq,trafic}$, est ainsi obtenue. La Figure \ref{fig:diagramBlocProtocol} résume le schéma global du procédé suivi.

\begin{figure}[ht]
\centering
\includegraphics[width=0.7\linewidth]{./figures/NMF/Bloc_diagram_estimateur_FR.pdf}
\caption{Diagramme bloc du protocole expérimental.}
\label{fig:diagramBlocProtocol}
\end{figure}

Les niveaux sonores exacts et estimés sont ensuite comparés à travers un calcul de métrique. Le choix du calcul de métrique est un choix important puisqu'il conditionne les conclusions qui seront faites. 
Dans le cadre de la séparation de source, plusieurs métriques ont été développées afin d'harmoniser et de faciliter la comparaison entre les performances des méthodes notamment dans \cite{vincent2006performance}.
Ici, puisque la performance de l'estimateur est liée à son erreur d'estimation du niveau sonore du trafic, c'est un indicateur d'écart qui est choisi. Parmi ces métriques qui composent cette classe (somme des carrés des résidus, la racine de l'erreur quadratique moyenne ($RMSE$ pour \textit{Root Mean Square Error} en anglais), l'erreur  absolue moyenne en pourcentage ($MAPE$)  \dots), c'est l'erreur absolue moyenne, $MAE$ (\textit{Mean Absolute Error}) qui est retenue :

\begin{equation}\label{eq:mae}
MAE = \frac{\sum_{i = 1}^{M} \vert L_{eq, trafic, \delta t}^i - \tilde{L}_{eq, trafic, \delta t}^i \vert}{M}.
\end{equation}

Contrairement à l'erreur RMSE, qui revient à la racine carrée de la moyenne du carré des différences entre les données observées et réelles, qui pénalise plus les valeurs qui dévie fortement, l'erreur $MAE$ présente l'intérêt de considérer un poids identique entre chaque différence et ainsi de gagner en interprétabilité. 
Le choix de confronter des grandeurs logarithmiques (les niveaux sonores en dB) à un opérateur arithmétique (l'opérateur moyenne) peut paraitre discutable mais est justifié par la stabilité qu'il possède contrairement aux erreurs $RMSE$ ou $MAPE$ qui sont plus sensibles aux anomalies. De plus, l'usage d'opérateurs statistiques sur des grandeurs logarithmiques est courantes dans la communauté d'acoustique environnementale \cite{aumond2018kriging,morillas2014uncertainty}.
Par la réalisation de moyenne arithmétique simple sur des grandeurs des dB, l'erreur $MAE$ présente l'avantage comparer équitablement les performances des estimateurs dans des cas variés. En effet, pour un estimateur qui estime le niveau sonore du trafic moitié moins, que le niveau soit fort (94 dB) ou faible (54 dB). L'erreur, dans les deux cas, sera de 3 dB. Dans un cas linéaire, ces erreurs seraient respectivement de 0,5 Pa et de 0,05 Pa. La métrique serait alors plus impactée par les erreurs dans les forts niveaux que dans les faibles. Pour mieux équilibrer cette disparité, l'utilisation de cette métrique à des grandeurs en dB est cohérente et adaptée aux cas d'études présents. Toutefois, la valeur fournie par l'erreur $MAE$ n'est pas considérée comme une erreur s'exprimant en dB mais comme un indice de performance sans dimension.\\

À partir de cette proposition, il reste maintenant à :
\begin{itemize}
\item \textbf{déterminer quelle méthode de séparation de sources choisir comme estimateur, }
\item \textbf{construire correctement des corpus de scènes sonores urbaines pour tester l'efficacité de l'estimateur retenu.}
\end{itemize}



%
%%\bibliographystyle{unsrt}
%%\bibliography{../bibliographie}
%%
%%\end{document}

%
%Dans le domaine fréquentiel, le produit de convolution s'exprime sous la forme :
%
%\begin{equation}
%\hat{M}_{i}(f) = \hat{s}_j(f)\hat{\delta}_{ij}(f)e^{i2\pi f \tau_{ij}}
%\end{equation}
%
%avec $\hat{g}(f)$, la transformée de Fourier de la fonction $g(t)$, $\hat{g}(f) = \frac{1}{\sqrt{2\pi}}\int_{-\infty}^{+\infty}g(t)e^{-i2\pi ft} dt$. $\hat{\delta}_{ij}(f)$ s'apparente alors à un filtre de propagation.
